\documentclass[addpoints, answers]{exam} % добавить или удалить answers в скобках, чтобы показать ответы
% \documentclass[addpoints, answers]{exam} % добавить или удалить answers в скобках, чтобы показать ответы


\usepackage[utf8]{inputenc}
\usepackage[russian]{babel}
%\usepackage[OT1]{fontenc}
\usepackage{booktabs}
\usepackage{amsmath}
\usepackage{tikz}
\usepackage{amsfonts}
\usepackage{amssymb}
\usepackage[left=2cm,right=2cm,top=2cm,bottom=2cm]{geometry}
\DeclareMathOperator{\E}{\mathbb{E}}
\DeclareMathOperator{\Var}{\mathbb{V}\mathrm{ar}}
\DeclareMathOperator{\Cov}{\mathbb{C}\mathrm{ov}}
\DeclareMathOperator{\sgn}{\rm sgn}
\let\P\relax
\DeclareMathOperator{\P}{\mathbb{P}}
\newcommand{\cN}{\mathcal{N}}
\newcommand{\RR}{\mathbb{R}}
\newcommand{\hbeta}{\hat{\beta}}

\usepackage{floatrow}
%\newfloatcommand{capbtabbox}{table}[][\FBwidth]

\begin{document}

\pagestyle{headandfoot}
\runningheadrule
\firstpageheader{НИУ-ВШЭ}{Высшая математика, стр. \thepage\ из \numpages}{олимпиада 2016}
\firstpageheadrule
\runningheader{НИУ-ВШЭ}{Высшая математика, стр. \thepage\ из \numpages}{олимпиада 2016}
\firstpagefooter{}{}{}
\runningfooter{}{}{}
\runningfootrule




\hqword{Задача}
\hpgword{Страница}
\hpword{Максимум}
\hsword{Баллы}
\htword{Итого}
\pointname{\%}
%\renewcommand{\solutiontitle}{\noindent\textbf{Решение:}\par\noindent}
\renewcommand{\solutiontitle}{}

%Таблица с результатами заполняется проверяющим работу. Пожалуйста, не делайте в ней пометок.

%\begin{center}
%  \gradetable[h][questions]
%\end{center}

\begin{center}
\textbf{Variant A} % Вариант А
\end{center}

\begin{questions}

\question Let $S$ be the $n\times n$ «shipbuilding timber» matrix, i.e. the square matrix with all elements equal to $1$ and $I$ be the $n\times n$ identity matrix. Let $A=aI+bS$ where $a$ and $b$ are scalar parameters.

\begin{parts}
\part[2] Express the matrix $A^2$ as a linear combination of matrices $I$ and $S$
\begin{solution}
\[
S^2=(aI+bS)^2=a^2I^2+b^2S^2+abIS+abSI=a^2I+b^2nS+2abS=a^2I+(b^2n+2ab)S
\]
\end{solution}

\part[6] Find the inverse of $A$ if it is known that it exists and can be represented as a linear combination of $I$ and $S$
\begin{solution}
Допустим обратная к $A$ матрица имеет вид $A^{-1}=cI+dS$.
\[
(aI+bS)(cI+dS)=acI+(ad+dc+bdn)S
\]
Чтобы этот результа равнялся $I$ нам нужно чтобы:
\[
\begin{cases}
ac=1 \\
ad+db+bdn = 0
\end{cases}
\]
Выражаем $c$ и $d$ через $a$ и $b$:
\[
\begin{cases}
c=1/a \\
d=-b/(a^2+b)
\end{cases}
\]

Итого,
\[
A^{-1}=a^{-1}I - \frac{b}{a^2+b}S
\]



\end{solution}


\part[2] Using your result in previous part or otherwise find the inverse of

\[
\begin{pmatrix}
-1 & 1 & 1 & 1 \\
1 & -1 & 1 & 1 \\
1 & 1 & -1 & 1 \\
1 & 1 & 1 & -1
\end{pmatrix}
\]

\begin{solution}
Здесь $A=-2I+S$. Значит
\[
A^{-1}=-0.5I - \frac{1}{4+1}S =
\begin{pmatrix}
-0.7 & -0.2 & -0.2 & -0.2 \\
-0.2 & -0.7 & -0.2 & -0.2 \\
-0.2 & -0.2 & -0.7 & -0.2 \\
-0.2 & -0.2 & -0.2 & -0.7
\end{pmatrix}
\]
\end{solution}

\end{parts}



\end{questions}



\begin{figure}[b]
\caption{Таблица значений функции распределения для стандартной нормальной величины}
  \begin{minipage}[b]{0.35\linewidth}
    \centering
    \begin{tikzpicture}
% define normal distribution function 'normaltwo'
    \def\normaltwo{\x,{4*1/exp(((\x-3)^2)/2)}}

% input y parameter
    \def\y{4.4}

% this line calculates f(y)
    \def\fy{4*1/exp(((\y-3)^2)/2)}

% Shade orange area underneath curve.
    \fill [fill=gray!30] (2.6,0) -- plot[domain=0:4.4] (\normaltwo) -- ({\y},0) -- cycle;

% Draw and label normal distribution function
    \draw[domain=0:6] plot (\normaltwo) node[right] {};

% Add dashed line dropping down from normal.
    \draw[dashed] ({\y},{\fy}) -- ({\y},0) node[below] {$x$};

% Optional: Add axis labels
%    \draw (-.2,2.5) node[left] {$f_Y(u)$};
    \draw (3,2) node[below] {$F(x)$};

% Optional: Add axes
    \draw[->] (0,0) -- (6.2,0) node[right] {};
%    \draw[->] (0,0) -- (0,5) node[above] {};

\end{tikzpicture}
%    \rule{6cm}{6cm} %to simulate an actual figure
\par\vspace{0pt}
  \end{minipage}%
  \begin{minipage}[b]{0.60\linewidth}
    \centering
\begin{tabular}{rr|rr|rr|rr}
  \hline
$x$ & $F(x)$ & $x$ & $F(x)$ & $x$ & $F(x)$ & $x$ & $F(x)$ \\
  \hline
0.050 & 0.520 & 0.750 & 0.773 & 1.450 & 0.926 & 2.150 & 0.984 \\
  0.100 & 0.540 & 0.800 & 0.788 & 1.500 & 0.933 & 2.200 & 0.986 \\
  0.150 & 0.560 & 0.850 & 0.802 & 1.550 & 0.939 & 2.250 & 0.988 \\
  0.200 & 0.579 & 0.900 & 0.816 & 1.600 & 0.945 & 2.300 & 0.989 \\
  0.250 & 0.599 & 0.950 & 0.829 & 1.650 & 0.951 & 2.350 & 0.991 \\
  0.300 & 0.618 & 1.000 & 0.841 & 1.700 & 0.955 & 2.400 & 0.992 \\
  0.350 & 0.637 & 1.050 & 0.853 & 1.750 & 0.960 & 2.450 & 0.993 \\
  0.400 & 0.655 & 1.100 & 0.864 & 1.800 & 0.964 & 2.500 & 0.994 \\
  0.450 & 0.674 & 1.150 & 0.875 & 1.850 & 0.968 & 2.550 & 0.995 \\
  0.500 & 0.691 & 1.200 & 0.885 & 1.900 & 0.971 & 2.600 & 0.995 \\
  0.550 & 0.709 & 1.250 & 0.894 & 1.950 & 0.974 & 2.650 & 0.996 \\
  0.600 & 0.726 & 1.300 & 0.903 & 2.000 & 0.977 & 2.700 & 0.997 \\
  0.650 & 0.742 & 1.350 & 0.911 & 2.050 & 0.980 & 2.750 & 0.997 \\
  0.700 & 0.758 & 1.400 & 0.919 & 2.100 & 0.982 & 2.800 & 0.997 \\
   \hline
\end{tabular}
\par\vspace{0pt}
\end{minipage}
\label{fig:test}
\end{figure}

\begin{flushright}
Удачи!
\end{flushright}

\end{document}
