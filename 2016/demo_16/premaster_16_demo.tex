\documentclass[addpoints, answers]{exam} % добавить или удалить answers в скобках, чтобы показать ответы
% \documentclass[addpoints, answers]{exam} % добавить или удалить answers в скобках, чтобы показать ответы


\usepackage[utf8]{inputenc}
\usepackage[russian]{babel}
%\usepackage[OT1]{fontenc}
\usepackage{booktabs}
\usepackage{amsmath}
\usepackage{tikz}
\usepackage{amsfonts}
\usepackage{amssymb}
\usepackage[left=2cm,right=2cm,top=2cm,bottom=2cm]{geometry}
\DeclareMathOperator{\E}{\mathbb{E}}
\DeclareMathOperator{\Var}{\mathbb{V}\mathrm{ar}}
\DeclareMathOperator{\Cov}{\mathbb{C}\mathrm{ov}}
\DeclareMathOperator{\sgn}{\rm sgn}
\let\P\relax
\DeclareMathOperator{\P}{\mathbb{P}}
\newcommand{\cN}{\mathcal{N}}
\newcommand{\RR}{\mathbb{R}}
\newcommand{\hbeta}{\hat{\beta}}

\usepackage{floatrow}
%\newfloatcommand{capbtabbox}{table}[][\FBwidth]

\begin{document}

\pagestyle{headandfoot}
\runningheadrule
\firstpageheader{НИУ-ВШЭ}{Высшая математика, стр. \thepage\ из \numpages}{демо 2016}
\firstpageheadrule
\runningheader{НИУ-ВШЭ}{Высшая математика, стр. \thepage\ из \numpages}{демо 2016}
\firstpagefooter{}{}{}
\runningfooter{}{}{}
\runningfootrule




\hqword{Задача}
\hpgword{Страница}
\hpword{Максимум}
\hsword{Баллы}
\htword{Итого}
\pointname{\%}
%\renewcommand{\solutiontitle}{\noindent\textbf{Решение:}\par\noindent}
\renewcommand{\solutiontitle}{}

%Таблица с результатами заполняется проверяющим работу. Пожалуйста, не делайте в ней пометок.

%\begin{center}
%  \gradetable[h][questions]
%\end{center}

\begin{center}
\textbf{Демо-версия} % Вариант А
\end{center}

\begin{questions}

\question[10] Evaluate the following limit:

\[
\lim_{x \to 0} \sqrt[x]{\cos \sqrt{x}}
\]

\begin{solution}

\[
\lim_{x \rightarrow 0} \sqrt[x]{\cos \sqrt{x}} = \lim_{x \rightarrow 0} \left({\cos \sqrt{x}}\right) ^{1/x} =  \lim_{x \rightarrow 0} \exp
\left( \frac{\ln (\cos \sqrt{x})}{x} \right) = \exp \left( \lim_{x \rightarrow 0}
\frac{\ln (\cos \sqrt{x})}{x} \right),
\]

\noindent где последнее равенство использует теорему о предельном переходе для непрерывных функций. Далее используем разложение в ряд Маклорена.

\[
\frac{\ln (\cos \sqrt{x})}{x} = \frac{\ln (1-
\frac{x}{2}+\frac{x^2}{24}+o(x^2))}{x} = \frac{-\frac{x}{2}+o(x)}{x}
\]

\noindent Следовательно,

\[
\lim_{x \rightarrow 0} \frac{\ln (\cos \sqrt{x})}{x} = -\frac{1}{2},
\]

\noindent поэтому

\[
\lim_{x \rightarrow 0} \sqrt[x]{\cos \sqrt{x}} = \exp \left( -\frac{1}{2}\right) = \frac{1}{\sqrt{e}}.
\]
\end{solution}

\question[10] Find and classify the discontinuity points of the following function:

\[
f(x) = {\sgn} \left(\sin \left( \frac{\pi}{x}\right)\right).
\]

\begin{solution}
Точки, в которых данная функция может иметь разрыв: $x=0$, поскольку в ней равен нулю знаменатель аргумента функции, и точки $x=1/k, k \in \mathbb{Z}$, поскольку в них $\sin \left( \frac{\pi}{x}\right)$ меняет знак. В точках $x=1/k, k \in \mathbb{Z}$ функция имеет разрывы первого рода, так как существуют не равные между собой односторонние пределы. Например, рассмотрим $k=1$. Существует правосторонняя окрестность точки $x=1$, в которой функция $\sin \left( \frac{\pi}{x} \right)$ положительна. В самом деле, для $x \in (1,2)$ имеет место $\frac{\pi}{2} < \frac{\pi}{x} < \pi$. Для точек из этой окрестности имеем $f(x) = 1$, следовательно, $\lim_{x \rightarrow 1+0} f(x) = 1$. С другой стороны, существует левосторонняя окрестность точки $x=1$, в которой функция $\sin \left( \frac{\pi}{x} \right)$ отрицательна. В самом деле, для $x \in (1/2,1)$ имеет место $\pi < \frac{\pi}{x} < 2 \pi$. Для точек из этой окрестности имеем $f(x) = -1$, следовательно, $\lim_{x \rightarrow 1-0} f(x) = -1$. Аналогичные окрестности могут быть найдены для всех рассматриваемых точек.



В точке $x=0$ функция имеет разрыв второго рода, поскольку не существует односторонних пределов. Действительно, рассмотрим последовательности $a_n = \frac{2}{1+4 n}, n
\in \mathbb{N}$ и $b_n = \frac{2}{3+4 n}, n \in \mathbb{N}$, стремящиеся к нулю справа. Тогда $f(a_n) = {\sgn} \left(\sin \left( \frac{\pi}{\frac{2}{1+4 n}}\right)\right) = {\sgn} \left( \sin \left( \frac{\pi}{2} + 2 \pi n \right)\right)=1$ и $f(b_n) = {\sgn} \left(\sin \left( \frac{\pi}{\frac{2}{3+4 n}}\right)\right) = {\sgn} \left( \sin \left( \frac{3 \pi}{2} + 2 \pi n \right)\right)=-1$. Тем самым показано, что правостороннего предела $f(x)$ при $x$ стремящемся к нулю не существует. Аналогично можно показать, что не существует левостороннего предела, например, рассмотрев последовательности $-a_n$ и $-b_n$.
\end{solution}

\question Density function of a random variable $Y$ is given by

\[
f(y)=
\begin{cases}
    \frac{1}{\theta^2} y e^{-y/ \theta}, \text{ if } y>0 \\
    0, \text{ otherwise } \\
\end{cases}
\]

You have 3 observations on $Y$: $y_1 = 48, y_2 = 50, y_3 = 52$.

\begin{parts}
\part[4] Using maximum likelihood, find the estimate of $\theta$
\begin{solution}

  Нахождение оценки:

  \[
    \begin{array}{l}
  	\ln(L) = \sum_{i=1}^n (-\ln(\theta^2)+\ln(y_i) - \frac{y_i}{\theta}) \\
  	\ln(L) = -2n\ln(\theta)+\sum_{i=1}^n \ln(y_i)  - \frac{\sum_{i=1}^n y_i}{\theta} \\
  	\frac{\partial \ln(L)}{\partial \theta} = - \frac{2n}{\theta} + \frac{\sum_{i=1}^n y_i}{\theta^2} = 0 \\
  	\widehat{\theta} = \frac{\sum_{i=1}^n y_i}{2n} = \frac{\overline{y}}{2} \\
    \end{array}
  \]

  Подставляя наши данные, получаем $\widehat{\theta} = 25$.
\end{solution}

\part[3] Is the estimator $\hat\theta$ unbiased?

\begin{solution}
  Несмещенность:

  \[
    \begin{array}{l}
  	\E(\widehat{\theta}) = \frac{\E(y_i)}{2}
    \end{array}
  \]

  Найдём математическое ожидание $y_i$:

  \[
    \begin{array}{l}
  	\E(y_i) = \int_0^{+ \infty} \frac{1}{\theta^2} y^2 e^{-y/ \theta} dy
    \end{array}
  \]

  Интегрируя по частям, получаем:

  \[
    \begin{array}{l}
  	\E(y_i) = \int_0^{+ \infty} \frac{2y}{\theta} e^{-y/ \theta} dy \\
  	\E(y_i) = \int_0^{+ \infty} 2 e^{-y/ \theta} dy \\
  	\E(y_i) = 2 \theta
    \end{array}
  \]

  Тогда $\E(\widehat{\theta}) = \frac{\E(y_i)}{2} = \theta$. Оценка несмещенная.

\end{solution}

\part[3] Calculate the variance of $\hat\theta$


\begin{solution}
  Для расчёта дисперсии вычислим $\E(y_i^2)$:

  \[
  	\E(y_i^2) = \int_0^{+ \infty} \frac{1}{\theta^2} y^3 e^{-y/ \theta} dy
  \]

  Аналогично предыдущему случаю, интегрируем по частям. Получаем:

  \[
  	\E(y_i^2) = 6 \theta^2
  \]

  Тогда

  \[
  	\Var(y_i) = 6 \theta^2 - 4 \theta^2 = 2 \theta^2
  \]

  И дисперсия оценки

  \[
  	\Var(\widehat{\theta}) = \Var(\frac{\overline{y}}{2}) = \frac{1}{4n} \Var(y_i) = \frac{\theta^2}{2n}
  \]


\end{solution}

\end{parts}



\question Consider a function

\[
f(x)=\begin{cases}
    \frac{1}{x^2}, \text{ if } c_1<x<c_2 \\
    0, \text{ otherwise }
\end{cases}
\]

\begin{parts}
\part[5] Find all $c_1$ and $c_2$ such that the function $f$ is a density function for some random variable $X$

\begin{solution}
  Чтобы функция была функцией плотности:

  \[
  \begin{array}{l}
  \int_{c_1}^{c_2} \frac{1}{x^2}dx = 1 \\
  -\frac{1}{c_2}+\frac{1}{c_1}=1 \\
  c_1 = \frac{c_2}{1+c_2}
  \end{array}
  \]

\end{solution}

\part[5] Calculate the expected value and variance of the random variable $X$ for $c_2 = 9$

\begin{solution}
  Математическое ожидание:

  \[
  \E(X)=\int_{0.9}^9 \frac{1}{x^2} x dx = \int_{0.9}^9\frac{1}{x} dx = \ln(9)-\ln(0.9)
  \]

  Дисперсия:

  \[
  \begin{array}{l}
  \Var(X) = \E(X^2)-\E(X)^2  \\
  \E(X^2) = \int_{0.9}^9 \frac{1}{x^2} x^2 dx = 9-0.9 = 8.1 \\
  \Var(X) = 8.1 - (\ln(9)-\ln(0.9))^2
  \end{array}
  \]

\end{solution}


\end{parts}

\question Let $S$ be the $n\times n$ «shipbuilding timber» matrix, i.e. the square matrix with all elements equal to $1$.

\begin{parts}
\part[2] Express $S^2$ in terms of $S$
\begin{solution}
Перемножаем в лоб:
\[
S\cdot S = \begin{pmatrix}
n & \cdots & n \\
\vdots & & \vdots \\
n & \cdots & n
\end{pmatrix} = n S
\]
\end{solution}
\part[3] For the eigenvalues of $S$
\begin{solution}
Допустим, что $Sv=\lambda v$

Домножим обе стороны на $S$. Получим:
\[
S^2 v= \lambda Sv
\]

С другой стороны $S^2 v = n Sv$. Значит:

\[
n Sv = \lambda Sv
\]

Отсюда, либо $\lambda=n$, либо $Sv=0$, что означает, что $\lambda=0$.
\end{solution}

\part[3] For each eigenvalue of $S$ find at least on eigenvector
\begin{solution}
Разберёмся с собственными векторами для матрицы $S$. Ищем собственный вектор для $\lambda=0$. Получаем, что
\[
(1, 1, 1, \ldots, 1) \cdot v = 0
\]
Значит подходит любой ненулевой вектор с суммой компонент, равной нулю. Например, подойдёт
\[
v=(1,-1, 0, 0, \ldots, 0)
\]
Ищем собственный вектор для $\lambda=n$. Все строки матрицы $S$ одинаковы, поэтому все элементы вектора $Sv$ одинаковы. Значит в $v$ должны быть одинаковые элементы. Например, подойдёт
\[
v=(1,1, 1, 1, \ldots, 1)
\]

\end{solution}

\part[2] Find all the eigenvalues of the matrix $A=aI+bS$, where $I$ is the identity matrix.
\begin{solution}
Когда мы домножаем матрицу $S$ на число $b$ собственные числа домножаются на $b$. Если мы прибавляем константу $a$ по диагонали, то собственные числа увеличиваются на $a$. Значит собственные числа матрицы $A$ равны $a+bn$ и $a$.

Кстати, при домножении матрицы $S$ на константу собственные векторы не изменяются, равно как и при прибавлении константы $a$ по диагонали.
\end{solution}


\end{parts}

\question Solve the  differential equation:

\[
y''' -4y'' +y' =2x^{2} +1.
\]

\begin{solution}
Сначала запишем решение однородного дифференциального уравнения:

\[y''' -4y'' +y' =0.\]

Составим к нему характеристическое уравнение:

  \[\lambda ^{3} -4\lambda ^{2} +\lambda =0.\]

  \[\lambda _{1} =0,\lambda _{2} =2,\lambda _{3} =-2.\]

  То есть общее решение дифференциального уравнения может быть записано как

  \[y=C_{1} +C_{2} e^{2x} +C_{3} e^{-2x},\; C_{1} ,C_{2} ,C_{3} \in \RR \]

  Найдем частное решение этого дифференциального уравнения. В данной задаче -- резонансный случай, поскольку $\left(2x^{2} +1\right)e^{0} =2x^{2} +1$. То есть будем искать частное решение в виде $y=(ax^{2} +bx+c)x$. Тогда

  \[y' =3ax^{2} +2bx+c\]

  \[y'' =6ax+2b\]

  \[y''' =6a\]

  Следовательно,

  \[6a-4(6ax+2b)+3ax^{2} +2bx+c=2x^{2} +1\]

  \[3ax^{2} +2bx-24ax+6a-8b+c=2x^{2} +1.\]

  Отсюда

  \[\left\{\begin{array}{l} {6a-8b+c=1} \\ {2b-24a=0} \\ {3a=2} \end{array}\right. \]

  \[\left\{\begin{array}{l} {c=61} \\ {b=8} \\ {a=\frac{2}{3} } \end{array}\right. .\]

  Поэтому общее решение дифференциального уравнения:

  \[y=C_{1} +C_{2} e^{2x} +C_{3} e^{-2x} +\frac{2}{3} x^{3} +8x^{2} +61x,\; C_{1} ,C_{2} ,C_{3} \in \RR \]

\end{solution}


\question Let $A$, $B$ and $C$ be square matrix of size $n\times n$. Prove the following statements or provide counterexample:

\begin{parts}
\part[2] If $B=C^{-1} AC$, then $\det (A)=\det (B)$
\begin{solution}
$\det (B)=\det (C^{-1} AC)=\det (C^{-1} )\det (C)\det (A)=\det (C^{-1} C)\det (A)=\det (A)$
\end{solution}

\part[3] $\det ((A+B)^{2} )=\det (A^{2} +2AB+B^{2} )$
\begin{solution}
Неверно, контрпример:
\[A=\left(\begin{array}{cc} {1} & {1} \\ {0} & {1} \end{array}\right),B=\left(\begin{array}{cc} {0} & {1} \\ {0} & {1} \end{array}\right)\]
В этом случае $\det ((A+B)^{2} )=-2$, но $\det (A^{2} +2AB+B^{2} )=-3$.

\end{solution}

\part[3] $\det ((A+B)^{2} )=\det (A^{2} +B^{2} )$
\begin{solution}
Неверно, контрпример: $A=I$, $B=I$. Тогда $\det ((A+B)^{2} )=16$, $\det (A^{2} +B^{2} )=4$.
\end{solution}

\part[2] If $A$ is invertible, then $(I+A^{-1} )^{-1} =A(A+I)^{-1} $
\begin{solution}
$(I+A^{-1} )^{-1} =(AA^{-1} +A^{-1} )^{-1} =\left((A+I)A^{-1} \right)^{-1} =A(A+I)^{-1} $
\end{solution}

\end{parts}

\question[10] Solve the differential equation
\[
2xyy' - y' \ln y + y^2 + \ln x =0
\]
\begin{solution}
Домножим на $dx$
\[
(y^2+\ln x)dx + (2xy - \ln y)dy=0
\]
Убеждаемся, что это уравнение в полных дифференциалах:
\[
\frac{\partial}{\partial y}(y^2+\ln x)=\frac{\partial}{\partial x}(2xy - \ln y)
\]

И решаем его по стандартной схеме.

Находим интеграл функции при $dx$ по $x$

\[
F(x,y)=\int  y^2+\ln x \, dx = xy^2 + x(\ln x - 1) + C(y)
\]

Теперь приравниваем функцию при $dy$  и $F'_y(x,y)$:

\[
2xy + C'(y)=2xy - \ln y
\]

Отсюда находим $C(y)$:

\[
C(y)=\int -\ln y \, dy = y(1-\ln y) + C, \text{ где } C\in \RR
\]

Итого:

\[
xy^2+x(\ln x - 1) + y(1-\ln y)=C, \text{ где } C\in \RR
\]


\end{solution}


\end{questions}



\begin{figure}[b]
\caption{Таблица значений функции распределения для стандартной нормальной величины}
  \begin{minipage}[b]{0.35\linewidth}
    \centering
    \begin{tikzpicture}
% define normal distribution function 'normaltwo'
    \def\normaltwo{\x,{4*1/exp(((\x-3)^2)/2)}}

% input y parameter
    \def\y{4.4}

% this line calculates f(y)
    \def\fy{4*1/exp(((\y-3)^2)/2)}

% Shade orange area underneath curve.
    \fill [fill=gray!30] (2.6,0) -- plot[domain=0:4.4] (\normaltwo) -- ({\y},0) -- cycle;

% Draw and label normal distribution function
    \draw[domain=0:6] plot (\normaltwo) node[right] {};

% Add dashed line dropping down from normal.
    \draw[dashed] ({\y},{\fy}) -- ({\y},0) node[below] {$x$};

% Optional: Add axis labels
%    \draw (-.2,2.5) node[left] {$f_Y(u)$};
    \draw (3,2) node[below] {$F(x)$};

% Optional: Add axes
    \draw[->] (0,0) -- (6.2,0) node[right] {};
%    \draw[->] (0,0) -- (0,5) node[above] {};

\end{tikzpicture}
%    \rule{6cm}{6cm} %to simulate an actual figure
\par\vspace{0pt}
  \end{minipage}%
  \begin{minipage}[b]{0.60\linewidth}
    \centering
\begin{tabular}{rr|rr|rr|rr}
  \hline
$x$ & $F(x)$ & $x$ & $F(x)$ & $x$ & $F(x)$ & $x$ & $F(x)$ \\
  \hline
0.050 & 0.520 & 0.750 & 0.773 & 1.450 & 0.926 & 2.150 & 0.984 \\
  0.100 & 0.540 & 0.800 & 0.788 & 1.500 & 0.933 & 2.200 & 0.986 \\
  0.150 & 0.560 & 0.850 & 0.802 & 1.550 & 0.939 & 2.250 & 0.988 \\
  0.200 & 0.579 & 0.900 & 0.816 & 1.600 & 0.945 & 2.300 & 0.989 \\
  0.250 & 0.599 & 0.950 & 0.829 & 1.650 & 0.951 & 2.350 & 0.991 \\
  0.300 & 0.618 & 1.000 & 0.841 & 1.700 & 0.955 & 2.400 & 0.992 \\
  0.350 & 0.637 & 1.050 & 0.853 & 1.750 & 0.960 & 2.450 & 0.993 \\
  0.400 & 0.655 & 1.100 & 0.864 & 1.800 & 0.964 & 2.500 & 0.994 \\
  0.450 & 0.674 & 1.150 & 0.875 & 1.850 & 0.968 & 2.550 & 0.995 \\
  0.500 & 0.691 & 1.200 & 0.885 & 1.900 & 0.971 & 2.600 & 0.995 \\
  0.550 & 0.709 & 1.250 & 0.894 & 1.950 & 0.974 & 2.650 & 0.996 \\
  0.600 & 0.726 & 1.300 & 0.903 & 2.000 & 0.977 & 2.700 & 0.997 \\
  0.650 & 0.742 & 1.350 & 0.911 & 2.050 & 0.980 & 2.750 & 0.997 \\
  0.700 & 0.758 & 1.400 & 0.919 & 2.100 & 0.982 & 2.800 & 0.997 \\
   \hline
\end{tabular}
\par\vspace{0pt}
\end{minipage}
\label{fig:test}
\end{figure}

\begin{flushright}
Удачи!
\end{flushright}

\end{document}
