\documentclass[12pt]{article} % размер шрифта

\usepackage{tikz} % картинки в tikz
\usepackage{microtype} % свешивание пунктуации

\usepackage{array} % для столбцов фиксированной ширины

\usepackage{indentfirst} % отступ в первом параграфе

\usepackage{sectsty} % для центрирования названий частей
\allsectionsfont{\centering} % приказываем центрировать все sections

\usepackage{amsmath} % куча стандартных математических плюшек

\usepackage[top=2cm, left=1.5cm, right=1.5cm, bottom=2cm]{geometry} % размер текста на странице

\usepackage{lastpage} % чтобы узнать номер последней страницы

\usepackage{enumitem} % дополнительные плюшки для списков
%  например \begin{enumerate}[resume] позволяет продолжить нумерацию в новом списке
\usepackage{caption} % подписи к картинкам без плавающего окружения figure


\usepackage{fancyhdr} % весёлые колонтитулы
\pagestyle{fancy}
\lhead{Три задачи по эконометрике}
\chead{}
\rhead{2017-xx-xx}
\lfoot{}
\cfoot{}
\rfoot{\thepage/\pageref{LastPage}}
\renewcommand{\headrulewidth}{0.4pt}
\renewcommand{\footrulewidth}{0.4pt}



\usepackage{todonotes} % для вставки в документ заметок о том, что осталось сделать
% \todo{Здесь надо коэффициенты исправить}
% \missingfigure{Здесь будет картина Последний день Помпеи}
% команда \listoftodos — печатает все поставленные \todo'шки

\usepackage{booktabs} % красивые таблицы
% заповеди из документации:
% 1. Не используйте вертикальные линии
% 2. Не используйте двойные линии
% 3. Единицы измерения помещайте в шапку таблицы
% 4. Не сокращайте .1 вместо 0.1
% 5. Повторяющееся значение повторяйте, а не говорите "то же"

\usepackage{fontspec} % поддержка разных шрифтов
\usepackage{polyglossia} % поддержка разных языков

\setmainlanguage{russian}
\setotherlanguages{english}

\setmainfont{Linux Libertine O} % выбираем шрифт
% можно также попробовать Helvetica, Arial, Cambria и т.Д.

% чтобы использовать шрифт Linux Libertine на личном компе,
% его надо предварительно скачать по ссылке
% http://www.linuxlibertine.org/

\newfontfamily{\cyrillicfonttt}{Linux Libertine O}
% пояснение зачем нужно шаманство с \newfontfamily
% http://tex.stackexchange.com/questions/91507/

\AddEnumerateCounter{\asbuk}{\russian@alph}{щ} % для списков с русскими буквами
\setlist[enumerate, 2]{label=\asbuk*),ref=\asbuk*} % списки уровня 2 будут буквами а) б) ...

%% эконометрические и вероятностные сокращения
\DeclareMathOperator{\Cov}{Cov}
\DeclareMathOperator{\Corr}{Corr}
\DeclareMathOperator{\Var}{Var}
\DeclareMathOperator{\E}{E}
\def \hb{\hat{\beta}}
\def \hs{\hat{\sigma}}
\def \htheta{\hat{\theta}}
\def \s{\sigma}
\def \hy{\hat{y}}
\def \hY{\hat{Y}}
\def \v1{\vec{1}}
\def \e{\varepsilon}
\def \he{\hat{\e}}
\def \z{z}
\def \hVar{\widehat{\Var}}
\def \hCorr{\widehat{\Corr}}
\def \hCov{\widehat{\Cov}}
\def \cN{\mathcal{N}}


\begin{document}



\begin{enumerate}

\item Регрессионная модель имеет вид $y_i= \beta_1+ \beta_x x_{i}+ \beta_z z_{i}+ \beta_w w_{i}+u_i$. Исследователь Феофан оценил эту модель по 20 наблюдениям и оказалось, что $R^2=0.8$. Феофан хочет проверить гипотезу $H_0$ о том, что $\beta_x = \beta_z$ и одновременно $\beta_z + \beta_w = 0$. Предпосылки теоремы Гаусса-Маркова на ошибки $u_i$ выполнены, кроме того, $u_i$ нормально распределены.

\begin{enumerate}
\item Какую вспомогательную регрессию достаточно оценить Феофану для проверки $H_0$?
\item Во вспомогательной регрессии оказалось, что $R^2 = 0.7$. Отвергается ли $H_0$ на 5\%-ом уровне значимости?
\item На сколько процентов изменилась несмещённая оценка дисперсии случайной ошибки при переходе ко вспомогательной регрессии?
\end{enumerate}

Решение:

\begin{enumerate}
\item Подставляем ограничения и получаем
\[
y_i = \beta_1 + \beta_x (x_i + z_i - w_i) + u_i
\]
\item Применяем $F$-тест:
\[
F_{obs} = \frac{(R^2_{UR} - R^2_{R})/n_{rest}}{(1-R^2_{UR})/(n-k_{UR})} = \frac{(0.8-0.7)/2}{(1-0.8)/16}=4
\]
По таблицам находим критическое значение:
\[
F_{cr, 2, 16} = 3.63
\]

Вывод: $H_0$ отвергается.

\item На сколько процентов изменилась несмещённая оценка дисперсии случайной ошибки при переходе ко вспомогательной регрессии?

Вспоминаем, что $\hat{\sigma}^2 = RSS/ (n-k) = (1-R^2)\cdot TSS/(n-k)$. Величина $TSS$ не изменяется. Меняется только $a=(1-R^2)/(n-k)$. В исходной регрессии $a_{old}=0.2/16$, во вспомогательной $a_{new}=0.3/18$. Процентное изменение равно $(a_{new} - a_{old})/a_{old} \approx 33\%$.

\end{enumerate}

\item Исследователь Феофан изучает регрессию со $100$ наблюдениями и $10$ оцениваемыми коэффициентами. Предпосылки теоремы Гаусса-Маркова на ошибки $u_i$ выполнены, кроме того, $u_i$ нормально распределены.

Феофан хочет оценить неизвестную дисперсию $\sigma^2=\Var(u_i)$ по формуле $\hat{\sigma}^2 = c\cdot RSS$ так, чтобы величина среднеквадратичной ошибки была минимальной. Какое значение $c$ получит Феофан?

Подсказка: Феофан смутно помнит, что дисперсия $\chi^2$-распределения с $d$ степенями свободы равна $2d$.

Решение:

Заметим, что $RSS/\sigma^2 \sim \chi^2_{n-k}$, поэтому:
\[
MSE = \Var(\hat{\sigma}^2) + bias^2(\hat{\sigma}^2)=\sigma^4 \cdot (c^2 2(n-k) + (c(n-k)-1)^2).
\]
Минимизируя по $c$, получаем $c=1/(n-k+2) = 1/92$.



\item На работе Феофан построил парную регрессию по трём наблюдениям и посчитал прогнозы $\hat{y}_i$. Придя домой он отчасти вспомнил результаты:

\begin{tabular}{rr}
\toprule
$y_i$ & $\hy_i$ \\
\midrule
$0$ & $1$ \\
$6$ & ? \\
$6$ & ? \\
\bottomrule
\end{tabular}

Поднапрягшись, Феофан вспомнил, что третий прогноз был больше второго. Помогите Феофану восстановить пропущенные значения.

Решение:

На две неизвестных $a$ и $b$ нужно два уравнения. Эти два уравнения — ортогональность вектора остатков плоскости регрессоров. А именно:

\[
\begin{cases}
\sum_i (y_i - \hy_i) = 0 \\
\sum_i (y_i - \hy_i) \hy_i = 0 \\
\end{cases}
\]

В нашем случае

\[
\begin{cases}
-1 +(6-a) + (6-b) = 0 \\
-1 + (6 - a)a + (6-b)b = 0 \\
\end{cases}
\]

Решаем квадратное уравнение и получаем два решения: $a=4$ и $a=7$. Итого: $a=4$, $b=7$.


\end{enumerate}

\end{document}
