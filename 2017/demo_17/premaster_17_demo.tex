% \documentclass[addpoints, answers]{exam} % добавить или удалить answers в скобках, чтобы показать ответы
\documentclass[addpoints]{exam} % добавить или удалить answers в скобках, чтобы показать ответы


\usepackage[utf8]{inputenc}
\usepackage[russian]{babel}
%\usepackage[OT1]{fontenc}
\usepackage{booktabs}
\usepackage{amsmath}
\usepackage{tikz}
\usepackage{amsfonts}
\usepackage{amssymb}
\usepackage[left=2cm, right=2cm, top=2cm, bottom=2cm]{geometry}
\DeclareMathOperator{\E}{\mathbb{E}}
\DeclareMathOperator{\Var}{\mathbb{V}\mathrm{ar}}
\DeclareMathOperator{\Cov}{\mathbb{C}\mathrm{ov}}
\DeclareMathOperator{\Corr}{\mathbb{C}\mathrm{orr}}
\DeclareMathOperator{\sgn}{sgn}
\DeclareMathOperator{\tr}{tr}
\DeclareMathOperator{\rank}{rank}

\let\P\relax
\DeclareMathOperator{\P}{\mathbb{P}}
\newcommand{\cN}{\mathcal{N}}
\newcommand{\RR}{\mathbb{R}}
\newcommand{\hbeta}{\hat{\beta}}

\usepackage{floatrow}
%\newfloatcommand{capbtabbox}{table}[][\FBwidth]

\begin{document}

\pagestyle{headandfoot}
\runningheadrule
\firstpageheader{Higher School of Economics}{Further mathematics, p. \thepage\ of \numpages}{demo 2017}
\firstpageheadrule
\runningheader{Higher School of Economics}{Further mathematics, p. \thepage\ of \numpages}{demo 2017}
\firstpagefooter{}{}{}
\runningfooter{}{}{}
\runningfootrule




\hqword{Задача}
\hpgword{Страница}
\hpword{Максимум}
\hsword{Баллы}
\htword{Итого}
\pointname{\%}
%\renewcommand{\solutiontitle}{\noindent\textbf{Решение:}\par\noindent}
\renewcommand{\solutiontitle}{}

%Таблица с результатами заполняется проверяющим работу. Пожалуйста, не делайте в ней пометок.

%\begin{center}
%  \gradetable[h][questions]
%\end{center}

\begin{center}
\textbf{Exam demo-version} % Вариант А
\end{center}

\begin{questions}

\question[10] Evaluate the following limit:

\[
\lim_{x \to 0} \sqrt[x]{\cos \sqrt{x}}
\]

\begin{solution}

\[
\lim_{x \rightarrow 0} \sqrt[x]{\cos \sqrt{x}} = \lim_{x \rightarrow 0} \left({\cos \sqrt{x}}\right) ^{1/x} =  \lim_{x \rightarrow 0} \exp
\left( \frac{\ln (\cos \sqrt{x})}{x} \right) = \exp \left( \lim_{x \rightarrow 0}
\frac{\ln (\cos \sqrt{x})}{x} \right),
\]

\noindent где последнее равенство использует теорему о предельном переходе для непрерывных функций. Далее используем разложение в ряд Маклорена.

\[
\frac{\ln (\cos \sqrt{x})}{x} = \frac{\ln (1-
\frac{x}{2}+\frac{x^2}{24}+o(x^2))}{x} = \frac{-\frac{x}{2}+o(x)}{x}
\]

\noindent Следовательно,

\[
\lim_{x \rightarrow 0} \frac{\ln (\cos \sqrt{x})}{x} = -\frac{1}{2},
\]

\noindent поэтому

\[
\lim_{x \rightarrow 0} \sqrt[x]{\cos \sqrt{x}} = \exp \left( -\frac{1}{2}\right) = \frac{1}{\sqrt{e}}.
\]
\end{solution}

\question[10] Find and classify the discontinuity points of the following function:

\[
f(x) = {\sgn} \left(\sin \left( \frac{\pi}{x}\right)\right).
\]

\begin{solution}
Точки, в которых данная функция может иметь разрыв: $x=0$, поскольку в ней равен нулю знаменатель аргумента функции, и точки $x=1/k, k \in \mathbb{Z}$, поскольку в них $\sin \left( \frac{\pi}{x}\right)$ меняет знак. В точках $x=1/k, k \in \mathbb{Z}$ функция имеет разрывы первого рода, так как существуют не равные между собой односторонние пределы. Например, рассмотрим $k=1$. Существует правосторонняя окрестность точки $x=1$, в которой функция $\sin \left( \frac{\pi}{x} \right)$ положительна. В самом деле, для $x \in (1,2)$ имеет место $\frac{\pi}{2} < \frac{\pi}{x} < \pi$. Для точек из этой окрестности имеем $f(x) = 1$, следовательно, $\lim_{x \rightarrow 1+0} f(x) = 1$. С другой стороны, существует левосторонняя окрестность точки $x=1$, в которой функция $\sin \left( \frac{\pi}{x} \right)$ отрицательна. В самом деле, для $x \in (1/2,1)$ имеет место $\pi < \frac{\pi}{x} < 2 \pi$. Для точек из этой окрестности имеем $f(x) = -1$, следовательно, $\lim_{x \rightarrow 1-0} f(x) = -1$. Аналогичные окрестности могут быть найдены для всех рассматриваемых точек.



В точке $x=0$ функция имеет разрыв второго рода, поскольку не существует односторонних пределов. Действительно, рассмотрим последовательности $a_n = \frac{2}{1+4 n}, n
\in \mathbb{N}$ и $b_n = \frac{2}{3+4 n}, n \in \mathbb{N}$, стремящиеся к нулю справа. Тогда $f(a_n) = {\sgn} \left(\sin \left( \frac{\pi}{\frac{2}{1+4 n}}\right)\right) = {\sgn} \left( \sin \left( \frac{\pi}{2} + 2 \pi n \right)\right)=1$ и $f(b_n) = {\sgn} \left(\sin \left( \frac{\pi}{\frac{2}{3+4 n}}\right)\right) = {\sgn} \left( \sin \left( \frac{3 \pi}{2} + 2 \pi n \right)\right)=-1$. Тем самым показано, что правостороннего предела $f(x)$ при $x$ стремящемся к нулю не существует. Аналогично можно показать, что не существует левостороннего предела, например, рассмотрев последовательности $-a_n$ и $-b_n$.
\end{solution}





\question It is known, that the $2 \times 2$ matrix $A$ has $\tr(A) = -7$ (matrix trace, the sum of diagonal elements) and $\det(A) = 0$.

\begin{parts}

\part[5] Find the eigenvalues of $A$
\begin{solution}

Let $\lambda_1, \lambda_2$ be the eigenvalues of A. Then $\tr(A) = \lambda_1 + \lambda_2$, $\det(A) = \lambda_1 \cdot \lambda_2$. Hence, $\lambda_1 = -7, \lambda_2 = 0$

\end{solution}

\part[5] Find $B^{2017}$ for
$B =
\begin{pmatrix}
1 & -4\\
2 & -8
\end{pmatrix}$



\begin{solution}

The matrix $B$ has eigenvalues $\lambda_1 = -7$ and $\lambda_2 = 0$ and corresponding eigenvectors $\begin{pmatrix}
1 & 2
\end{pmatrix}$ and $\begin{pmatrix}
4 & 1
\end{pmatrix}$. Then $B$ can be represented as $B=
\begin{pmatrix}
1 & 4\\
2 & 1
\end{pmatrix}
\begin{pmatrix}
-7 & 0\\
0 & 0
\end{pmatrix}
\begin{pmatrix}
1 & 4\\
2 & 1
\end{pmatrix}^{-1}
$.

Hence, $B^{2017} = \begin{pmatrix}
1 & 4\\
2 & 1
\end{pmatrix}
\begin{pmatrix}
(-7)^{2017} & 0\\
0 & 0
\end{pmatrix}
\begin{pmatrix}
1 & 4\\
2 & 1
\end{pmatrix}^{-1}
$

\end{solution}


\end{parts}



\question It is known that $A$ is a square matrix and
\[
A^T X A = \begin{pmatrix}
18 & 0 & 2\\
0 & 0 & 3\\
2 & 3 & 10
\end{pmatrix},
\]

\begin{parts}

\part[3] Find $\rank(X)$

\begin{solution}
$\det(A^T X A) = \det(A) \cdot \det(X) \cdot \det(A) = 18 \cdot (- 3 \cdot 3) = -162 \neq 0$ (since $\det(AB) = \det(A) \det(B)$). Hence, $\det(X) \neq 0$ and $\rank(X) = 3$.
\end{solution}

\part[2]  For given matrix $A = \begin{pmatrix}
3 & 2 & 0\\
0 & 3 & 0\\
0 & 5 & 1
\end{pmatrix}
$ find the determinant $\det(X)$

\begin{solution}

$\det(A) = 9$, $\det(A)^2 \cdot \det(X) = -162$, hence, $\det(X) = -2$.

\end{solution}

\part[5] For given matrix $A = \begin{pmatrix}
3 & 2 & 0\\
0 & 3 & 0\\
0 & 5 & 1
\end{pmatrix}$
find the matrix $X$ and test it for positive and negative definiteness.

\begin{solution}

It is not hard to find that $X = \begin{pmatrix}
2 & -22/9 & 2/3 \\
-22/9 & 724/27 & -145/9\\
2/3 & -145/9 & 10
\end{pmatrix}
$.


Check the principal minors signs: $\Delta_1 > 0, \Delta_2 > 0, \Delta_3 < 0$.
Using Sylvester's criterion, we conclude that $X$ is indefinite.

\end{solution}

\end{parts}


\question[10] Solve the  differential equation:

\[
y''' -4y'' +y' =2x^{2} +1.
\]

\begin{solution}
Сначала запишем решение однородного дифференциального уравнения:

\[y''' -4y'' +y' =0.\]

Составим к нему характеристическое уравнение:

  \[\lambda ^{3} -4\lambda ^{2} +\lambda =0.\]

  \[\lambda _{1} =0,\lambda _{2} =2 + \sqrt{3},\lambda _{3} =2 - \sqrt{3}.\]

  То есть общее решение дифференциального уравнения может быть записано как

  \[y=C_{1} +C_{2} e^{\lambda_2 x} +C_{3} e^{\lambda_3 x},\; C_{1} ,C_{2} ,C_{3} \in \RR \]

  Найдем частное решение этого дифференциального уравнения. В данной задаче -- резонансный случай, поскольку $\left(2x^{2} +1\right)e^{0} =2x^{2} +1$. То есть будем искать частное решение в виде $y=(ax^{2} +bx+c)x$. Тогда

  \[y' =3ax^{2} +2bx+c\]

  \[y'' =6ax+2b\]

  \[y''' =6a\]

  Следовательно,

  \[6a-4(6ax+2b)+3ax^{2} +2bx+c=2x^{2} +1\]

  \[3ax^{2} +2bx-24ax+6a-8b+c=2x^{2} +1.\]

  Отсюда

  \[\left\{\begin{array}{l} {6a-8b+c=1} \\ {2b-24a=0} \\ {3a=2} \end{array}\right. \]

  \[\left\{\begin{array}{l} {c=61} \\ {b=8} \\ {a=\frac{2}{3} } \end{array}\right. .\]

  Поэтому общее решение дифференциального уравнения:

  \[y=C_{1} +C_{2} e^{\lambda_2 x} +C_{3} e^{\lambda_3 x} +\frac{2}{3} x^{3} +8x^{2} +61x,\; C_{1} ,C_{2} ,C_{3} \in \RR \]

\end{solution}





\question[10] Find the points of maximum of the function
\[
F\left(u,v\right)=\sqrt{u}\left(\sqrt{u}-2\right)-\sqrt{v}\left(\sqrt{v}-2\right),
\]
given that  $\sqrt{u}\le 2,\ \sqrt{v}\le 2$


\begin{solution}
  \begin{enumerate}
  \item  Выполняем замену переменных $x=\sqrt{u},\ y=\sqrt{v}$. Далее путем несложных алгебраических преобразований выражения для целевой функции приводим ее к виду:$G\left(x,y\right)={\left(x-1\right)}^2-{\left(y-1\right)}^2+3$. При этом ограничения принимают вид $x\in \left[0,2\right],\ y\in \left[0,2\right]$

  \item  Определяем наличие экстремумов внутри области поиска, определенной ограничениями.

  \[\frac{\partial G\left(x,y\right)}{\partial x}=2\left(x-1\right),\ \frac{\partial G\left(x,y\right)}{\partial y}=-2\left(y-1\right),\ \frac{{\partial }^2G\left(x,y\right)}{\partial x\partial y}=\left( \begin{array}{cc}
  2 & 0 \\
  0 & -2 \end{array}
  \right)\]


Используя критерий Сильвестра, легко проверить, что в точке (1,1) у функции $G(x,y)$ нет экстремумов.

\item  Проанализируем наличие экстремумов на границах области поиска.

Граница$\left\{x=0,\ y\in \left[0,2\right]\right\},\ G\left(0,y\right)=1-{\left(y-1\right)}^2$. В точке (0, 1) достигается \textbf{\underbar{максимум}} равный 1, в конечных точках функция равна 0.

Граница $\left\{y=0,\ x\in \left[0,2\right]\right\},\ G\left(x,0\right)={\left(x-1\right)}^2-1$. В точке (1, 0) достигается минимум равный -- 1, в конечных точках функция равна 0.

Граница $\left\{x=2,\ y\in \left[0,2\right]\right\},\ G\left(0,y\right)=1-{\left(y-1\right)}^2$. В точке (2, 1) достигается \textbf{\underbar{максимум}} равный 1, в конечных точках функция равна 0.

Граница $\left\{y=2,\ x\in \left[0,2\right]\right\},\ G\left(x,2\right)={\left(x-1\right)}^2-1$. В точке (1, 2) достигаетсяминимум равный -- 1, в конечных точках функция равна 0.

\item  Возвращаемся к начальному преобразованию.
\end{enumerate}

Ответ: Точки максимума: (0, 1) и (4, 1)

Неполная попытка решить задачу не выполняя преобразование — не более 8 баллов за верный ответ.

\end{solution}

\newpage

\question
There are three coins in the bag. Two coins are unbiased, and for the third coin the probability of «head» is equal to $0.8$. James Bond chooses one coin at random from the bag and tosses it

\begin{parts}
\part[5] What is the probability that it will show «head»?
\begin{solution}
\[
\P(B) = \P(\text{head}) = \frac{2}{3} \cdot 0.5 + \frac{1}{3} \cdot 0.8 = \frac{18}{30} = \frac{3}{5} = 0.6
\]
\end{solution}

\part[5] What is the conditional probability that the coin is unbiased if it shows «head»?
\begin{solution}
\[
\P(A|B) = \frac{\P(A\cap B)}{\P(B)} = \frac{\frac{2}{3} \cdot 0.5}{0.6} = \frac{10}{18} = \frac{5}{9}
\]
\end{solution}

\end{parts}

\question
The pair of random variables $X$ and $Y$ with $\E(X)=0$ and $\E(Y)=1$ has the following covariance matrix
\[
\begin{pmatrix}
10 & -2 \\
-2 & 9
\end{pmatrix}.
\]

\begin{parts}
\part[5] Find $\Var(X+Y)$, $\Corr(X, Y)$, $\Cov(X - 2Y +1 , 7 + X + Y)$
\begin{solution}
  \[
  \Var(X+Y) = 10 + 9 - 2 \cdot 2 = 15
  \]
  \[
  \Corr(X, Y) = \frac{-2}{\sqrt{10\cdot 9}}
  \]
  \[
  \Cov(X - 2Y +1 , 7 + X + Y) = \Var(X) - 2\Cov(Y, X) + \Cov(X, Y) - 2 \Var(Y) = 10 - (-2) - 18 = -6
  \]
\end{solution}

\part[5] Find the value of $a$ if it is known that $X$ is independent of $Y - aX$.
\begin{solution}
  For independent variables the covariance is equal to zero: $\Cov(X, Y - aX) = 0$. Hence, $a = \frac{\Cov(X, Y)}{\Var(X)} = -0.2$.
\end{solution}

\end{parts}



\question You have  height measurements of a random sample of 100 persons, $y_1$, \ldots, $y_{100}$. It is known that $\sum_{i=1}^{100} y_i = 15800$ and $\sum_{i=1}^{100} y_i^2 = 2530060$.

\begin{parts}
\part[3] Calculate unbiased estimate of population mean and population variance of the height

\begin{solution}
Оценка среднего: $\bar y = 15800/100=158$.

Несмещённая оценка дисперсии
\[
\hat\sigma^2 = \frac{\sum (y_i-\bar y )^2}{n-1}=\frac{\sum y_i^2 - n \bar y^2}{n-1}=340
\]
\end{solution}
\part[3] At 4\% significance test the null-hypothesis that the population mean is equal to 155 cm, against two-sided alternative.

\begin{solution}
Наблюдаемое значение $Z$-статистики
\[
Z_{obs}=\frac{158-155}{\sqrt{340}/\sqrt{100}}=1.63
\]

Критическое значение $Z_{crit}=2.05$.

Вывод: гипотеза $H_0$ не отвергается.
\end{solution}
\part[2] Find the p-value
\begin{solution}
Находим по таблице, что площадь справа от $1.63$ примерно равна 5\%. Значит P-значение равно $10\%$.
\end{solution}
\part[2] Find the 96\% confidence interval for the population mean
\begin{solution}
Интервал имеет вид
\[
[158 - 2.05 \cdot \sqrt{340/100} ; 158 + 2.05 \cdot \sqrt{340/100} ]
\]
Итого: $[154.2;161.8]$
\end{solution}
\end{parts}


\question Density function of a random variable $Y$ is given by

\[
f(y)=
\begin{cases}
    \frac{1}{\theta^2} y e^{-y/ \theta}, \text{ if } y>0 \\
    0, \text{ otherwise } \\
\end{cases}
\]

You have 3 observations on $Y$: $y_1 = 48, y_2 = 50, y_3 = 52$.

\begin{parts}
\part[4] Using maximum likelihood, find the estimate of $\theta$
\begin{solution}

  Нахождение оценки:

  \[
    \begin{array}{l}
  	\ln(L) = \sum_{i=1}^n (-\ln(\theta^2)+\ln(y_i) - \frac{y_i}{\theta}) \\
  	\ln(L) = -2n\ln(\theta)+\sum_{i=1}^n \ln(y_i)  - \frac{\sum_{i=1}^n y_i}{\theta} \\
  	\frac{\partial \ln(L)}{\partial \theta} = - \frac{2n}{\theta} + \frac{\sum_{i=1}^n y_i}{\theta^2} = 0 \\
  	\widehat{\theta} = \frac{\sum_{i=1}^n y_i}{2n} = \frac{\overline{y}}{2} \\
    \end{array}
  \]

  Подставляя наши данные, получаем $\widehat{\theta} = 25$.
\end{solution}

\part[3] Is the estimator $\hat\theta$ unbiased?

\begin{solution}
  Несмещенность:

  \[
    \begin{array}{l}
  	\E(\widehat{\theta}) = \frac{\E(y_i)}{2}
    \end{array}
  \]

  Найдём математическое ожидание $y_i$:

  \[
    \begin{array}{l}
  	\E(y_i) = \int_0^{+ \infty} \frac{1}{\theta^2} y^2 e^{-y/ \theta} dy
    \end{array}
  \]

  Интегрируя по частям, получаем:

  \[
    \begin{array}{l}
  	\E(y_i) = \int_0^{+ \infty} \frac{2y}{\theta} e^{-y/ \theta} dy \\
  	\E(y_i) = \int_0^{+ \infty} 2 e^{-y/ \theta} dy \\
  	\E(y_i) = 2 \theta
    \end{array}
  \]

  Тогда $\E(\widehat{\theta}) = \frac{\E(y_i)}{2} = \theta$. Оценка несмещенная.

\end{solution}

\part[3] Calculate the variance of $\hat\theta$


\begin{solution}
  Для расчёта дисперсии вычислим $\E(y_i^2)$:

  \[
  	\E(y_i^2) = \int_0^{+ \infty} \frac{1}{\theta^2} y^3 e^{-y/ \theta} dy
  \]

  Аналогично предыдущему случаю, интегрируем по частям. Получаем:

  \[
  	\E(y_i^2) = 6 \theta^2
  \]

  Тогда

  \[
  	\Var(y_i) = 6 \theta^2 - 4 \theta^2 = 2 \theta^2
  \]

  И дисперсия оценки

  \[
  	\Var(\widehat{\theta}) = \Var(\frac{\overline{y}}{2}) = \frac{1}{4n} \Var(y_i) = \frac{\theta^2}{2n}
  \]


\end{solution}

\end{parts}


\end{questions}



\begin{figure}[b]
\caption{Distribution function of a standard normal random variable}
  \begin{minipage}[b]{0.35\linewidth}
    \centering
    \begin{tikzpicture}
% define normal distribution function 'normaltwo'
    \def\normaltwo{\x,{4*1/exp(((\x-3)^2)/2)}}

% input y parameter
    \def\y{4.4}

% this line calculates f(y)
    \def\fy{4*1/exp(((\y-3)^2)/2)}

% Shade orange area underneath curve.
    \fill [fill=gray!30] (2.6,0) -- plot[domain=0:4.4] (\normaltwo) -- ({\y},0) -- cycle;

% Draw and label normal distribution function
    \draw[domain=0:6] plot (\normaltwo) node[right] {};

% Add dashed line dropping down from normal.
    \draw[dashed] ({\y},{\fy}) -- ({\y},0) node[below] {$x$};

% Optional: Add axis labels
%    \draw (-.2,2.5) node[left] {$f_Y(u)$};
    \draw (3,2) node[below] {$F(x)$};

% Optional: Add axes
    \draw[->] (0,0) -- (6.2,0) node[right] {};
%    \draw[->] (0,0) -- (0,5) node[above] {};

\end{tikzpicture}
%    \rule{6cm}{6cm} %to simulate an actual figure
\par\vspace{0pt}
  \end{minipage}%
  \begin{minipage}[b]{0.60\linewidth}
    \centering
\begin{tabular}{rr|rr|rr|rr}
  \hline
$x$ & $F(x)$ & $x$ & $F(x)$ & $x$ & $F(x)$ & $x$ & $F(x)$ \\
  \hline
0.050 & 0.520 & 0.750 & 0.773 & 1.450 & 0.926 & 2.150 & 0.984 \\
  0.100 & 0.540 & 0.800 & 0.788 & 1.500 & 0.933 & 2.200 & 0.986 \\
  0.150 & 0.560 & 0.850 & 0.802 & 1.550 & 0.939 & 2.250 & 0.988 \\
  0.200 & 0.579 & 0.900 & 0.816 & 1.600 & 0.945 & 2.300 & 0.989 \\
  0.250 & 0.599 & 0.950 & 0.829 & 1.650 & 0.951 & 2.350 & 0.991 \\
  0.300 & 0.618 & 1.000 & 0.841 & 1.700 & 0.955 & 2.400 & 0.992 \\
  0.350 & 0.637 & 1.050 & 0.853 & 1.750 & 0.960 & 2.450 & 0.993 \\
  0.400 & 0.655 & 1.100 & 0.864 & 1.800 & 0.964 & 2.500 & 0.994 \\
  0.450 & 0.674 & 1.150 & 0.875 & 1.850 & 0.968 & 2.550 & 0.995 \\
  0.500 & 0.691 & 1.200 & 0.885 & 1.900 & 0.971 & 2.600 & 0.995 \\
  0.550 & 0.709 & 1.250 & 0.894 & 1.950 & 0.974 & 2.650 & 0.996 \\
  0.600 & 0.726 & 1.300 & 0.903 & 2.000 & 0.977 & 2.700 & 0.997 \\
  0.650 & 0.742 & 1.350 & 0.911 & 2.050 & 0.980 & 2.750 & 0.997 \\
  0.700 & 0.758 & 1.400 & 0.919 & 2.100 & 0.982 & 2.800 & 0.997 \\
   \hline
\end{tabular}
\par\vspace{0pt}
\end{minipage}
\label{fig:test}
\end{figure}

\begin{flushright}
Good luck!
\end{flushright}

\end{document}
