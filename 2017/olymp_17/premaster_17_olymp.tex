\documentclass[addpoints, answers]{exam} % добавить или удалить answers в скобках, чтобы показать ответы
% \documentclass[addpoints]{exam} % добавить или удалить answers в скобках, чтобы показать ответы

% DONE: 1, 2, 5, 6, 7, 8, 9, 10


\usepackage[utf8]{inputenc}
\usepackage[british, russian]{babel}
%\usepackage[OT1]{fontenc}
\usepackage{booktabs}
\usepackage{amsmath}
\usepackage{tikz}
\usepackage{amsfonts}
\usepackage{amssymb}
\usepackage{comment}
\usepackage[left=2cm, right=2cm, top=2cm, bottom=2cm]{geometry}
\DeclareMathOperator{\E}{\mathbb{E}}
\DeclareMathOperator{\Var}{\mathbb{V}\mathrm{ar}}
\DeclareMathOperator{\Cov}{\mathbb{C}\mathrm{ov}}
\DeclareMathOperator{\Corr}{\mathbb{C}\mathrm{orr}}
\DeclareMathOperator{\sgn}{sign}
\DeclareMathOperator{\tr}{tr}
\DeclareMathOperator{\rank}{rank}

\let\P\relax
\DeclareMathOperator{\P}{\mathbb{P}}
\newcommand{\cN}{\mathcal{N}}
\newcommand{\RR}{\mathbb{R}}
\newcommand{\hbeta}{\hat{\beta}}
\newcommand{\hb}{\hbeta}

\usepackage{floatrow}
%\newfloatcommand{capbtabbox}{table}[][\FBwidth]

\usepackage{lastpage}

%\usepackage{fancyhdr} % весёлые колонтитулы
%\pagestyle{fancy}
\lhead{Higher School of Economics}
\chead{}
\rhead{Maths for economists, olympiad 2017}
\lfoot{}
\cfoot{}
\rfoot{\thepage/\pageref{LastPage}}
%\renewcommand{\headrulewidth}{0.4pt}
%\renewcommand{\footrulewidth}{0.4pt}




\begin{document}







\hqword{Problem}
\hpgword{Page}
\hpword{Maximum}
\hsword{Points}
\htword{Total}
\pointname{\%}
%\renewcommand{\solutiontitle}{\noindent\textbf{Решение:}\par\noindent}
\renewcommand{\solutiontitle}{}

The table below is for grading. Please, leave it blank :)
\begin{center}
  \gradetable[h][questions]
\end{center}

You may write your solutions in Russian or in English.

\begin{flushright}
Good luck!
\end{flushright}


\begin{questions}

\question[10] Evaluate the following limit:

\[
\lim_{x \to \infty} \left( \sqrt{x+\sqrt{x}} - \sqrt{x} \right).
\]

\begin{solution}

The function may be written in the following way $\sqrt{x+\sqrt{x}} - \sqrt{x} = \sqrt{x} \left( \sqrt{1+x^{-1/2}} -1 \right)$. Using substituion $x^{-1/2} = y$ one may find the Taylor expansion of the expression inside brackets. This will give \textbf{(7 points)}
\[
\sqrt{x} \left( \left( 1+\frac{1}{2} \sqrt{\frac{1}{x}} + o \left(\sqrt{\frac{1}{x}}\right) \right) -1 \right).
\]

The limit is equal to \textbf{(3 points)}

\[
\lim_{x \to \infty} \sqrt{x} \left( \frac{1}{2} \sqrt{\frac{1}{x}} + o \left(\sqrt{\frac{1}{x}} \right) \right) = \frac{1}{2}.
\]

There is also another approach: multiply and divide by conjugate expression $\sqrt{x + \sqrt{x}} + \sqrt{x}$. Idea is worth \textbf{(5 points)}. All the rest is worth \textbf{(5 points)}.

\end{solution}

\question[10] Let $D(x)$ be so-called Dirichlet function, which equals 1 if its argument is rational and 0 otherwise, and let $k$ be a natural number.

Prove that the function $x^k D(x)$ is nowhere differentiable if $k=1$ and is differentiable only at $x=0$ if $k=2017$.

\begin{solution}

For every $k$ the function $x^k D(x)$ is continuous only at $x=0$. To see it one may sketch the graph of the function. So the function is not differentiable for $x\neq 0$. \textbf{(3 points)}

Let's consider the derivative at $x=0$ for $k=1$:

\[
\lim_{\Delta x \to 0} \frac{\Delta x D(\Delta x) - 0 \cdot D(0)}{\Delta x} = \lim_{\Delta x \to 0} D(\Delta x).
\]

If all $\Delta x$ are rational then $D(\Delta x)=1$.  If all $\Delta x$ are irrational then $D(\Delta x)=0$. The derivative does not exist at $x=0$. \textbf{(3 points)}

Let's consider the derivative at $x=0$ for $k=1$:

\[
\lim_{\Delta x \to 0} \frac{\Delta x^{2017} D(\Delta x) - 0 \cdot D(0)}{\Delta x} = \lim_{\Delta x \to 0} \Delta x^{2016} D(\Delta x).
\]

As  $D(\Delta x)$ is bounded and $\Delta x^{2016}$ tends to zero, their product tends to zero. So, the limit exists and the function is differentiable at $x=0$. \textbf{(4 points)}

\end{solution}


\question The matrices $A$ and $B$ are symmetric $3 \times 3$ matrices. Eigenvalues of the matrix $A$ are $\lambda^A_1 = 4, \lambda^A_2 = 2, \lambda^A_3 = 1$, eigenvalues of the matrix $B$ are $\lambda^B_1 = 11, \lambda^B_2 = 5, \lambda^B_3 = 1$.

\begin{parts}

\part[2] Find the trace (the sum of diagonal elements) of the matrix $A+B$.
\begin{solution}

We note that $\tr(A+B) = \tr(A) + \tr(B)$. Hence,
\[
\tr(A+B) = \sum_{i = 1}^3 \lambda^A_i + \sum_{i = 1}^3 \lambda^B_i = 7 + 17 = 24
\]

\end{solution}

\part[3]
Let the matrix $C$ be $3 \times 3$ matrix with $\det(C) = 1$. Find $\tr(C^{-1} A C)$.
\begin{solution}

Using the properties of trace, we get: $\tr(C^{-1} A C) = \tr(A C^{-1} C) = \tr(A) = 7$.

\end{solution}

\part[5] Prove that $\tr(A^k) = \sum_i \lambda_i^k$.
\begin{solution}

Diagonalize the matrix $A$: $A = P \Lambda P^{-1}$, where $P$ — the matrix of eigenvectors, $\Lambda$ — the diagonal matrix with eigenvalues on the diagonal. Hence, $A^k = P \Lambda^k P^{-1}$, where $\Lambda^k$ — diagonal matrix with k-th powers of eigenvalues on the diagonal. Then $\tr(A^k) = \tr(P \Lambda^k P^{-1}) = \tr(\Lambda^k) = \sum_i \lambda_i^k$.

\end{solution}

\end{parts}




\question Consider the matrix
\[
A = \begin{pmatrix}
-1 & 2 & 0\\
0 & 1 & 0\\
-18 & 18 & 8
\end{pmatrix}.
\]


\begin{parts}

\part[4] Find the eigenvalues and eigenvectors of matrix $A$.
\begin{solution}

$ det(A - \lambda I) = (1 - \lambda) (-1 - \lambda) (8 - \lambda) = 0$. Therefore, eigenvalues: $\lambda_1 = 1, \lambda_2 = -1, \lambda_3 = 8$. Eigenvectors are $v_1 = (1, 1, 0), v_2 = (1, 0, 2), v_3 = (0, 0, 1)$ respectively.

\end{solution}

\part[6] Find the matrix $A^{1/3}$.  By definition, $A^{1/3}$ is such a matrix that $(A^{1/3})^3 = A$.
\begin{solution}

Diagonalize matrix $A$: $A = P \Lambda P^{-1}$, where $P$ — matrix of eigenvectors, $\Lambda$ — diagonal matrix with eigenvalues on the diagonal.
$P = \begin{pmatrix}
1 & 1 & 0\\
1 & 0 & 0\\
0 & 2 & 1
\end{pmatrix}
$,
$P^{-1} = \begin{pmatrix}
0 & 1 & 0\\
1 & -1 & 0\\
-2 & 2 & 1
\end{pmatrix}
$,
$A^{1/3} = P \Lambda^{1/3} P^{-1}$, where $\Lambda^{1/3}$ is a diagonal matrix with $\lambda^{1/3}$ on the diagonal. Hence,
\[
A^{1/3} = \begin{pmatrix}
1 & 1 & 0\\
1 & 0 & 0\\
0 & 2 & 1
\end{pmatrix}
\begin{pmatrix}
1 & 0 & 0\\
0 & -1 & 0\\
0 & 0 & 2
\end{pmatrix}
\begin{pmatrix}
0 & 1 & 0\\
1 & -1 & 0\\
-2 & 2 & 1
\end{pmatrix} =
\begin{pmatrix}
-1 & 2 & 0\\
0 & 1 & 0\\
-6 & 6 & 2
\end{pmatrix}.
\]


\end{solution}

\end{parts}




\question[10] Solve the  differential equation:

\[
(3x^2y^4+2xy) \, dx + (2y^2-3x^2) \, dy = 0
\]


\begin{solution}
Заметим, что $y=0$ является решением уравнения. Запомнив это, поделим левую и правую часть исходного уравнения на $y^4$. Это интегрирующий множитель. Полученное уравнение имеет вид:
\[\left(3x^2+\frac{2x}{y^3}\right)dx+\left(\frac{2}{y^2}-\frac{3x^2}{y^4}\right)dy=0\]
Тогда, если $P\left(x,y\right)=\left(3x^2+\frac{2x}{y^3}\right)$ и $Q\left(x,y\right)=\left(\frac{2}{y^2}-\frac{3x^2}{y^4}\right)$, то $\frac{\partial P}{\partial y}=\frac{\partial Q}{\partial x}=-6\frac{x}{y^4}$. Имеем уравнение в полных дифференциалах.

Составляем и решаем систему уравнений для нахождения потенциала $u(x,y)$:
\[
\begin{cases}
\frac{\partial u}{\partial x}=3x^2+\frac{2x}{y^3} \\
\frac{\partial u}{\partial x}=\frac{2}{y^2}-\frac{3x^2}{y^4}.
\end{cases}
\]
Из первого уравнения находим, что $u\left(x,y\right)=x^3+\frac{x^2}{y^3}+h(y)$, где $h(y)$ — произвольная непрерывно дифференцируемая функция от $y$. Подставляем найденное во второе уравнение:
\[{u'}_y\left(x,y\right)=\frac{-3x^2}{y^4}+h'(y)\]
\[\frac{-3x^2}{y^4}+h'\left(y\right)=\frac{2}{y^2}-\frac{3x^2}{y^4}\]
\[h\left(y\right)=-\frac{2}{y}-C\]
$C$ — произвольная константа.

Тогда решение исходного уравнения
\[
x^3+\frac{x^2}{y^3}-\frac{2}{y}=C
\]
\end{solution}





\question Let $F(x, y) = xy$ and  $G(x, y; a, b) = y + bx - a$.

\begin{parts}

\part[3] For each value of parameters $(a, b)$ find the conditional extremum if it exists, classify it and find the extremal value $F^*(a, b)$.

\begin{solution}
One may use the Lagrange multiplier method or direct substitution or graphical analysis. There is no extremum for $b=0$. For $b>0$ there is one conditional maximum $(x^*, y^*) = (a/2b, a/2)$. For $b<0$ the same point is the conditional minimum of $F$. The optimal value is $F^*(a, b) = a^2/4b$.
\end{solution}

\part[4] Find all possible values of $F^*(a, b)$ in the region
\[
D_1 = \begin{cases}
a \geq b \\
a \in (0, 1) \\
\end{cases}
\]

\begin{solution}
We need to find the range of the fraction $a^2/4b$ in the region $D_1$.

Using negative values of $b$ we may set $F^*$ to arbitrary negative value.

As $a \neq 0$ we note that $F^*$ can't be equal to zero.

Let's consider positive $b$. If it exists, the minimal value of $F^*$ should be when $a=b$. In this case $F^* = b^2/4b=b/4$. We can take arbitrary small value of $b$, so on the line $a=b$ we may reach $F^* \in (0, 1/4)$. If we fix the value of $a$ and consider all $b \leq a$ then $F^* \in [1/4, +\infty)$.

Finally, $F^* \in (-\infty; 0) \cap (0; +\infty)$.
\end{solution}


\part[3] Find all possible values of $F^*(a, b)$ in the region
\[
D_2 = \begin{cases}
a \geq b \\
a \in (0, 1) \\
b \geq 0.5 \\
\end{cases}
\]

\begin{solution}
We need to find the range of the fraction $a^2/4b$ in the region $D_2$.

The minimum will be achieved on the line $a=b$, so $F^*_{min} = b^2/4b=b/4$. As $b \geq 0.5$ we see that $F^*_{min} = 1/8$.

The supremum will be achieved for $a=1$ and $b=0.5$, so $F^*_{sup} = a^2/4b=1/2$.

Finally, $F^* \in [1/8; 1/2)$.
\end{solution}


\end{parts}

\newpage

\question
The island is populated with knights and knaves. Each sentence of a knight is true with probability $0.9$ independently of other sentences. Each sentence of a knave is true with probability $0.2$ independently of other sentences.
The proportion of knights on the island is equal to $0.7$. You meet one person on the island at random and asked him, whether he is a knight.


\begin{parts}
\part[2] What is the probability that he will say «I am a knight»?
\begin{solution}
\[
\P(B) = 0.7 \cdot 0.9 + 0.3 \cdot 0.8 =  0.87
\]
\end{solution}

\part[4] What is the conditional probability that he is a knight given that he said «I am a knight»?
\begin{solution}
\[
\P(A|B) = \frac{\P(A\cap B)}{\P(B)} = \frac{0.7 \cdot 0.9}{0.87} = \frac{63}{87} \approx 0.72
\]
\end{solution}


\part[4] What is the conditional probability that he is a knight given that he said «I am a knight», paused and said «I am not a knight»?
\begin{solution}
\[
\P(A|C) = \frac{\P(A\cap C)}{\P(C)} = \frac{0.7 \cdot 0.9 \cdot 0.1}{0.7 \cdot 0.9 \cdot 0.1 + 0.3 \cdot 0.2 \cdot 0.8} =
\frac{63}{111}\approx 0.57
\]
\end{solution}

\end{parts}

\question
The joint density of random variables $X$ and $Y$ is given by the function
\[
f(x, y) = \begin{cases}
x + y, \text{ if } x \in [0; 1], \, y \in [0; 1] \\
0, \text{ otherwise } \\
\end{cases}
\]

\begin{parts}
\part[2] Are $X$ and $Y$ independent? Give short argument.
\begin{solution}
No, the density function can not be represented as a product $f(x,y) \neq f_X(x) f_Y(y)$.
\end{solution}

\part[4] Find $\P(Y > 2X)$ and $\E(XY)$
\begin{solution}
\[
\P(Y>2X) = \int_0^{1/2} \int_{2x}^1 x+y \, dy \, dx = \int_0^{1/2} x + 0.5 - 4x^2 \, dx = \frac{5}{24}
\]
\[
\E(XY)= \int_0^1 \int_0^1 xy(x + y) \, dx \, dy  = \frac{1}{3}
\]
\end{solution}

\part[4] Find marginal density $f_X(x)$ and conditional density $f_{Y|X}(y|x)$
\begin{solution}
\[
f_X(x) = \int_0^1 f(x, y) \, dy =  \begin{cases}
x + 0.5, \text{ if } x\in [0; 1] \\
0, \text{ otherwise } \\
\end{cases}
\]

\[
f(y|x)=\frac{f(x, y)}{f_X(x)}= \begin{cases}
\frac{x+y}{x+0.5}, \text{ if } x \in [0; 1], \, y \in [0; 1] \\
0, \text{ otherwise } \\
\end{cases}
\]
\end{solution}




\end{parts}



\question Boris loves hunting Pokemons. Today he randomly captured three Pokemons.

Boris has sorted Pokemons by their height in ascending order and obtained their ranks $H_i$. The lowest Pokemon gets the rank $H_i = 1$, the tallest gets the rank $H_i = 3$. After sorting Pokemons by their combat power Boris obtained the ranks $C_i$ in the same manner. Height and combat power of Pokemons are continuously distributed, so ties are impossible.

Boris would like to test the hypothesis $H_0$: height and combat power are independent. He calculates $\hat \rho$, sample Pearson correlation coefficient between ranks $C_i$ and $H_i$.

\begin{parts}
\part[6] Find the distribution of $\hat \rho$ under $H_0$, that is find all possible values of $\hat \rho$ and their probabilities.
\begin{solution}
Let's sort Pokemons by $H_i$ and consider possible orderings of $C_i$:

\begin{tabular}{ccc}
\toprule
Ordering of $C_i$ & Value of $\hat\rho$ & Probability \\
\midrule
1, 2, 3             &      $1$        &  $1/6$  \\
1, 3, 2             &      $0.5$        &  $1/6$  \\
2, 1, 3             &      $0.5$        &  $1/6$  \\
3, 2, 1             &      $-1$        &  $1/6$  \\
2, 3, 1             &      $-0.5$        &  $1/6$  \\
3, 1, 2             &      $-0.5$        &  $1/6$  \\
\bottomrule
\end{tabular}

So the distribution of $\hat\rho$ is

\begin{tabular}{ccccc}
\toprule
$\hat\rho$ & $-1$ & $-0.5$ & $0.5$ & $1$ \\
Probability & $1/6$ & $2/6$ & $2/6$ & $1/6$ \\
\bottomrule
\end{tabular}

\end{solution}

\part[4] Find the minimal threshold value $\rho^*$ that will be exceeded by $\hat\rho$ with probability less or equal to $0.2$ under $H_0$.

\begin{solution}
  From the distribution table we find that $\rho^* = 0.5$. Indeed $\P(\hat \rho > 0.5) = 1/6 \leq 0.2$. The value of $\rho^*$ can't be decreased further as probability would jump to $3/6$.
\end{solution}
\end{parts}


\question[10]

You estimated two models using $47$ observations:

\begin{enumerate}
\item[A.]  $\hat y_i=40+0.3x_i+0.8z_i-1.8w_i$, $R^2=0.82$

\item[B.]  $\hat y_i=65+0.6x_i+0.51z_i$, $R^2=0.7$
\end{enumerate}

Test the hypothesis $\beta_w=-1$ against $\beta_w \ne -1$ on 5\% significance level. Here $\beta_w$ is the coefficient before the variable $w$  in the first regression.

\begin{solution}

Поскольку необходимо проверить гипотезу, что $\beta_w=-1$, то необходимо использовать $t$-тест ($F$-тест для проверки одной гипотезы эквивалентен $t$-статистике в квадрате). Необходимая $t$-статистика рассчитывается по формуле $\frac{\hat{\beta}_{w} +1}{se(\hat{\beta }_{w} )}$, где нам не известен знаменатель.

Теперь обратимся к условию. Вторая модель является ограниченной (restricted) версией первой модели при ограничении $\beta_w=0$. Поэтому на основе данных об $R^2$ можно проверить гипотезу $\beta_w=0$ с помощью соответствующего $F$-теста. $F=\frac{(R_{UR}^{2} -R_{R}^{2} )/q}{(1-R_{UR}^{2} )/(n-k)} =\frac{(0.82-0.7)/1}{(1-0.82)/(47-4)} =\frac{0.12}{0.18/43} =29.27$.

Как было сказано выше для тестирования одного ограничения $F(1,n-k)=t_{n-k}^{2} $, следовательно $t_{43} =-5.41$. Мы берём отрицательное значение, так как сама оценка коэффициента отрицательная. Отсюда можно получить $se(\hat{\beta }_{w} )$, так как $\frac{\hat{\beta }_{w} }{se(\hat{\beta }_{w} )} =-5.41$, следовательно
\[
se(\hat{\beta }_{w} )=\frac{\hat{\beta }_{w} }{-5.41} =\frac{-1.8}{-5.41} =0.333.
\]
Исходя из полученных значений, необходимая $t$-статистика равна $\frac{-1.8+1}{0.333} =-2.403$. Критическое значение равно  $-2.018$, соответственно нулевая гипотеза отвергается на 5\% уровне значимости.
\end{solution}




\end{questions}


\begin{comment}

\begin{figure}[b]
\caption{Distribution function of a standard normal random variable}
  \begin{minipage}[b]{0.35\linewidth}
    \centering
    \begin{tikzpicture}
% define normal distribution function 'normaltwo'
    \def\normaltwo{\x,{4*1/exp(((\x-3)^2)/2)}}

% input y parameter
    \def\y{4.4}

% this line calculates f(y)
    \def\fy{4*1/exp(((\y-3)^2)/2)}

% Shade orange area underneath curve.
    \fill [fill=gray!30] (2.6,0) -- plot[domain=0:4.4] (\normaltwo) -- ({\y},0) -- cycle;

% Draw and label normal distribution function
    \draw[domain=0:6] plot (\normaltwo) node[right] {};

% Add dashed line dropping down from normal.
    \draw[dashed] ({\y},{\fy}) -- ({\y},0) node[below] {$x$};

% Optional: Add axis labels
%    \draw (-.2,2.5) node[left] {$f_Y(u)$};
    \draw (3,2) node[below] {$F(x)$};

% Optional: Add axes
    \draw[->] (0,0) -- (6.2,0) node[right] {};
%    \draw[->] (0,0) -- (0,5) node[above] {};

\end{tikzpicture}
%    \rule{6cm}{6cm} %to simulate an actual figure
\par\vspace{0pt}
  \end{minipage}%
  \begin{minipage}[b]{0.60\linewidth}
    \centering
\begin{tabular}{rr|rr|rr|rr}
  \hline
$x$ & $F(x)$ & $x$ & $F(x)$ & $x$ & $F(x)$ & $x$ & $F(x)$ \\
  \hline
0.050 & 0.520 & 0.750 & 0.773 & 1.450 & 0.926 & 2.150 & 0.984 \\
  0.100 & 0.540 & 0.800 & 0.788 & 1.500 & 0.933 & 2.200 & 0.986 \\
  0.150 & 0.560 & 0.850 & 0.802 & 1.550 & 0.939 & 2.250 & 0.988 \\
  0.200 & 0.579 & 0.900 & 0.816 & 1.600 & 0.945 & 2.300 & 0.989 \\
  0.250 & 0.599 & 0.950 & 0.829 & 1.650 & 0.951 & 2.350 & 0.991 \\
  0.300 & 0.618 & 1.000 & 0.841 & 1.700 & 0.955 & 2.400 & 0.992 \\
  0.350 & 0.637 & 1.050 & 0.853 & 1.750 & 0.960 & 2.450 & 0.993 \\
  0.400 & 0.655 & 1.100 & 0.864 & 1.800 & 0.964 & 2.500 & 0.994 \\
  0.450 & 0.674 & 1.150 & 0.875 & 1.850 & 0.968 & 2.550 & 0.995 \\
  0.500 & 0.691 & 1.200 & 0.885 & 1.900 & 0.971 & 2.600 & 0.995 \\
  0.550 & 0.709 & 1.250 & 0.894 & 1.950 & 0.974 & 2.650 & 0.996 \\
  0.600 & 0.726 & 1.300 & 0.903 & 2.000 & 0.977 & 2.700 & 0.997 \\
  0.650 & 0.742 & 1.350 & 0.911 & 2.050 & 0.980 & 2.750 & 0.997 \\
  0.700 & 0.758 & 1.400 & 0.919 & 2.100 & 0.982 & 2.800 & 0.997 \\
   \hline
\end{tabular}
\par\vspace{0pt}
\end{minipage}
\label{fig:test}
\end{figure}

\end{comment}


\end{document}
