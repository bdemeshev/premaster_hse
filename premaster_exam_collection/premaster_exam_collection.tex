\documentclass[pdftex,12pt,a4paper]{article}


\input{title_bor_utf8}

\title{Подборка вступительных экзаменов в магистратуру. \\Факультет экономики, НИУ-ВШЭ}
\author{Коллектив авторов}

\begin{document}
\parindent=0 pt % отступ равен 0

\maketitle

\tableofcontents

\section{Образец задания по высшей математике для программы <<Математическое моделирование>>(год?)}
\subsection{1 вариант}
\begin{enumerate}
\item Вычислить предел $\lim_{x\to 0}\frac{e^x \sin x-x(1+x)}{x^3}$.\\
\item Пусть A =
$\left(\begin{array}{cc}
9 & 1\\
1 & 9
\end{array}\right)$. Найти:
\begin{enumerate}
\item $A^{-1}$
\item $A^{10}$
\item Такую симметрическую неотрицательно определенную матрицу $B$, что $A=B\cdot B$
\item Такую симметрическую неотрицательно определенную матрицу $C$, что $A^{-1}=C\cdot C$.
\end{enumerate}
\item Найти и классифицировать точки экстремума функции $f(x,y)=x^3+y^3+3xy$.\\
\item Найти и классифицировать экстремумы функции $f(x,y)=9x^2+9y^2+2xy$ при ограничении 
$x^2+y^2=1$.\\
\item Решить дифференциальное уравнение $y'''-8y=0$.\\
\item Решить систему дифференциальных уравнений 
$\left\{ \begin{aligned}
\ddot{x}&=2y\\
\ddot{y}&=-2x\\
\end{aligned} \right.$\\
\item Безработный индивид с вероятностью 20\% находит работу в течение ближайшего месяца (независимо от того, сколько времени он уже ищет работу). Индивид, имеющий работу, теряет её в течение месяца с вероятностью 5\%. Известно, что на данный момент индивид Петя является безработным.
\begin{enumerate}
\item Какова вероятность того, что через два месяца Петя тоже будет безработным?
\item По прошествии двух месяцев выясняется, что Петя является безработным. Какова вероятность того, что месяц назад он работал (предполагается, что за месяц Петя может сделать только один переход между состояниями <<безработица>> и <<занятость>>)?
\end{enumerate}
\item Контрольные камеры ДПС на МКАД зафиксировали скорость движения шести автомобилей: 89, 83, 78, 96, 80, 78 км/ч. Предполагается, что скорость распределена по нормальному закону.
\begin{enumerate}
\item Постройте 95\% доверительный интервал для средней скорости автомобилей, если известно, что настоящая дисперсия равна 50 $($км/ч$)^2$.
\item Постройте 80\% доверительный интервал для дисперсии скорости.
\end{enumerate} 
\item Имеется множество C, состоящее из $n$ элементов. Сколькими способами можно выбрать в C два подмножества A и B так, чтобы
\begin{enumerate}
\item множества A и B не пересекались
\item множество A содержалось бы в множестве B?
\end{enumerate}
\item В дереве по 2010 вершин степеней 3, 4 и 5 и нет вершин больших степеней. Сколько в этом дереве может быть
\begin{enumerate}
\item вершин степени 1?
\item вершин степени 2?
\end{enumerate}
Укажите все возможные варианты ответа.
\end{enumerate}

\subsection{2 вариант}
\begin{enumerate}
\item Вычислить предел $\lim_{x\to +\infty} x \sqrt{x}(\sqrt{x+1}+\sqrt{x-1}-2\sqrt{x})$.\\
\item Пусть $A=
\left(\begin{array}{ccc}
2 & 1 & 1\\
1 & 2 & 1\\
1 & 1 & 2
\end{array}\right)$ и $x=(x_1 x_2 x_3)^T$. Привести квадратичную форму $f(x)=x^T Ax$ к каноническому виду при помощи ортогонального преобразования (требуется указать, как сам канонический вид квадратичной формы, так и ортогональное преобразование, которое приводит форму к каноническому виду).\\
\item Найти и классифицировать точки экстремума функции $f(x,y)=3x^2-2x\sqrt{y}+y-8x+8$.\\
\item Найти и классифицировать экстремумы функции $f(x,y,z)=2x-y+9z^2$ при двух ограничениях $y+6xz=-1$ и $3z-2x=1$.\\
\item Решить дифференциальное уравнение $(x^2-y^2)dy + 2xydx=0$.\\
\item Решить систему дифференциальных уравнений 
$\left\{
\begin{aligned}
\ddot{x}+\dot{x}+\dot{y} & = 7\\
\dot{x}+\ddot{y}         & = e^t\\
\end{aligned}\right.$\\
\item Фирма производит микросхемы. Известно, что производство микросхем может находиться в одном из двух состояниях: нормальном (доля дефектных микросхем 10\%) и проблемном (доля дефектных микросхем 55\%). Для контроля состояния производства утром производится случайная выборка размером в 10 микросхем из продукции первого часа работы. Если из них 3 и более дефектные, производство останавливается до выяснения причины проблемы.
\begin{enumerate}
\item Найдите вероятность ложного срабатывания тревоги.
\item Найдите вероятность того, что проблемное состояние не будет идентифицировано.
\end{enumerate}
\item Доходность ценных бумаг на New York Фондовой бирже имеет нормальное распределение. В таблице приведены данные о доходности 10 видов ценных бумаг:\\ \\
\begin{tabular}{c|ccccccccccc}
  № & 1 & 2 & 3 & 4 & 5 & 6 & 7 & 8 & 9 & 10 & $\sum$ \\
  \hline
  X & 10 & 16 & 5 & 10 & 12 & 8 & 4 & 6 & 5 & 4 & 80 \\
  $X^2$ & 100 & 256 & 25 & 100 & 144 & 64 & 16 & 36 & 25 & 16 & 782\\
\end{tabular} \\

\begin{enumerate}
\item Найти точечные несмещенные и состоятельные оценки для математического ожидания и дисперсии доходности.
\item Найти 90\% доверительный интервал для математического ожидания доходности.
\end{enumerate}
\item Пусть $X_1,..,X_n$ --- выборка из нормально распределенной генеральной совокупности,т.е. $X_i~N(\mu,\sigma^2), i=1,...,n.$\\
Построены следующие оценки для математического ожидания $\mu$:\\
$\mu_1=\bar{X}, \mu_2=X_1, \mu_3=\frac{X_1}{2}+\frac{1}{2(n-1)}(X_2+...+X_n)$.
\begin{enumerate}
\item Какая из этих оценок является несмещенной?
\item Какая из этих оценок является наиболее эффективной?
\item Какая из этих оценок является состоятельной?
\end{enumerate}

\item Оценка зависимости выпуска фирмы от капитальных и трудовых затрат вида $Q=AK^{\beta_2}L^{\beta_3}$ с помощью модели $\ln Q=\beta_1+\beta_2\ln K+\beta_3\ln L+u$ по 40 наблюдениям дала следующие результаты (в скобках указаны стандартные ошибки коэффициентов регрессии):\\
$\ln Q=1.37+0.632 (0.257)\ln K+0.452(0.219)\ln L, R^2=0.98, \widehat{\Cov}(\hat{\beta}_2,\hat{\beta}_3)=-0.044$\\
\\На уровне значимости 5 \% проверить гипотезы
\begin{enumerate}
\item о значимости вклада труда/капитала в формирование выпуска
\item о наличии постоянной отдачи от масштаба.
\end{enumerate}

\end{enumerate}


\section{2007}

\subsection{17.07.2007, вариант A}
\begin{enumerate}
\item Найдите 
\begin{equation}
\lim_{x \to 2}\frac{\sqrt[3]{6+x}-\sqrt{x+2}}{\sqrt[3]{25+x}-\sqrt{7+x}}
\end{equation}
\item Матрица вида А=
$\left( \begin{array}{cc}
a & b\\
c & -a
\end{array} \right)$ 
имеет собственное значение $\lambda_1=3$, которому соответствует собственный вектор 
$\left( \begin{array}{c}
1\\
1
\end{array} \right).$ Второй собственный вектор этой матрицы --- 
$\left( \begin{array}{c}
1\\
-2
\end{array} \right).$ 
Вычислите определитель данной матрицы.\\
\item Найдите стационарные точки функции $f(x,y)=e^{2x+3y}(8x^2-6xy+3y^2)$ и определите их тип.\\
\item Найдите минимумы и максимумы функции $f(x,y)=2x^2-3xy-2y^2$ при ограничении $x^2+y^2=10$.\\
\item Найдите решение дифференциального уравнения $xy'-6y=10x^4-16x^2$, удовлетворяющее условию $y(1)=0$. Постройте эскиз графика данного решения. \\
\item Найдите общее решение дифференциального уравнения $y''-4y'+3y=e^{2x}$.\\
\item Время, проводимое покупателем в супермаркете, можно считать нормально распределенной случайной величиной. Известно, что математическое ожидание этой случайной величины составляет 1 час 20 минут, а стандартное отклонение равно 15 минутам. Найдите при этих условиях вероятность того, что из трех незнакомых между собой покупателей хотя бы один проведет в супермаркете более полутора часов.\\
\item Функция плотности двумерной случайной величины $(X,Y)$ имеет вид\\
$p(x,y)=\begin{cases}
c(-x),\text{ если } x\in [0;1],y \in [1;2]\\
0,\text{ иначе }
\end{cases}$\\
Найдите:
\begin{enumerate}
\item значение константы с
\item вероятность того, что $Y\leq2X$
\item математическое ожидание $\E(Y)$
\end{enumerate}
\item Проверка 175 старых домов города показала, что в 56 из них электропровода требуют срочного ремонта. На уровне значимости 5\% проверьте гипотезу о том, что доля всех старых домов города, в которых требуется срочный ремонт электропроводки, составляет не менее 33\%.\\
\item По данным 22 наблюдений в рамках классической нормальной линейной регрессии была получена модель $\hat{Y}_i=\hat{\alpha}+\hat{\beta}X_i$ с коэффициентом детерминации $R^2=0.93$. Проверьте гипотезу об адекватности этой регресии
\end{enumerate}

\subsection{17.07.2007, вариант A1}
\begin{enumerate}
\item Найдите 
\begin{equation}
\lim_{x\to 0}\frac{\sqrt{1+\tg x}-\sqrt{1-\tg x}}{\sin x}
\end{equation}
\item Матрица вида А=
$\left( \begin{array}{cc}
0.6 & 0.8\\
c & d
\end{array} \right)$ 
коммутирует с матрицей B=
$\left( \begin{array}{cc}
0 & 1\\
-1 & 0
\end{array} \right)$, то есть выполняется равенство $A\cdot B=B\cdot A$. Найдите константы $c,d$ и матрицу $A^{-2}$ (матрицу, обратную матрице $A^2=A\cdot A$).\\

\item Найдите стационарные точки функции $f(x,y)=x^3+y^3+21x^2+18xy+21y^2$ и определите их тип.\\
\item Найдите минимумы и максимумы функции $f(x,y)=x^3\cdot y^4$ в области $x>0,y>0$ при ограничении $3x+4y=7$.\\
\item Найдите решение дифференциального уравнения $xy'+3y=x^2$, удовлетворяющее условию $y(1)=1/3$. Постройте эскиз графика найденного решения. \\
\item Найдите общее решение дифференциального уравнения $y''-y=e^{2x}$.\\
\item Синоптики Аляски и Чукотки независимо друг от друга предсказывают погоду (<<ясно>> или <<пасмурно>>) в Беринговом проливе, ошибаясь с вероятностями 0.1 и 0.2 соответственно. Их предсказания на завтра разошлись. Какова вероятность того, синоптики Аляски ошиблись?\\
\item Функция плотности двумерной случайной величины $(X,Y)$ имеет вид\\
$p(x,y)=\begin{cases}
c,\text{ при } x\in [0;1],y \in [0;1], y\geq x\\
0,\text{ в остальных случаях }
\end{cases}$\\
Найдите:
\begin{enumerate}
\item значение константы с
\item вероятность того, что $Y\leq 2X$
\item математическое ожидание $\E(Y)$
\end{enumerate}
\item Медицинское обследование 180 пациентов показало, что у 63 из них наблюдалось улучшение состояния после лечения новым препаратом. Найдите 95\% доверительный интервал для теоретической доли тех пациентов, у которых може наблюдаться такое улучшение.\\
\item По данным 30 наблюдений в рамках классической нормальной линейной регрессии была получена модель $\hat{Y}_i=\hat{\alpha}+\hat{\beta}X_i$, причём $\hat{\beta}=2.04, \hat{\sigma}_{\hat{\beta}}=0.75$. Проверьте адекватность этой регрессии.
\end{enumerate}


\subsection{17.07.2007, вариант В}
\begin{enumerate}
\item Найдите предел
\begin{equation}
\lim_{x\to 0} \frac{\sin 3x-\tg 4x}{\arcsin 2x}
\end{equation}
\item Найдите матрицу $X$ из уравнения $A\cdot X\cdot A'=B$, где $A=\left(\begin{array}{cc}
2 & 7\\
1 & 4
\end{array}\right), B=\left(\begin{array}{cc}
7 & 2\\
2 & 0
\end{array}\right), A'=\left(\begin{array}{cc}
2 & 1\\
7 & 4
\end{array}\right)$ --- матрица, транспонированная по отношению к матрице А.\\
\item Найдите минимумы и максимумы функции $f(x,y)=\ln x+\ln y-3x-y-6xy$.\\
\item Найдите решение дифференциального уравнения $y''+y'-2y=3$, удовлетворяющее условиям $y(0)=0$, $y'(0)=0$. Постройте эскиз графика найденного решения.\\
\item Время обслуживания одного вызова на междугородней телефонной станции можно считать нормально распределенной случайной величиной. При этом математическое ожидание составляет 1,5 минуты, а стандартное отклонение равно 0,5 минуты. Найдите при этих условиях вероятность того, что время обслуживания хотя бы одного из двух независимых вызовов составит более двух минут.\\
\item Функция плотности случайной величины $X$ имеет вид\\
$f(x,y)=\begin{cases}
c(1-|x|), |x|\leq 1\\
0, |x|>1.
\end{cases}$\\
Найдите значение константы $c$, математическое ожидание и дисперсию величины $X$.\\
\item Для случайной выборки, состоящей из 8 наблюдений, извлеченных из нормальной генеральной совокупности, был получен следующий 90\% доверительный интервал для математического ожидания $\mu$: $18,1<\mu< 18.9$. Постройте 95\% доверительный интервал для этого математического ожидания.\\
\item По данным 26 наблюдений в рамках классической нормальной регрессии была получена модель $\hat{Y}_i=\hat{\alpha}+\hat{\beta}X_i$, в которой $\hat{\beta}=2.0598$, $\hat{\sigma}_{\hat{\beta}}=0.0153$. На уровне значимости 5\% проверьте гипотезу
$\begin{aligned}
H_0&: \beta=2\\
H_a&: \beta\ne 2
\end{aligned}$
\end{enumerate}

\subsection{17.07.2007, вариант В1}
\begin{enumerate}
\item  Найдите предел
\begin{equation}
\lim_{x\to 0} \frac{\sqrt{1+x+x^2}-1}{x}
\end{equation}
\item Матрица $A=\left(\begin{array}{cc}
4 & a \\ 
-6 & -a
\end{array}\right)$ имеет собственное значение $\lambda=1$. Найдите константу $a$ и собственные значение матрицы $A^{-1}$.
\item Найдите стационарные точки в области $x>0$, $y>0$ и определите их тип для функции
\begin{equation}
f(x,y)=x^2y(4-x-y)
\end{equation}
\item Найдите решение дифференциального уравнения $y''+2y'+y=1$, удовлетворяющее условиям $y(0)=-1$, $y'(0)=4$. Постройте эскиз графика найденного решения.
\item Синоптики Аляски и Чукотки независимо друг от друга предсказывают погоду (<<ясно>> или <<пасмурно>>) в Беринговом проливе, ошибаясь с вероятностями 0.1 и 0.2 соответственно. Их предсказания на завтра совпали. Какова вероятность того, что эти предсказания ошибочны?
\item Функция плотности случайной величины $X$ имеет вид
\begin{equation}
f(x,y)=\left\{\begin{array}{c}
cx^2,\quad x\in[0;1] \\ 
0,\quad x\notin [0;1]
\end{array} \right.
\end{equation}
Найдите значение константы $c$, математическое ожидание и дисперсию величины $X$.
\item Для случайной выборки из 8 автомобилей средняя скорость на определенном участке трассы составила $\bar{X}=115$ км/ч, а выборочное стандартное отклонение --- $\hat{\sigma}=2$ км/ч. Предполагая нормальность закона распределения скорости постройте 95\% доверительный интервал для математического ожидания $\mu$ скорости.
\item По 28 наблюдениям в рамках классической нормальной линейной регрессии была получена модель $\hat{Y}_i=\hat{\alpha}+\hat{\beta}X_i$, в которой $\hat{\beta}=1.57$, $\hat{\sigma}_{\hat{\beta}}=0.05$. Вычислите коэффициент детерминации $R^2$.
\end{enumerate}

\subsection{17.07.2007, вариант В2}
\begin{enumerate}
\item  Найдите предел
\begin{equation}
\lim_{x\to 0} \frac{\sqrt{1+x}-\sqrt{1+x^2}}{\sqrt{1+x}-1}
\end{equation}
\item Матрица вида $A=\left(\begin{array}{cc}
2a & 1 \\ 
-3a & -1
\end{array}\right)$ имеет собственное значение $\lambda=2$. Найдите константу $a$ и собственные значение матрицы $A^{-1}$ (матрицы, обратной к матрице $A$).\\
\item Найдите стационарные точки функции $f(x,y)=x^2+xy-9x-3y$  и определите их тип.\\
\item Найдите решение дифференциального уравнения $y''-16y=32$, удовлетворяющее условиям $y(0)=0$, $y'(0)=0$. Постройте эскиз графика найденного решения.\\
\item Синоптики Аляски и Чукотки независимо друг от друга предсказывают погоду (<<ясно>> или <<пасмурно>>) в Беринговом проливе, ошибаясь с вероятностями 0.05 и 0.1 соответственно. Их предсказания на завтра совпали. Какова вероятность того, что эти предсказания верны?\\
\item Функция плотности случайной величины $X$ имеет вид
$p(x,y)=\begin{cases}
c/x^2,\text{ при } x\in [1;2]\\
0,\text{ в остальных случаях }
\end{cases}$
Найдите значение константы $c$, математическое ожидание и дисперсию случайной величины $X$.\\
\item Для случайной выборки из 10 студентов средний балл за контрольную работу составил $\bar{X}=71.2$ балла (по шкале в 100 баллов), причем выборочное стандартное отклонение $\hat{\sigma}=15.4$ балла. Постройте 95\% доверительный интервал для математического ожидания $\mu$ балла $X$ за эту контрольную работу (в предположении нормального закона распределения случайной величины $X$).\\ 
\item По 32 наблюдениям в рамках классической нормальной линейной регрессии была получена модель $\hat{Y}_i=\hat{\alpha}+\hat{\beta}X_i$, в которой $\hat{\beta}=0.32$, $\hat{\sigma}_{\hat{\beta}}=0.04$. Вычислите коэффициент детерминации $R^2$.
\end{enumerate}



\section{2008}

\subsection{22.07.2008, вариант А}
\begin{enumerate}
\item  Найдите предел 
\begin{equation}
\lim_{x\to 0} \left(\frac{1+6x+5x^2}{1-3x-x^2} \right)^{-3/x}
\end{equation}
\item В чемпионате по шахматам участвовало $n$ участников. Каждый участник сыграл с каждым один раз. Известно, что ничьих не было. По результатам матча судья составил матрицу $A$ размера $n\times n$ по принципу: $a_{ij}=1$, если игрок $i$ выиграл у игрока $j$; $a_{ij}=-1$, если игрок $i$ проиграл игроку $j$; диагональные элементы равны нулю, $a_{ii}=0$.
\begin{enumerate}
\item Найдите $\det(A)$ при $n=2$
\item Найдите $A+A^{t}$
\item Найдите $\det(A)$ при $n=1111$
\end{enumerate}
\item Найдите локальные минимумы и максимумы функции $f(x,y)=(x^2+y^2)e^{-4x^2-y^2}$
\item С помощью метода множителей Лагранжа найдите условные минимумы и максимумы функции $f(x,y,z)=5x^3y^5z^3$ при ограничении $x+y+5z=110$.
\item Решите дифференциальное уравнение $y'''+6y''-7y'=14+8e^{x}$
\item Для дифференциального уравнения $y'=(y+x)/(y-x)$ найдите
\begin{enumerate}
\item общее решение
\item частное решение, проходящее через точку $(x;y)=(0;1)$
\end{enumerate}
\item На острове Двупогодном погода бывает двух видов: пасмурная и ясная. Первого января губернатор острова «разгоняет»  тучи, поэтому в этот день на острове всегда ясно. В каждый последующий день погода меняется случайным образом согласно двум закономерностям. После пасмурного дня ясный наступает с  вероятностью 0.3, после ясного дня ясный наступает с вероятностью 0.8. 
\begin{enumerate}
\item Какова вероятность того, что второе января будет ясное?
\item Какова вероятность того, что второе января было ясное, если известно, что третье -- было пасмурное?
\end{enumerate}
\item Задана совместная функция плотности случайных величин $X$ и $Y$:
\begin{equation}
f_{X,Y}(x,y)=\left\{\begin{array}{c}
cx+y/2,\quad x,y\in [0;1] \\ 
0,\quad \text{иначе}
\end{array}  \right.
\end{equation}
Найдите $c$, $\E(XY)$ и $\P(XY<1/2)$
\item Исследуется зависимость спроса $Q$ на некоторый товар от его цены $P$. Предположим, что модель $\ln(Q)=\alpha+\beta\ln(P)+\varepsilon$ удовлетворяет всем условиям классической линейной регрессионной модели с нормально распределенной случайной ошибкой. Функция спроса оценивается по 10 наблюдениям. Известно, что 99\% доверительный интервал для коэффициента эластичности $\beta$ равен $(-1.44;-0.88)$.
\begin{enumerate}
\item Определите значение оценки $\hat{\beta}$ и оценки ее дисперсии.
\item Можно ли утверждать, что спрос зависит от цены товара?
\end{enumerate}
\item Распределение заработной платы работников подчиняется закону Эрланга с функцией плотности $f(x)=\frac{x}{\lambda^2}\exp(-x/\lambda)$ при $x>0$. Оцените значение параметра $\lambda$ по выборке $X_1$, $X_2$, \ldots, $X_n$ методом максимального правдоподобия. Будет ли полученная оценка несмещенной?

Примечание: $\int_0^{\infty} x^n \exp(-x/\lambda)=\lambda^{n+1}n!$
\end{enumerate}

\subsection{22.07.2008, вариант B}
\begin{enumerate}
\item  Найдите предел 
\begin{equation}
\lim_{x\to 0} \left(\frac{1+7x+2x^2}{1-x-x^2} \right)^{5/x}
\end{equation}
\item В чемпионате по шахматам участвовало $n$ участников. Каждый участник сыграл с каждым один раз. Известно, что ничьих не было. По результатам матча судья составил матрицу $A$ размера $n\times n$ по принципу: $a_{ij}=1$, если игрок $i$ выиграл у игрока $j$; $a_{ij}=-1$, если игрок $i$ проиграл игроку $j$; диагональные элементы равны нулю, $a_{ii}=0$.
\begin{enumerate}
\item Найдите $\det(A)$ при $n=2$
\item Найдите $A+A^{T}$
\item Найдите $\det(A)$ при $n=231$
\end{enumerate}
\item Найдите локальные минимумы и максимумы функции $f(x,y)=(x^2+y^2)e^{-x^2-9y^2}$
\item С помощью метода множителей Лагранжа найдите условные минимумы и максимумы функции $f(x,y,z)=6xy^{11}z^3$ при ограничении $x+2y+z=60$.
\item Решите линейное дифференциальное уравнение $y'''-3y''+2y'=4-e^{x}$
\item Для дифференциального уравнения $y'=(y+x)/(y-x)$ найдите
\begin{enumerate}
\item Найдите все решения дифференциального уравнения $y'=\frac{y+7x}{y-x}$
\item Выберите из них решение, проходящее через точку $(x;y)=(0;1)$
\end{enumerate}
\item На острове Двупогодном погода бывает двух видов: пасмурная и ясная. Первого января губернатор острова «разгоняет»  тучи, поэтому в этот день на острове всегда ясно. В каждый последующий день погода меняется случайным образом согласно двум закономерностям. После пасмурного дня ясный наступает с  вероятностью 0.4, после ясного дня ясный наступает с вероятностью 0.9. 
\begin{enumerate}
\item Какова вероятность того, что второе января будет ясное?
\item Какова вероятность того, что второе января было ясное, если известно, что третье -- было пасмурное?
\end{enumerate}
\item Задана совместная функция плотности случайных величин $X$ и $Y$:
\begin{equation}
f_{X,Y}(x,y)=\left\{\begin{array}{c}
cx+y/2,\quad x,y\in [0;1] \\ 
0,\quad \mbox{иначе}
\end{array}  \right.
\end{equation}
Найдите $c$, $\E(XY)$ и $\P(XY<1/2)$
\item Распределение доходов некоторой группы населения подчиняется закону Парето с $f(x)=\frac{1}{2\gamma}\left(\frac{2}{x}\right)^{1+1/\gamma}, x>2, 0<\gamma<1$.
Требуется оценить значение параметра $\gamma$ с помощью метода максимального правдоподобия по данным случайной выборки $n$ налоговых деклараций, заполненных респондентами из исследуемой доходной группы. Будет ли полученная оценка несмещенной?
\item Исследуется зависимость спроса $Q$ на некоторый товар от его цены $P$. Предположим, что модель $\ln(Q)=\alpha+\beta\ln(P)+\varepsilon$ удовлетворяет всем условиям классической линейной регрессионной модели с нормально распределенной случайной ошибкой. Функция спроса оценивается методом наименьших квадратов по 10 наблюдениям. Доверительный интервал для коэффициента эластичности $\beta$, соответствующий уровню доверия 95\%, принимает значение $(-2.4350,-1.8802)$. 
\begin{enumerate}
\item Определите значение МНК-оценки и оценки ее дисперсии для коэффициента эластичности.
\item На уровне значимости 1\% проверить гипотезу о единичной эластичности.
\end{enumerate}
\end{enumerate} 

\subsection{22.07.2008, вариант C}
\begin{enumerate}
\item  Найдите предел 
\begin{equation}
\lim_{x\to 0} \frac{e^{7x}+4e^{5x}-5e^{-4x}}{\arcsin \left(\arcsin(6x)\right)}
\end{equation}
\item Известно, что $A=
\left(\begin{array}{cc}
0.9 & 0.5\\
0.1 & 0.5
\end{array}\right)$.
\begin{enumerate}
\item Найдите собственные значения и собственные векторы матрицы $A$
\item Надйите $\lim_{n\to \infty} A^n$
\end{enumerate}
\item Найдите локальные минимумы и максимумы функции $f(x,y)=2x^5+5y^2+\frac{10}{xy}$\\
\item Найдите решение дифференциального уравнения $y'=\frac{6y-2xy^3-\sin(xy)-xy\cos(xy)}{3x^2y^2-6x+x^2\cos(xy)}$\\
\item Вася кидает дротик в мишень три раза. Известно, что во второй раз он попал ближе к центру, чем в первый раз. Какова условная вероятность того, что в третий раз он попадёт ближе к центру, чем в первый раз?\\
Указание: Предположить, что результаты бросков (расстояние от дротика до центра мишени) независимы друг от друга и имеют одинаковое непрерывное распределение.\\
\item Задана функция плотности случайной величины $X$:\\
$f_X(x)=\begin{cases}
c(x+x^2), x \in [0;1]
0, \text{иначе}
\end{cases}$.
\begin{enumerate}
\item Найдите значение константы $c$
\item Найдите  $\E(X^2)$
\item Найдите $\P(X>0.5)$
\end{enumerate}
\item Используя ежегодные данные об объеме импорта товаров $Y$ в личном располагаемом доходе $X$ в США за 1978 --- 1997 годы и предполагая,что модель $Y=\alpha+\beta X+\varepsilon$ удовлетворяет всем условиям классической линейной регрессионной модели с нормально распределенной ошибкой, исследователь получил методом наименьших квадратов следующее уравнени регрессии: $Y=-261.09+0.2452X$. Оценка дисперсии случайной ошибки $\varepsilon$ --- $\hat{\sigma}^2=475.48$, коэффициент детерминации $R^2=0.9388$.
\begin{enumerate}
\item На уровне значимости 5\% проверить гипотезу о независимости объема импорта от личного располагаемого дохода.
\item Предполагая, что в 1998 году располагаемый доход составил 2800 млрд. долларов, вычислить прогнозное значение для ожидаемого объема импорта. Какова точность полученного прогноза?
\end{enumerate}
\item В рекламе утверждалось, что из двух типов пластиковых карт: <<Visa>> и <<American Express>> богатые люди предпочитают второй, т.е. владельцы второго типа карт ежемесячно тратят больше денег. Выборочное обследование показало, что ежемесячные расходы по картам каждого типа достаточно хорошо описываются нормальным законом распределения. Средние месячные расходы 31 обладателя <<Visa>> оказались равны \$500 при выборочной дисперсии 39000 $\$^2$, а среднемесячные расходы 29 обладателей <<American Express>> --- \$ 580 при выборочной дисперсии 32000 $\$^2$. Проверить утверждение рекламы при 5\% уровне значимости.
\end{enumerate} 

\subsection{22.07.2008, вариант D}
\begin{enumerate}
\item  Найдите предел 
\begin{equation}
\lim_{x\to 0} \frac{e^{-7x}-3e^{-5x}+2e^{4x}}{\tg\tg(5x)}
\end{equation}
\item Известно, что $A=
\left(\begin{array}{cc}
0.3 & 0.5\\
0.7 & 0.5
\end{array}\right)$.
\begin{enumerate}
\item Найдите собственные значения и собственные векторы матрицы $A$
\item Надйите $\lim_{n\to \infty} A^n$
\end{enumerate}
\item Найдите локальные минимумы и максимумы функции $f(x,y)=2x^3+3y^2+\frac{6}{xy}$\\
\item Найдите решение дифференциального уравнения 
\begin{equation}
y'=\frac{18x^2y-y-2xy^2\cos (x^2y)}{x-6x^3+\sin(x^2y)+x^2y\cos(x^2 y)}
\end{equation} 
\item Вася кидает дротик в мишень три раза. Известно, что во второй раз он попал дальше от центра, чем в первый раз. Какова условная вероятность того, что в третий раз он попадёт ближе к центру, чем в первый раз?\\
Указание: Предположить, что результаты бросков (расстояние от дротика до центра мишени) независимы друг от друга и имеют одинаковое непрерывное распределение.\\
\item Задана функция плотности случайной величины $X$:\\
$f_X(x)=\begin{cases}
c(x+x^3), x \in [0;1] \\
0, \text{иначе}
\end{cases}$.
\begin{enumerate}
\item Найдите значение константы $c$
\item Найдите  $\E(X^2)$
\item Найдите $\P(X>0.5)$
\end{enumerate}
\item Используя ежегодные данные об объеме импорта товаров $Y$ в личном располагаемом доходе $X$ в США за 1978 --- 1997 годы и предполагая,что модель $Y=\alpha+\beta X+\varepsilon$ удовлетворяет всем условиям классической линейной регрессионной модели с нормально распределенной ошибкой, исследователь получил методом наименьших квадратов следующее уравнение регрессии: $Y=-261.09+0.2452X$ и оценку дисперсии $\widehat{\Var}(\hat{\beta})=0.0004$. 
\begin{enumerate}
\item Построить 95\% доверительный интервал для коэффициента наклона.
\item На уровне значимости 5\% проверить гипотезу о зависимости объема импорта от личного располагаемого дохода.
\end{enumerate}
\item Изучается эффективность нового метода обучения. У группы из 40 студентов, обучавшихся по новой методике, средний балл на экзамене составил 322.12, а выборочное стандартное отклонение --- 54.53. Аналогичные показатели для независимой выборки из 60 студентов того же курса, обучавшихся по старой методике, приняли значения 304.61 и 62.61 соответственно. Предполагая, что экзаменационный балл случайно выбранного студента хорошо описывается нормальным законом, проверить гипотезу об эффективности новой методики.
\end{enumerate} 


\section{2010}
\subsection{07.2010, вступительный экзамен}

! Здесь не хватает 9 и 10 задач

\begin{enumerate}
\item $\lim_{x\to 0}\frac{e^{x}\sin(x)-x(1+x)}{x^{3}}$

Ответ: $\frac{1}{3}$

\item Пусть $A=\left(\begin{array}{cc}9 & 1 \\ 1 & 9 \end{array}\right)$. Найдите $A^{-1}$, $A^{10}$, $A^{0.5}$, $A^{-0.5}$

Решение: $\lambda_{1}=8$, $\lambda_{2}=10$, $v_{1}=(1,-1)$, $v_{2}=(1,1)$

\item Найдите и классифицируйте экстремумы $f(x,y)=x^{3}+y^{3}+3xy$

Ответ: $(-1,-1)$ - локальный максимум

\item Найдите и классифицируйте экстремумы $f(x,y)=9x^{2}+9y^{2}+2xy$ при ограничении $x^{2}+y^{2}=1$.

Ответ: $(1/\sqrt{2},1/\sqrt{2})$ - максимум, $(-1/\sqrt{2},-1/\sqrt{2})$ - максимум, $(-1/\sqrt{2},1/\sqrt{2})$ - минимум, $(1/\sqrt{2},-1/\sqrt{2})$ - минимум

\item Решите уравнение $y'''-8y=0$

\item Решите систему: $x''=2y$ и $y''=-2x$.

Решение: $x(t)=e^{t}(C_{1}\sin(t)-C_{2}\cos(t))+e^{-t}(C_{4}\cos(t)-C_{3}\sin(t))$
$y(t)=e^{t}(C_{1}\cos(t)+C_{2}\sin(t))+e^{-t}(C_{3}\cos(t)+C_{4}\sin(t))$

\item Безработный индивид с вероятностью 20\% находит работу в течение ближайшего месяца (независимо от того, сколько времени он уже ищет работу). Индивид, имеющий работу, теряет ее в течение месяца с вероятностью 5\%. Известно, что на данный момент Петя является безработным.

Какова вероятность, что через два месяца Петя будет безработным?

Прошло два месяца, и Петя оказался безработным. Какова вероятность, что месяц назад он работал? (предполагается, что за месяц Петя может сделать только один переход между состояниями <<безработица>> и <<занятость>>).

Ответы: $0.65$, $1/65$

\item Контрольные камеры ДПС на МКАД зафиксировали скорость движения 6 автомобилей: 89, 83, 78, 96, 80, 78 км/ч. Предположим, что скорость распределена по нормальному закону.

Постройте 95\% доверительный интервал для средней скорости автомобилей, если истинная дисперсия равна 50 км/ч$^2$.

Постройте 80\% доверительный интервал для дисперсии скорости.

Ответ: $78.34<\mu<89.66$ и $27.94<\sigma^{2}<160.25$

\item ...
\item ...

\end{enumerate}

\section{2011}

\subsection{22 июля 2011, 1 вариант}
\begin{enumerate}
\item (10 баллов) Найдите и классифицируйте экстремумы функции $f(x,y,z)=2x-y+3z$ при ограничении $x^2+y^2+z^2=14$.\\
\item \begin{enumerate}
\item (4 балла) $n\times n$ матрица $A$ удовлетворяет соотношению $A^2-3A+I_n=0$. ($I_n$ --- единичная матрица). Может ли матрица $A$ быть вырожденной? Невырожденной?
\item (2 балла) $n\times n$ матрица $A$ удовлетворяет соотношению $A^2=0$. Следует ли отсюда, что $A=0? (n>1)$.
\item (4 балла) Множество многочленов степени 3, $M=\left\{f(x)=a_0+a_1x+a_2x^2+a_3x^3\right\}$ является линейным пространством относительно естественных операций сложения многочленов и умножения многочлена на число. Рассмотрим линейное преобразование $M\longrightarrow_A M$, такое, что $Af(x)=xf'(x)$. Найдите собственные числа и собственные векторы этого преобразования.
\end{enumerate}
\item Имеется матрица $A=\left(\begin{array}{cc}
4 & 2\\
2 & 4
\end{array}\right)$.
\begin{enumerate}
\item (3 балла) Найдите собственные числа и собственные векторы матрицы А.
\item (2 балла) Пусть $\vec{x}$ --- вектор-столбец подходящего размера, и $f(\vec{x})=\vec{x}^T A \vec{x}$. Какие значения может принимать функция $f(\vec{x})$ при произвольном векторе $\vec{x}$?
\item  (5 баллов) Обозначим через $||A||=[tr(A^TA)]^{1/2}$ норму матрицы $А$. ($А^T$ --- транспонированная матрица, $tr(B)$ --- след матрицы $B$.\\
Найдите $\lim_{n\to \infty}\frac{1}{||A^n||} A^n$.
\end{enumerate}
\item (10 баллов) Функция $y(x)$ на отрезке [0,2] удовлетворяет дифференциальному уравнению $y''+4y=0$, с граничными условиями: $y=0, y'=2 при x=0$. Найдите $y(2)$.\\
\item (10 баллов) Функция $y(t)$ удовлетворяет уравнению $y''(t)-8y'(t)-9y(t)+7=0$, а функция $x(t)$ равна $x(t)=\frac{1}{4}(y'(t)-y(t)-1)$. Найдите функцию  $y(t)$, такую, что $x(0)=x'(0)=0$.\\
\item Два стрелка стреляют по мишени (каждый делает один выстрел). Для первого стрелка вероятность промаха составляет 0.3, для второго --- 0.5. Результаты выстрелов независимы.
\begin{enumerate}
\item (5 баллов) Какова вероятность того, что мишень будет поражена хотя бы одним из них?
\item (5 баллов) При выстреле двух стрелков мишень была поражена (хотя бы одним выстрелом). Какова вероятность того, что второй стрелок промахнулся?
\end{enumerate}
\item Случайная величина X принимает значения в интервале [0,2], и на этом интервале ее функция распределения равна $F(x)=cx^3$, где c --- некоторая константа.
\begin{enumerate}
\item (2 балла) Найдите  $\P(X<0.3\mid X<0.6)$.
\item (3 балла) Найдите $\Cov\left(X+1,\frac{1}{X}\right)$.
\end{enumerate}
\item Имеется случайная выборка $X_1,...,X_n$, где все  $X_i$ независимы и принимают значения 1, 3 и 5 со следующими вероятностями:\\
\begin{tabular}{cccc}
\hline
$x$ & 1 & 3 & 5\\
$\P(X_i=x)$ & $a$ & 0.2 & 0.8-$a$\\
\hline
\end{tabular}
 \begin{enumerate}
 \item (5 баллов) Какие значения являются допустимыми для параметра $a$? Постройте оценку параметра $a$ методом моментов. Обязательно ли оценка принадлежит области допустимых значений параметра $a$?
 \item (5 баллов) При каком значении $m$ оценка $\hat{a}=mX_1-1.15+\frac{1}{n-1}\sum_{i=2}^n X_i$ параметра $a$ является несмещённой?
 \end{enumerate}
\item Страховая компания выплачивает агентам комиссию. План возмещения убытков предполагает, что средние выплаты комиссий составят 32 тысячи долларов в год. Если средние выплаты будут меньше запланированных, то план потребуется изменить. Для проверки гипотезы о том, что средние выплаты равны 32 тысячам долларов, против альтернативной гипотезы о том, что средние выплаты меньше 32 тысяч, была сформирована случайная выборка из 49 агентов. В этой выборке средние выплаты комиссий составили 29.5 тысяч долларов, а несмещённая оценка дисперсии оказалась равна 36. Для проверки гипотезы выбран уровень значимости 5\%.
\begin{enumerate}
\item (3 балла) Рассчитайте статистику, с помощью которой проверяется указанная гипотеза.
\item (2 балла) Рассчитайте критическое значение этой статистики.
\item (2 балла) Выясните, даёт ли выборочное исследование основание для пересмотра плана пересмотра убытков.
\item (3 балла) Определите, при каких уровнях значимости основная гипотеза будет отвергаться, а при каких --- нет.
\end{enumerate}
\item При 20 наблюдениях с помощью МНК оценивается регрессионное уравнение $y_i=\beta_1+\beta_2 x_i+\beta_3 z_i+\beta_4 t_i+\epsilon_i$ при условиях на ошибки, соответствующих стандартной модели множественной регрессии. Полученные вектор оценок коэффициентов и оценка его матрицы ковариаций равны:\\
$\hat{\beta}=\left[ \begin{array}{c}
8.4739\\
20.8209\\
1.2309\\
-17.4765
\end{array}\right],
\widehat{\Var}(\hat{\beta})=\left[\begin{array}{cccc}
54.94838 & -24.62334 & -30.31618 & -0.628223\\
-24.62334 & 85.97937 & 8.523841 & -72.60611\\
-30.31618 & 8.523841 & 19.45426 & 6.176577\\
-0.628223 & -72.60611 & 6.176577 & 77.56094
\end{array}\right]$\\, а оценка дисперсии ошибок регрессии и коэффициент детерминации равны $s^2=117.0376, R^2=0.243649$.\\ 
Оценивание на тех же данных уравнения $y_i=\gamma_1+\gamma_2 t_i+\epsilon_i$ дало значение коэффициента детерминации $R^2=0.003539$.
\begin{enumerate}
\item (5 баллов) На 5\% уровне значимости тестируйте гипотезу $H_0:\beta_2=0$ против альтернативы $H_a: \beta_2 \ne 0$, а также тестируйте гипотезу $H_0:\beta_3=0$ против альтернативы $H_a: \beta_3 \ne 0$
\item (5 баллов) На 5\%-ном уровне значимости тестируйте гипотезу $H_0:\beta_2=\beta_3=0$ против альтернативы $H_a$: <<не $H_0$>>.
\item (5 баллов) На 5\%-ном уровне значимости тестируйте гипотезу $H_0:\beta_2=\beta_3$ против альтернативы $H_a$: $\beta_2>\beta_3$.
\end{enumerate}

\end{enumerate}

\subsection{22 июля 2011, 2 вариант}
\begin{enumerate}
\item (10 баллов) Найдите и классифицируйте экстремумы функции $f(x,y,z)=x+2y+3z$ при ограничении $x^2+y^2+z^2=14$.\\
\item \begin{enumerate}
\item (4 балла) $n\times n$ матрица $A$ удовлетворяет соотношению $A^2+2A+I_n=0$. ($I_n$ --- единичная матрица). Может ли матрица $A$ быть вырожденной? Невырожденной?
\item (2 балла) $n\times n$ матрица $A$ удовлетворяет соотношению $A^2=A$. Следует ли отсюда, что  есть только две возможности: $A=0$ или $A=I_n? (n>1)$.
\item (4 балла) Множество многочленов степени 3, $M=\left\{f(x)=a_0+a_1x+a_2x^2+a_3x^3\right\}$ является линейным пространством относительно естественных операций сложения многочленов и умножения многочлена на число. Рассмотрим линейное преобразование $M\longrightarrow_A M$, такое, что $(Af)(x)=\frac{f(x)-f(0)}{x}$. Найдите собственные числа и собственные векторы этого преобразования.
\end{enumerate}
\item Имеется матрица $A=\left(\begin{array}{cc}
7 & 4\\
4 & 1
\end{array}\right)$.
\begin{enumerate}
\item (3 балла) Найдите собственные числа и собственные векторы матрицы А.
\item (2 балла) Пусть $\vec{x}$ --- вектор-столбец подходящего размера, и $f(\vec{x}=\vec{x}^T A \vec{x}$. Какие значения может принимать функция $f(\vec{x})$ при произвольном векторе $\vec{x}$?
\item  (5 баллов) Обозначим через $||A||=[tr(A^TA)]^{1/2}$ норму матрицы $А$. ($А^T$ --- транспонированная матрица, $tr(B)$ --- след матрицы $B$.\\
Найдите $\lim_{n\to \infty}\frac{1}{||A^n||} A^n$.
\end{enumerate}
\item (10 баллов) Функция $y(x)$ на отрезке [0,3] удовлетворяет дифференциальному уравнению $y''+3y'=0$, с граничными условиями: $y(0)=0, y'(3)=-3$. Найдите $y(3)$.\\
\item (10 баллов) Функция $y(t)$удовлетворяет уравнению $y''(t)-8y'(t)+12y(t)+8=0$, а функция $x(t)$ равна $x(t)=\frac{1}{2}(y'(t)-4y(t)-2)$. Найдите функцию  $y(t)$, такую, что $x(0)=x'(0)=0$.\\
\item На учениях два самолёта атакуют цель (каждый выпускает одну ракету). Известно, что первый самолёт поражает цель с вероятностью 0.6, а второй --- с вероятностью 0.4. Пусть самолёты поражают цель независимо друг от друга.
\begin{enumerate}
\item (5 баллов) Какова вероятность того, что цель будет поражена хотя бы одним самолётом?
\item (5 баллов) При разборе учений выяснилось, что цель была поражена только одним самолётом. Какова вероятность того, что цель поразил первый самолёт?
\end{enumerate}
\item Случайная величина X принимает значения в интервале [0,3], и на этом интервале ее функция распределения равна $F(x)=cx^2$, где c --- некоторая константа.
\begin{enumerate}
\item (2 балла) Найдите  $\P(X>2\mid X>1)$.
\item (3 балла) Найдите $\Cov\left(X^2+3,\frac{1}{X}\right)$.
\end{enumerate}
\item Имеется случайная выборка $X_1,...,X_n$, где все  $X_i$ независимы и принимают значения -1, 1 и 4 со следующими вероятностями:\\
\begin{tabular}{cccc}
\hline
$x$ & -1 & 1 & 4\\
$\P(X_i=x)$ & 0.3 & $a$ & 0.7-$a$\\
\hline
\end{tabular}
 \begin{enumerate}
 \item (5 баллов) Какие значения являются допустимыми для параметра $a$? Постройте оценку параметра $a$ методом моментов. Обязательно ли оценка принадлежит области допустимых значений параметра $a$?
 \item (5 баллов) При каком значении $m$ оценка $\hat{a}=X_1-\frac{5}{6}+\frac{m}{n-1}\sum_{i=2}^n X_i$ параметра $a$ является несмещённой?
 \end{enumerate}
\item Фирма-производитель некоторого лекарственного препарат следит за тем, чтобы концентрация посторонних примесей в препарате в среднем составляла не более 0.03. Для проверки гипотезы о том, что концентрация посторонних примесей равна 0.03, против альтернативной гипотезы о том, что эта концентрация выше 0.03, была взята случайная выборка из 64 образцов препарата. Средняя концентрация примесей в выборке составила 0.0327, а несмещённая оценка дисперсии составила 0.0009. Для проверки гипотезы выбран уровень значимости 10\%.
\begin{enumerate}
\item (3 балла) Рассчитайте статистику, с помощью которой проверяется указанная гипотеза.
\item (2 балла) Рассчитайте критическое значение этой статистики.
\item (2 балла) Выясните, даёт ли выборочное исследование основание считать, что средняя концентрация посторонних примесей превышает допустимый предел в 0.03?
\item (3 балла) Определите, при каких уровнях значимости основная гипотеза будет отвергаться, а при каких --- нет.
\end{enumerate}
\item При 20 наблюдениях с помощью МНК оценивается регрессионное уравнение $y_i=\beta_1+\beta_2 x_i+\beta_3 z_i+\beta_4 t_i+\epsilon_i$ при условиях на ошибки, соответствующих стандартной модели множественной регрессии. Полученные вектор оценок коэффициентов и оценка его матрицы ковариаций равны:\\
$\hat{\beta}=\left[ \begin{array}{c}
9.979620\\
-0.493709\\
0.281451\\
2.955317
\end{array}\right],
\widehat{\Var}(\hat{\beta})=\left[\begin{array}{cccc}
0.264234 & 0.009178 & -0.116760 & -0.264416\\
0.009178 & 0.077753 & 0.014406 & -0.114635\\
-0.116760 & 0.014406 & 0.061492 & 0.098570\\
-0.264416 & -0.116760 & 0.098570 & 0.436001
\end{array}\right]$\\, а оценка дисперсии ошибок регрессии и коэффициент детерминации равны $s^2=0.49367627, R^2=0.832389$.\\ 
Оценивание на тех же данных уравнения $y_i=\gamma_1+\gamma_2 t_i+\epsilon_i$ дало значение коэффициента детерминации $R^2=0.786790$.
\begin{enumerate}
\item (5 баллов) На 5\% уровне значимости тестируйте гипотезу $H_0:\beta_2=0$ против альтернативы $H_a: \beta_2 \ne 0$, а также тестируйте гипотезу $H_0:\beta_3=0$ против альтернативы $H_a: \beta_3 \ne 0$
\item (5 баллов) На 5\%-ном уровне значимости тестируйте гипотезу $H_0:\beta_2=\beta_3=0$ против альтернативы $H_a$: <<не $H_0$>>.
\item (5 баллов) На 5\%-ном уровне значимости тестируйте гипотезу $H_0:\beta_2=\beta_3$ против альтернативы $H_a: \beta_2>\beta_3$.
\end{enumerate}
\end{enumerate}

\subsection{Решение варианта 1, 22 июля 2011}
\begin{enumerate}
\item Выписываем функцию Лагранжа: $L=2x-y+3z-\lambda (x^2+y^2+z^2-14)$.\\\\
Условия первого порядка:
$\left\{\begin{aligned}
-2\lambda x+2&=0\\
-2\lambda y-1&=0\\
-2\lambda z+3&=0\\
-x^2-y^2-z^2+14&=0\\
\end{aligned}\right.$\\
Решения системы (критические точки): \\
Точка А: $\left[ x=2,y=-1,z=3,\lambda=1/2\right], f(A)=14$\\
Точка В: $\left[ x=-2,y=1,z=-3,\lambda=-1/2\right], f(B)=-14$\\
Из геометрических соображений очевидно, что одна из точек есть минимум, а другая --- максимум. (График функции есть гиперплоскость, которая ограничивается на сферу).\\
На всякий случай, окаймленная матрица Гессе:
$\left(\begin{array}{cccc}
0 & -2x & -2y & -2z\\
-2x & -2\lambda & 0 & 0\\
-2y & 0 & -2\lambda & 0\\
-2z & 0 & 0 & -2\lambda
\end{array}\right)$\\
На всякий случай, окаймленная матрица Гессе в А: 
$\left(\begin{array}{cccc}
0 & -4 & 2 & -6\\
-4 & -1 & 0 & 0\\
2 & 0 & -1 & 0\\
-6 & 0 & 0 & -1\lambda
\end{array}\right)$.\\
Миноры: $\bigtriangleup_4=-56<0, \bigtriangleup_3=20>0$, максимум.\\
На всякий случай, окаймленная матрица Гессе в В: 
$\left(\begin{array}{cccc}
0 & 4 & -2 & 6\\
4 & 1 & 0 & 0\\
-2 & 0 & 1 & 0\\
6 & 0 & 0 & 1\lambda
\end{array}\right)$.\\
Миноры: $\bigtriangleup_4=-56<0, \bigtriangleup_3=-20<0$, минимум.\\
Баллы:\\
5 баллов за найденные точки\\
5 баллов за обоснование того, что они есть максимум и минимум (с гессианами или без)
\item \begin{enumerate}
\item $I_n=A(3I_n-A)=AB$, т.е. существует обратная матрица $B$, т.е. матрица $A$ невырожденная.
\item Для $n=1$ следует, т.к. из $a^2=0$ следует $a=0$. Для $n>1$ это не верно, например: $A=\left[\begin{array}{cc}
0 & 1\\
0 & 0
\end{array}\right],A^2=
\left[\begin{array}{cc}
0 & 0\\
0 & 0
\end{array}\right]=0.$
\item Рассмотрим базис $\{1,x,x^2,x^3\}$, в этом базисе матрица оператора имеет вид: 
$A=\left[\begin{array}{cccc}
0 & 0 & 0 &0\\
0 & 1 & 0 & 0\\
0 & 0 & 2 & 0\\
0 & 0 & 0 & 3
\end{array}\right]$, т.е. собственные числа 0,1,2,3 и соответствующие собственные векторы $1,x,x^2,x^3$
\end{enumerate}
\item \begin{enumerate}
\item Характеристическое уравнение $(4-\lambda)(4-\lambda)-4=0$, собственные числа: 6,2.\\ Нормированные собственные векторы: $\frac{1}{\sqrt{2}}\left[\begin{array}{c}
1\\
1
\end{array}\right]$ для $\lambda=6$ и $\frac{1}{\sqrt{2}}\left[\begin{array}{c}
1\\
-1
\end{array}\right]$ для $\lambda=2$
\item Собственные числа положительные. Значит, квадратичная форма принимает неотрицательные значения.
\item Матрица А представима в виде $A=CDC^{-1}$, где $C=\frac{1}{\sqrt{2}}\left[\begin{array}{cc}
1 & 1\\
1 & -1
\end{array}\right]$ --- ортогональная матрица, а 
 $D=\left[\begin{array}{cc}
6 & 0\\
0 & 2
\end{array}\right]$ --- диагональная.\\ Значит, $A^n=CD^n C^{-1}$, $D^n=\left[\begin{array}{cc}
6^n & 0\\
0 & 2^n
\end{array}\right]$.
\begin{multline}
{||A^n||}^2=[tr((A^n)^T A^n)]=tr((CD^n C^T)^T CD^n C^T)=tr(CD^n C^T CD^n C^T)=\\
=tr(CD^{2n} C^T)=tr(D^2n C^T C)=tr(D^{2n})=6^{2n}+2^{2n}
\end{multline}

\begin{multline}
\lim_{n\to \infty}\frac{1}{||A^n||}A^n=C\cdot\left(\lim_{n\to \infty}\frac{1}{||A^n||}D^n\right)\cdot C^T=\\
= C\cdot\left(\lim_{n\to \infty}\frac{1}{(6^{2n}+2^{2n})^{1/2}}\right)\left[\begin{array}{cc}
6^n & 0\\
0 & 2^n
\end{array}\right]\cdot C^T=C\cdot\left(\begin{array}{cc}
1 & 0\\
0 & 0
\end{array}\right) \cdot C^T =\\
= \frac{1}{\sqrt{2}}\left[\begin{array}{cc}
1 & 1\\
1 & -1
\end{array}\right]\left(\begin{array}{cc}
1 & 0\\
0 & 0
\end{array}\right)\frac{1}{\sqrt{2}}\left[\begin{array}{cc}
1 & 1\\
1 & -1
\end{array}\right]=\frac{1}{2}\left(\begin{array}{cc}
1 & 1\\
1 & 1
\end{array}\right)
\end{multline}

\end{enumerate}
\item $\lambda^2+4=0 \Rightarrow \lambda_1=2i, \lambda_2=-2i.$ \\
Общее решение дифференциального уравнения имеет вид: $y(x)=C_1\cdot \cos(2x)+C_2\cdot \sin(2x), \forall C_1,C_2 \in \R$.\\
Учитывая $y=0, y'=2$ при $x=0$ получаем $C_1=0, C_2=1$. Функция равна $y(x)=\sin(2x)$, соответственно, $y(2)=\sin(4)\approx 0.757$\\
Баллы:\\
5 баллов за найденное общее решение\\
5 баллов за частное решение и верный ответ.
\item $y(t)=7/9$ является частным решением неоднородного уравнения $y''(t)-8y'(t)-9y(t)+7=0$. Найдем общее решение однородного уравнения $y''(t)-8y'(t)-9y(t)=0$.\\
Характеристическое уравнение $\lambda^2-8\lambda-9=0$ имеет корни $\lambda_1=-1, \lambda_2=9$. Тогда общее решение неоднородного уравнения имеет вид $y(t)=c_1e^{-t}+c_2e^{9t}+\frac{7}{9}$, соответственно,
\begin{multline}
x(t)=\frac{1}{4}(y'(t)-y(t)-1)=\frac{1}{4}\left((c_1e^{-t}+c_2e^{9t}+\frac{7}{9})'-(c_1e^{-t}+c_2e^{9t}+\frac{7}{9})-1\right)= \\
= \frac{1}{4}\left(-c_1e^{-t}+9c_2e^{9t}-c_1e^{-t}-c_2e^{9t}-\frac{7}{9}-1\right)=\frac{1}{4}\left(-2c_1e^{-t}+8c_2e^{9t}-\frac{16}{9}\right)=\\
=\frac{1}{2}\left(-c_1e^{-t}+4c_2e^{9t}-\frac{8}{9}\right)
\end{multline}
Из граничных условий получаем: 
\[\left\{\begin{aligned}
2x(0) &=-c_1+4c_2-\frac{8}{9}&=0\\
2x'(0)&=c_1+36c_2 &=0
\end{aligned}\right.\]
Отсюда получаем 
\begin{equation}
c_1=-\frac{4}{5}, c_2=\frac{1}{45}, y(t)=-\frac{4}{5}e^{-t}+\frac{1}{45}e^{9t}+\frac{7}{9}
\end{equation}
Баллы:\\
5 баллов за найденное общее решение неоднородного уравнения\\
5 баллов за частное решение и верный ответ
\item \begin{enumerate}
\item Обозначим события: $A$ --- <<первый стрелок промахнулся>>, $B$ --- <<второй стрелок промахнулся>>. Тогда событие <<мишень поражена>> можно записать как $\bar{C}=A\cap B$. Искомая вероятность: $\P(C)=1-\P(A\cap B)=1-\P(A)\cdot \P(B)=1-0.3\cdot 0.5=0.85$.
\item Здесь нужно найти условную вероятность события $B$ при условии $C$. По определению условной вероятности, $\P(B\mid C)=\frac{\P(B\cap C)}{\P(C)}$. Совместное наступление событий $B$ и $C$ (второй стрелок промахнулся, но мишень была поражена) эквивалентно тому, что первый стрелок поразил мишень, а второй промахнулся, т.е. $B\cap C=\bar{A}\cap B$. Таким образом, \begin{equation}
P(B\mid C)=\frac{\P(\bar{A}\cap B}{\P(C)}=\frac{\P(\bar{A)\P(B)}}{\P(C)}=\frac{(1-0.3)\cdot 0.5}{0.85}=\frac{0.35}{0.85}\approx 0.4118.
\end{equation}
\end{enumerate}
\item Сначала найдем $c$: $1=F(2)=c\cdot 2^3$, получаем $c=1/8$.
\begin{enumerate}
\item 
\begin{equation}
\P(X<0.3\mid X<0.6)=\frac{\P(X<0.3\cap X<0.6}{\P(C<0.6)}=\frac{\P(X<0.3)}{\P(X<0.6)}=\frac{F(0.3)}{F(0.6)}={\frac{0.3}{0.6}}^3=\frac{1}{8}=0.125
\end{equation}
\item 
\begin{align}
&f(x)=F'(x)=\frac{3}{8}x^2.\\
&EX=\int_0^2 x\frac{3}{8}x^2 dx=\frac{3}{8}\cdot\frac{2^4}{4}=\frac{3}{2}=1.5\\
&\E(\frac{1}{X})=\int_0^2 x^{-1}\frac{3}{8}x^2 dx=\frac{3}{8}\cdot\frac{2^2}{2}=\frac{3}{4}=0.75.\\
&\Cov(X+1,\frac{1}{X})=\Cov(X,\frac{1}{X})=\E(X\cdot \frac{1}{X})-(EX)\E(\frac{1}{X})=1-\frac{3}{2}\cdot \frac{3}{4}=1-\frac{9}{8}=-\frac{1}{8}=-0.125.
\end{align}
\end{enumerate}
\item \begin{enumerate}
\item Допустимое множество значений параметра $a: [0,0.8]$.\\
Найдем математическое ожидание величин $X_i: \E(X_i)=1\cdot a+3\cdot 0.2+5(0.8-a)=4.6-4a$.\\
Приравняем его к выборочному среднему: $4.6-4a=\bar{X}$.\\
Решив полученное уравнение относительно $a$, получаем оценку метода моментов:
\begin{equation}
\hat{a}_{MM}=\frac{4.6-\bar{X}}{4}=1.15-\frac{\bar{X}}{4}
\end{equation}
Эта оценка не может не принадлежать области допустимых значений параметра $a$.
\item Найдём математическое ожидание предложенной оценки: 
\begin{equation}
\E(\hat{a}=m\E(X_1)-1.15+\frac{1}{n-1}\sum_{i=2}^n \E(X_i)=\\=4.6m-4ma-1.15+\frac{1}{n-1}(n-1)(4.6-4a)=(4.6-4a)(m+1)-1.15.
\end{equation}
Оценка $\hat{a}$ будет несмещенной, если $\E(\hat{a})=a$, или $(4.6-4a)(m+1)-1.15=a$. Решив уравнение относительно $m$, получаем \begin{equation}
m=\frac{5a-3.45}{4.6-4a}
\end{equation}
Поскольку $m$, а, следовательно, и $\hat{a}$, зависит от неизвестного параметра, то такого значения $m$ не существует.
\end{enumerate}
\item \begin{enumerate}
\item Тестируем нулевую гипотезу $H_0: \mu_=\mu_0$ против альтернативы $H_A:\mu<\mu_0$ в случае произвольной генеральной совокупности и большого объёма выборки. Для решения этой задачи воспользуемся статистикой $z=\frac{\bar{X}-\mu_0}{s/\sqrt{n}}\sim_{H_0} N(0,1)$. (Можно также предположить, что генеральная совокупность нормальна, и использовать распределение Стьюдента.)\\
\begin{equation}
z=\frac{29.5-32}{\sqrt{36}/\sqrt{49}}=-\frac{5/2}{6/7}=-\frac{35}{12}=-2.917.
\end{equation}
\item Если пользоваться нормальным распределением, то критическое значение $z_{crit}=-z_{0.05}=-1.645$ (Для распределения Стьюдента $t_{crit}=-t_{n-1,\alpha}=-t_{48,0.05}=-1.677.$ Нужного числа степеней свободы в таблице нет, но можно установить, что $t_{crit} \in (-1.684,-1.671).$
\item Так как $z<z_{crit} (z<t_{crit})$, то нулевая гипотеза отвергается, план возмещения убытков стоит пересмотреть.
\item При использовании нормального распределения P-значение=$\P(Z<-2.917)=0.0018$. Таким образом, при уровне значимости выше 0.18\% основная гипотеза будет отвергаться, а при уровне ниже 0.18\% --- не будет. Если пользоваться распределением Стьюдента, то из таблиц можно установить, что P-значение $\in (0.001,0.005)$.
\end{enumerate}
\item \begin{enumerate}

\item 
\begin{equation}
t_{\hat{\beta_2}}=\frac{\hat{\beta}_2}{s_{\hat{\beta}_2}}=\frac{20.8209}{\sqrt{85.97937}}=2.245; t_{\hat{\beta_3}}=\frac{\hat{\beta}_3}{s_{\hat{\beta}_3}}=\frac{1.2309}{\sqrt{19.45426}}=0.279
\end{equation}
поскольку $|t_{\hat{\beta}_2}|>t_{0.25}(16)=2.12$, то гипотеза $H_0: \beta_2=0$ отвергается, соответственно, гипотеза $H_0:\beta_3=0$ не отвергается.
\item 
\begin{equation}
F=\frac{(R_{UR}^2-R_R^2)/q}{(1-R_{UR}^2)/(n-k)}=\frac{(0.243649-0.0035359)/2}{(1-0.243649)/16}=2.54<F_{0.05}(2.16)=3.63
\end{equation}
т.е. гипотеза $\beta_2=\beta_3=0$ не отвергается.
\item Рассмотрим разность $\hat{\beta}_2-\hat{\beta}_3$, оценка её дисперсии равна 
\begin{equation}
\hat{V}(\hat{\beta}_2-\hat{\beta}_3)=\hat{V}(\hat{\beta}_2)+\hat{V}(\hat{\beta}_3)-2\hat{\Cov}(\hat{\beta}_2,\hat{\beta}_3)=85.97937+19.45424-2\cdot8.523841=88.38595
\end{equation}
критическая статистика $t=\frac{\hat{\beta}_2-\hat{\beta}_3}{\sqrt{\hat{V}(\hat{\beta}_2-\hat{\beta}_3)}}$ при нулевой гипотезе имеет распределение $t(16)$. Поскольку $t_{0.05}(16)=1.746$, а $t=2.08>1.746$, то гипотеза $H_0: \beta_2=\beta_3$ отвергается в пользу альтернативы $H_1:\beta_2>\beta_3$
\end{enumerate}
\end{enumerate}

\subsection{Решение варианта 2, 22 июля 2011}
\begin{enumerate}
\item Выписываем функцию Лагранжа: $L=x+2y+3z-\lambda (x^2+y^2+z^2-14)$.\\\\
Условия первого порядка:
$\left\{\begin{aligned}
-2\lambda x+1&=0\\
-2\lambda y+2&=0\\
-2\lambda z+3&=0\\
-x^2-y^2-z^2+14&=0\\
\end{aligned}\right.$\\
Решения системы (критические точки): \\
Точка А: $\left[ x=1,y=2,z=3,\lambda=1/2\right], f(A)=14$\\
Точка В: $\left[ x=-1,y=-2,z=-3,\lambda=-1/2\right], f(B)=-14$\\
Из геометрических соображений очевидно, что одна из точек есть минимум, а другая --- максимум. (График функции есть гиперплоскость, которая ограничивается на сферу).\\
На всякий случай, окаймленная матрица Гессе:
$\left(\begin{array}{cccc}
0 & -2x & -2y & -2z\\
-2x & -2\lambda & 0 & 0\\
-2y & 0 & -2\lambda & 0\\
-2z & 0 & 0 & -2\lambda
\end{array}\right)$\\
На всякий случай, окаймленная матрица Гессе в А: 
$\left(\begin{array}{cccc}
0 & -2 & -4 & -6\\
-2 & -1 & 0 & 0\\
-4 & 0 & -1 & 0\\
-6 & 0 & 0 & -1\lambda
\end{array}\right)$.\\
Миноры: $\bigtriangleup_4=-56<0, \bigtriangleup_3=20>0$, максимум.\\
На всякий случай, окаймленная матрица Гессе в В: 
$\left(\begin{array}{cccc}
0 & 2 & 4 & 6\\
2 & 1 & 0 & 0\\
4 & 0 & 1 & 0\\
6 & 0 & 0 & 1\lambda
\end{array}\right)$.\\
Миноры: $\bigtriangleup_4=-56<0, \bigtriangleup_3=-20<0$, минимум.\\
Баллы:\\
5 баллов за найденные точки\\
5 баллов за обоснование того, что они есть максимум и минимум (с гессианами или без)
\item \begin{enumerate}
\item $I_n=A(-2I_n-A)=AB$, т.е. существует обратная матрица $B$, т.е. матрица $A$ невырожденная.
\item Для $n=1$ следует, т.к. из $a^2=a$ следует $a_1=0,a_2=1$. Для $n>1$ это не верно, например: $A=\left[\begin{array}{cc}
0 & 0\\
0 & 1
\end{array}\right],A^2=
\left[\begin{array}{cc}
0 & 0\\
0 & 1
\end{array}\right]$
\item Рассмотрим базис $\{1,x,x^2,x^3\}$, в этом базисе матрица оператора имеет вид: 
$A=\left[\begin{array}{cccc}
0 & 1 & 0 &0\\
0 & 0 & 1 & 0\\
0 & 0 & 0 & 1\\
0 & 0 & 0 & 0
\end{array}\right]$, т.е. собственные числа 0. Cобственные векторы находятся из условия \begin{equation}
Af)(x)=\frac{a_0+a_1 x+a_2 x^2+a_3 x^3-a_0}{x}=a_1+a_2 x+a_3 x^2=0
\end{equation} откуда $a_1=a_2=a_3=0$, т.е. имеется единственный (с точностью до множителя) собственный векторов $f(x)=1$.
\end{enumerate}
\item \begin{enumerate}
\item Характеристическое уравнение $(7-\lambda)(1-\lambda)-16=0$, собственные числа: 9,-1.\\ Cобственные векторы: $\left[\begin{array}{c}
2\\
1
\end{array}\right]$ для $\lambda=9$ и $\left[\begin{array}{c}
1\\
-2
\end{array}\right]$ для $\lambda=-1$
\item Собственные числа разного знака. Значит, квадратичная форма принимает любые значения.
\item Матрица А представима в виде $A=CDC^{-1}$, где $C=\frac{1}{\sqrt{5}}\left[\begin{array}{cc}
2 & 1\\
1 & -2
\end{array}\right]$ --- ортогональная матрица, а 
 $D=\left[\begin{array}{cc}
9 & 0\\
0 & -1
\end{array}\right]$ --- диагональная.\\ Значит, $A^n=CD^n C^{-1}$, $D^n=\left[\begin{array}{cc}
9^n & 0\\
0 & (-1)^n
\end{array}\right]$.\\
\begin{align}
&{||A^n||}^2=[tr((A^n)^T A^n)]=tr((CD^n C^T)^T CD^n C^T)=tr(CD^n C^T CD^n C^T)=tr(CD^{2n} C^T)=tr(D^2n C^T C)=tr(D^{2n})=9^{2n}+1.\\
&\lim_{n\to \infty}\frac{1}{||A^n||}A^n=C\cdot\left(\lim_{n\to \infty}\frac{1}{||A^n||}D^n\right)\cdot C^T=C\cdot\left(lim_{n\to \infty}\frac{1}{(9^{2n}+1)^{1/2}}\right)\left[\begin{array}{cc}
9^n & 0\\
0 & (-1)^n
\end{array}\right]\cdot C^T=C\cdot\left(\begin{array}{cc}
1 & 0\\
0 & 0
\end{array}\right) \cdot C^T = \frac{1}{\sqrt{5}}\left[\begin{array}{cc}
2 & 1\\
1 & -2
\end{array}\right]\left(\begin{array}{cc}
1 & 0\\
0 & 0
\end{array}\right)\frac{1}{\sqrt{5}}\left[\begin{array}{cc}
2 & 1\\
1 & -2
\end{array}\right]=\frac{1}{5}\left(\begin{array}{cc}
4 & 2\\
2 & 1
\end{array}\right)
\end{align}
\end{enumerate}
\item $\lambda^2+3\lambda=0 \Rightarrow \lambda_1=0, \lambda_2=-3.$ \\
Общее решение дифференциального уравнения имеет вид: $y(x)=C_1+C_2\cdot e^{-3x}, \forall C_1,C_2 \in \R$.\\
$0=y(0)=C_1+C_2\cdot e^{-3\cdot 0} \Rightarrow C_1+C_2=0\\
-3=y'(3)=-3C_2\cdot e^{-3\cdot 3}\Rightarrow C_2=e^9$.\\
 Функция равна $y(x)=e^9(-1+e^{-3x})$, соответственно, $y(3)=e^9(-1+e^{3\cdot3})=1-e^9\approx -8102$\\
Баллы:\\
5 баллов за найденное общее решение\\
5 баллов за частное решение и верный ответ.
\item $y(t)=-8/12=-2/3$ является частным решением неоднородного уравнения $y''(t)-8y'(t)+12y(t)+8=0$. Найдем общее решение однородного уравнения $y''(t)-8y'(t)+12y(t)=0$.\\
Характеристическое уравнение $\lambda^2-8\lambda+12=0$ имеет корни $\lambda_1=2, \lambda_2=6$. Тогда общее решение неоднородного уравнения имеет вид $y(t)=c_1e^{2t}+c_2e^{6t}-\frac{2}{3}$, соответственно,
\begin{equation}
x(t)=\frac{1}{2}\left((c_1e^{2t}+c_2e^{6t}-\frac{2}{3})'-4(c_1e^{2t}+c_2e^{6t}-\frac{2}{3})-2\right)=\frac{1}{2}(2c_1e^{2t}+6c_2e^{6t}-4c_1e^{2t}-4c_2e^{6t}+\frac{8}{3}-2)=-c_1e^{2t}+c_2e^{6t}+\frac{1}{3})
\end{equation}
Из граничных условий получаем: $\left\{\begin{aligned}
x(0) &=-c_1+c_2+\frac{1}{3}&=0\\
x'(0)&=-2c_1+6c_2 &=0
\end{aligned}\right.$\\\\
Отсюда получаем 
\begin{equation}
c_1=\frac{1}{2}, c_2=\frac{1}{6}, y(t)=\frac{1}{2}e^{2t}+\frac{1}{6}e^{6t}-\frac{2}{3}
\end{equation}
Баллы:\\
5 баллов за найденное общее решение неоднородного уравнения\\
5 баллов за частное решение и верный ответ
\item \begin{enumerate}
\item Обозначим события: $A$ --- <<первый самолёт поразил цель>>, $B$ --- <<второй самолёт поразил цель>>. Тогда событие <<цель поражена только одним самолётом>> можно записать как $C=(\bar{A}\cap B)\cup (A\cap \bar{B})$. В силу независимости событий $A$ и $B$ искомая вероятность равна:
\begin{equation}
\P(C)=\P(\bar{A})\P(B)+\P(A)\P(\bar{B})=(1-0.6)\cdot 0.4+0.6\cdot (1-0.4)=0.16+0.36=0.52
\end{equation}
\item Здесь нужно найти условную вероятность события $A$ при условии $C$. По определению условной вероятности, $\P(A\mid C)=\frac{\P(A\cap C)}{\P(C)}$. Совместное наступление событий $A$ и $C$ (первый самоёт поразил цель, и цель была поражена только одним самолётом) эквивалентно тому, что первый самолёт поразил цель, а второй --- нет, т.е. $A\cap C=A\cap \bar{B}$. Таким образом,
\begin{equation}
\P(A\mid C)=\frac{\P(A)\cap \bar{B}}{\P(C)}=\frac{\P(A)(1-\P(B))}{\P(C)}=\frac{0.6\cdot 0.6}{0.52}=\frac{0.36}{0.52}\approx 0.6923.
\end{equation}
\end{enumerate}
\item Сначала найдем $c$: $1=F(3)=c\cdot 3^2$, получаем $c=1/9$.
\begin{enumerate}
\item 
\begin{equation}
\P(X>2\mid X>1)=\frac{\P(X>2\cap X>1}{\P(X>1)}=\frac{\P(X>2)}{\P(X>1)}=\frac{1-F(2)}{1-F(1)}=\frac{1-\frac{1}{9}\cdot 4}{1-\frac{1}{9}}=\frac{5}{8}=0.625
\end{equation}
\item 
\begin{align}
& f(x)=F'(x)=\frac{2}{9}x.\\
& EX=\int_0^3 x^2\frac{2}{9}x dx=\frac{2}{9}\cdot\frac{3^4}{4}=\frac{9}{2}=4.5\\
& \E(\frac{1}{X})=\int_0^3 x^{-1}\frac{2}{9}x dx=3\frac{2}{9}=\frac{2}{3}\approx 0.667.\\
& \Cov(X^2+3,\frac{1}{X})=\Cov(X^2,\frac{1}{X})=\E(X^2\cdot \frac{1}{X})-(E(X^2))\E(\frac{1}{X})=2-\frac{9}{2}\cdot \frac{2}{3}=-1.
\end{align}

\end{enumerate}
\item \begin{enumerate}
\item Допустимое множество значений параметра $a: [0,0.7]$.\\
Найдем математическое ожидание величин $X_i: \E(X_i)=1\cdot 0.3+1\cdot a+4(0.7-a)=2.5-3a$.\\
Приравняем его к выборочному среднему: $2.5-3a=\bar{X}$.\\
Решив полученное уравнение относительно $a$, получаем оценку метода моментов:
$\hat{a}_{MM}=\frac{2.5-\bar{X}}{3}$. Эта оценка не может не принадлежать области допустимых значений параметра $a$.
\item Найдём математическое ожидание предложенной оценки:
\begin{equation}
\E(\hat{a}=\E(X_1)-\frac{5}{6}+\frac{m}{n-1}\sum_{i=2}^n \E(X_i)=2.5-3a-\frac{5}{6}+\frac{m}{n-1}(n-1)(2.5-3a)=(2.5-3a)(m+1)-\frac{5}{6}=a
\end{equation}
Оценка $\hat{a}$ будет несмещенной, если $\E(\hat{a})=a$, или $(2.5-3a)(m+1)-\frac{5}{6}=a$. Решив уравнение относительно $m$, получаем 
\begin{equation}
m=\frac{5a-3.45}{4.6-4a}
\end{equation}
Поскольку $m$, а, следовательно, и $\hat{a}$, зависит от неизвестного параметра, то такого значения $m$ не существует.
\end{enumerate}
\item \begin{enumerate}
\item Тестируем нулевую гипотезу $H_0: \mu_=\mu_0$ против альтернативы $H_A:\mu<\mu_0$ в случае произвольной генеральной совокупности и большого объёма выборки. Для решения этой задачи воспользуемся статистикой $z=\frac{\bar{X}-\mu_0}{s/\sqrt{n}}\sim_{H_0}N(0,1)$. (Можно также предположить, что генеральная совокупность нормальна, и использовать распределение Стьюдента.)\begin{equation}
z=\frac{0.0327-0.03}{\sqrt{0.0009}/\sqrt{64}}=-\frac{0.0027}{0.03/8}=0.72.
\end{equation}
\item Если пользоваться нормальным распределением, то критическое значение $z_{crit}=-z_{0.1}=1.28$ (Для распределения Стьюдента $t_{crit}=-t_{n-1,\alpha}=-t_{63,0.1}=1.295.$ Нужного числа степеней свободы в таблице нет, но можно установить, что $t_{crit} \in (1.289,1.296).$
\item Так как $z<z_{crit} (z<t_{crit})$, то нет оснований отвергать основную гипотезу и считать, что допустимый предел концентрации превышен.
\item При использовании нормального распределения P-значение=$\P(Z>0.72)=0.2358$. Таким образом, при уровне значимости выше 23.58\% основная гипотеза будет отвергаться, а при уровне ниже 23.58\% --- не будет. Если пользоваться распределением Стьюдента, то из таблиц можно установить, что P-значение $\in (0.2,0.25)$.
\end{enumerate}
\item \begin{enumerate}

\item \begin{equation}
t_{\hat{\beta_2}}=\frac{\hat{\beta}_2}{s_{\hat{\beta}_2}}=\frac{-0.49371}{\sqrt{0ю077753}}=-1.77; t_{\hat{\beta_3}}=\frac{\hat{\beta}_3}{s_{\hat{\beta}_3}}=\frac{0.281451}{\sqrt{0.061492}}=1.13
\end{equation}
поскольку $|t_{\hat{\beta}_2}|<t_{0.25}(16)=2.12$, то гипотеза $H_0: \beta_2=0$ не отвергается, соответственно, гипотеза $H_0:\beta_3=0$ не отвергается.
\item \begin{equation}
F=\frac{(R_{UR}^2-R_R^2)/q}{(1-R_{UR}^2)/(n-k)}=\frac{(0.832389-0.786790)/2}{(1-0.832389)/16}=2.18<F_{0.05}(2.16)=3.63
\end{equation} т.е. гипотеза $\beta_2=\beta_3=0$ не отвергается.
\item Рассмотрим разность $\hat{\beta}_2-\hat{\beta}_3$, оценка её дисперсии равна \begin{equation}
{V}(\hat{\beta}_2-\hat{\beta}_3)=\hat{V}(\hat{\beta}_2)+\hat{V}(\hat{\beta}_3)-2\hat{\Cov}(\hat{\beta}_2,\hat{\beta}_3)=0.077753+0.061492-2\cdot 0.014406=0.110433
\end{equation} критическая статистика $t=\frac{\hat{\beta}_2-\hat{\beta}_3}{\sqrt{\hat{V}(\hat{\beta}_2-\hat{\beta}_3)}}$ при нулевой гипотезе имеет распределение $t(16)$. Поскольку $t_{0.05}(16)=1.746$, а $t=-2.33<-1.746$, то гипотеза $H_0: \beta_2=\beta_3$ отвергается в пользу альтернативы $H_1:\beta_2<\beta_3$
\end{enumerate}
\end{enumerate}


\end{document}