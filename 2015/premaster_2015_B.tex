\documentclass[addpoints, answers]{exam} % добавить или удалить answers в скобках, чтобы показать ответы
\usepackage[utf8]{inputenc}
\usepackage[russian]{babel}
\usepackage{booktabs}
%\usepackage[OT1]{fontenc}
\usepackage{amsmath}
\usepackage{amsfonts}
\usepackage{amssymb}
\usepackage{tikz}
\usepackage[left=2cm,right=2cm,top=2cm,bottom=2cm]{geometry}
\DeclareMathOperator{\E}{\mathbb{E}}
\DeclareMathOperator{\Var}{\mathbb{V}\mathrm{ar}}
\DeclareMathOperator{\Cov}{\mathbb{C}\mathrm{ov}}
\let\P\relax
\DeclareMathOperator{\P}{\mathbb{P}}
\newcommand{\cN}{\mathcal{N}}
\newcommand{\hbeta}{\hat{\beta}}

\usepackage{floatrow}
\newfloatcommand{capbtabbox}{table}[][\FBwidth]

\begin{document}

\pagestyle{headandfoot}
\runningheadrule
\firstpageheader{НИУ-ВШЭ}{Высшая математика, стр. \thepage\ из \numpages}{17 июля 2015}
\firstpageheadrule
\runningheader{НИУ-ВШЭ}{Высшая математика, стр. \thepage\ из \numpages}{17 июля 2015}
\firstpagefooter{}{}{}
\runningfooter{}{}{}
\runningfootrule




\hqword{Задача}
\hpgword{Страница}
\hpword{Максимум}
\hsword{Баллы}
\htword{Итого}
\pointname{\%}
%\renewcommand{\solutiontitle}{\noindent\textbf{Решение:}\par\noindent}
\renewcommand{\solutiontitle}{}

%Таблица с результатами заполняется проверяющим работу. Пожалуйста, не делайте в ней пометок.

%\begin{center}
%  \gradetable[h][questions]
%\end{center}


\begin{center}
\textbf{Вариант B}
\end{center}

\begin{questions}

\question Функция $f$ задана формулой 
\[
f(x)=\begin{cases}
x^2\sin(1/x), \; \text{ если } x\neq 0 \\
0, \; \text{ если } x = 0
\end{cases}
\]
\begin{parts}
\part[8]
Найдите правую, $f'_+(0)$, и левую, $f'_-(0)$, производные функции $f$ в точке $x=0$
\begin{solution}
Правая:
\[
\lim_{x\to 0^+}\frac{f(x)-f(0)}{x}=\lim_{x\to 0^+}\frac{x^2\sin(1/x)-0}{x}=\lim_{t\to +\infty} \frac{\sin t -0 }{t}=0
\]
Левая:
\[
\lim_{x\to 0^-}\frac{f(x)-f(0)}{x}=\lim_{x\to 0^-}\frac{x^2\sin(1/x)-0}{x}=\lim_{t\to -\infty} \frac{\sin t -0 }{t}=0
\]
\end{solution}

\part[2]
Существует ли производная функции $f$ в точке $x=0$?
\begin{solution}
Левая производная  равняется правой производной --- производная в точке $x=0$  равна $0$.
\end{solution}

\end{parts}

\question[10] Вычислите интеграл
\[
\int \sin( \ln x) \, dx
\]

\begin{solution}
Интегрируем два раза по частям:
\[
\int \sin( \ln x) \, dx = 	x \sin( \ln x)  - \int \cos( \ln x) \, dx = x \sin( \ln x)  -  x \cos( \ln x)  - \int \sin( \ln x) \, dx 
\]


Выражаем искомый интеграл:
\[
\int \sin( \ln x) \, dx =\frac{1}{2}\left(  x \sin( \ln x)  -  x \cos( \ln x)  \right)
\]
\end{solution}


\question Рассмотрим систему уравнений $Ax=b$, где 

\[
x=\begin{pmatrix}
x_1 \\
x_2 \\
x_3 
\end{pmatrix}, \; 
A=\begin{pmatrix}
1 & -2 & 0 \\
2 & 1 &  -1\\
-1 & 7 & \alpha  
\end{pmatrix}, \;
b=\begin{pmatrix}
1 \\
\beta \\
2
\end{pmatrix}.
\]

\begin{parts}
\part[6]
Найдите ранг и определитель матрицы $A$  как функции от параметра $\alpha$
\begin{solution}
Находим определитель (например, разложением по столбцу содержащему $\alpha$):
\[
\det A=5\alpha+5
\]
Определитель обращается в ноль только при $\alpha = -1$. Отсюда делаем вывод и про ранг матрицы.
При $\alpha = -1$ ранг матрицы равен двум, а при $\alpha \neq -1$ он равен трём.
\end{solution}
\part[4]
Определите количество решений системы в зависимости от значений параметров $\alpha$ и $\beta$
\begin{solution}
Если $\alpha \neq -1$, то решение системы единственно вне зависимости от $\beta$. При $\alpha=-1$ строки матрицы $A$ линейно зависимы. Матрица $A$ необратима и система либо не имеет решений, либо имеет бесконечное множество решений. Выясним зависимость между строками матрицы $A$, выразим третью строку матрицы через первые две:
\[
(-1;7;-1)=y_1(1;-2;0)+y_2(2;1;-1)
\]
Решая эту систему находим, что $y_1=-3$, а $y_2=1$. Если это же соотношение выполняется для столбца $b$, то система имеет бесконечное количество решений, иначе --- ни одного.
\[
2=-3\cdot 1 + 1\cdot \beta
\]

Следовательно, при $\alpha=-1$ и $\beta=5$ система имеет бесконечное количество решений, а при $\alpha=-1$ и $\beta\neq 5$ --- ни одного.
\end{solution}

%\part 
%Какие значения принимает ранг матрицы $C$, состоящей из матрицы $A$ и приписанного справа вектора $b$, $C=[A \; b]$, в зависимости от параметров $\alpha$ и $\beta$?

\end{parts}



\question Вектор-строка $b$ состоит из последовательных чисел от 4 до 1, $b=(4,3,2,1)$. Матрица $B$ задана соотношением $B=b^Tb$. 

\begin{parts}
\part[5]
Найдите собственные числа матрицы $B$
\begin{solution}
Ранг матрицы равен количеству ненулевых собственных чисел. Ранг произведения матриц не превосходит ранга сомножителей, поэтому ранг матрицы $B$ равен одному. Матрица $B$ ненулевая, поэтому у неё три нулевых собственных числа и одно ненулевое. 

Заметим, что $B\cdot b^T=(b^Tb)b^T=b^T(bb^T)=b^T\cdot (16+9+4+1)=30b^T$. Следовательно, четвёртое собственное число --- 30.
\end{solution}

\part[3]
Для максимального собственного числа укажите хотя бы один собственный вектор
\begin{solution}
Попутно в прошлом пункте мы нашли, что у числа $30$ есть собственный вектор $b^T=(4,3,2,1)^T$.
\end{solution}
\part[2]
Является ли матрица $B$ положительно определённой? Положительно полуопределённой?
\begin{solution}
У матрицы $B$ нулевые и положительные собственные числа. Она является положительно полуопределённой, а положительно определённой не является.
\end{solution}
\end{parts}

\question Задано дифференциальное уравнение 
\[
x\frac{dy}{dx}-y=(x+y)\ln \left( \frac{x+y}{x}  \right)
\]
\begin{parts}
\part[8]
Решите дифференциальное уравнение
\begin{solution}
Уравнение однородно (сохраняет вид при одновременном умножении  $x$  и $y(x)$  на постоянный множитель), поэтому можно использовать замену  $y(x)=z(x)x$. 
 
Заметим, что в области определения уравнения $x\neq 0$, поэтому решения при такой замене не теряются.
\[
x(z'+z)-zx=(x+zx)\ln \frac{x+zx}{x}
\]
\[
z'=\frac{(1+z)\ln(1+z)}{x}
\]
 
Переменные разделяются
\[
\frac{dz}{(1+z)\ln(1+z)}=\frac{dx}{x}
\]

\[
\int \frac{dz}{(1+z)\ln(1+z)}=\int \frac{dx}{x}
\]

\[
\ln |\ln(1+z)|=\ln|x| + \ln C_0, \; C_0>0
\]
постоянную интегрирования, которая может иметь любой знак, удобно в данном случае записать как логарифм положительной постоянной
\[
|\ln(1+z)|=C_0|x|, \; C_0>0
\]
 
Снимая модуль в левой части получаем постоянную любого знака

\[
\ln(1+z)=C|x|
\]
 
или
 
 \[
1+z=e^{C|x|}
\]

При $x=0$  уравнение не имеет смысла,  и постоянные интегрирования при положительных и при отрицательных значениях $x$  можно выбирать независимо. Иначе говоря, найденное множество интегральных кривых можно перечислить выражением
 
  \[
1+z=e^{Cx}
\]
 
Возвращая подстановку, получаем решение
 
\[
y(x)=\left(e^{Cx}-1 \right)x
\]

\end{solution}

\part[2]
Дайте схематический рисунок интегральных кривых
\begin{solution}
\begin{tikzpicture}
  \draw[->] (-3,0) -- (3,0) node[right] {$x$};
  \draw[->] (0,-3) -- (0,3) node[above] {$y$};
  \draw[domain=-3:1.2,smooth,variable=\x,blue] plot ({\x},{(exp(\x)-1)*\x});
  \draw[domain=-1.2:3,smooth,variable=\x,red] plot ({\x},{(exp(-\x)-1)*\x});
\end{tikzpicture}
\end{solution}

\end{parts}

\question[10] 
Исследуйте на экстремумы функцию $F(x,y)=16x^3+2y^3-24xy-15$

\begin{solution}
Найдем точки, подозрительные на экстремум, решая следующую систему уравнений:
\[
\begin{cases}
\frac{\partial F}{\partial x}=48x^2-24y=0 \\
\frac{\partial F}{\partial y}=6y^2-24x=0 \\
\end{cases}
\]
Получим точки:  $(0;0)$, $(1;2)$.
Далее необходимо проверить выполнение условий второго порядка. Для этого найдем матрицу вторых производных исследуемой функции:
\[
\begin{pmatrix}
96x & -24 \\
-24 & 12y 
\end{pmatrix}.
\]



Проверим знакоопределенность этой матрицы в каждой из найденных подозрительных точек.


Для точки  $(0;0)$ имеем следующую матрицу:  
$\begin{pmatrix}
0 & -24 \\
-24 & 0
\end{pmatrix}$. 
Находим угловые миноры, $\Delta_1=0$, $\Delta_2=0-24^2<0$. Следовательно, точка $(0;0)$  не является точкой экстремума.


Для точки $(1;2)$  имеем следующую матрицу:
$\begin{pmatrix}
96 & -24 \\
-24 & 24
\end{pmatrix}.$  
Находим угловые миноры, $\Delta_1=96>0$, $\Delta_2=96\cdot 24 -24^2>0$.  Следовательно, точка $(1;2)$  является точкой минимума.

Значение функции в этой точке: $F(x,y)=-31$.
\end{solution}

\question Рассмотрим функцию $Q(x,y)=x-2y$, аргументы которой удовлетворяют условию $b+ax^2+y^2=0$.
Найдите при каких значениях параметров $a$ и $b$  функция $Q(x,y)$:

\begin{parts}
\part[4]
будет иметь ровно одну условную стационарную точку, определите, является ли данная точка экстремумом;
\part[4]
будет иметь более одной условной стационарной точки, определите, являются ли данные точки экстремумами;
\part[2]
не будет иметь стационарных точек.  
\end{parts}

Указание. Для нахождения условных стационарных точек используйте метод множителей Лагранжа. Дополнительных исследований проводить не требуется. 


\begin{solution}

\begin{enumerate}
\item Если $a=0,b\le 0$, то ограничению удовлетворяют все значения аргумента x при $y=\sqrt{-b} $. В этом случае остается найти экстремум функции $x-2\sqrt{-b} $, которого, очевидно не существует. В данном случае стационарных точек нет. 

\item  Если $a\ge 0,b>0$, то ограничению не удовлетворяет ни одна точка. В данном случае стационарных точек нет.

\item  Остается рассмотреть два случая $a>0$, $b<0$ и $a<0$, $b$-любое. В этом случае используем метод множителей Лагранжа. Функция Лагранжа имеет вид $L\left(x,y,\lambda \right)=x-2y+\lambda \left(b+ax^{2} +y^{2} \right)$. Безусловные стационарные точки функции Лагранжа совпадают с условными стационарными точками в постановке задачи. Стационарная точка определяется из условий 
\end{enumerate}

\[
\left\{\begin{array}{c} {\frac{\partial }{\partial x} L\left(x,y,\lambda \right)=1+2\lambda ax=0} \\ {\frac{\partial }{\partial y} L\left(x,y,\lambda \right)=-2+2\lambda y=0} \\ {\frac{\partial }{\partial \lambda } L\left(x,y,\lambda \right)=b+ax^{2} +y^{2} =0} \end{array}\right. 
\] 

Решение данной системы уравнений существует не всегда. 

С.1) При условии, что $a\ne -16$ решением являются точки

 $\left\{\begin{array}{c} {x_{k} =-\frac{1}{2\lambda a} } \\ {y_{k} =\frac{1}{\lambda } } \end{array}\right. ,\, \lambda =\pm \sqrt{-\frac{1+4a}{ab} } $. Далее решение при положительном значении $\lambda $ назовем $\left(x_{1} ,y_{1} \right)$, а при отрицательном $\lambda $ - $\left(x_{2} ,y_{2} \right)$. Определим их тип. Достаточным условием существования условного экстремума в условной стационарной точке является постоянство знака второго дифференциала функции Лагранжа при учете условия. Второй дифференциал имеет вид: $d^{2} L\left(x,y,\lambda \right)=\frac{\partial ^{2} }{\partial x^{2} } L\left(x,y,\lambda \right)dx^{2} +2\frac{\partial ^{2} }{\partial x\partial y} L\left(x,y,\lambda \right)dxdy+\frac{\partial ^{2} }{\partial y^{2} } L\left(x,y,\lambda \right)dy^{2} $ Из ограничения следует, что 
\[
dy=-\left({\frac{\partial }{\partial x} F\left(x,y\right) \mathord{\left/{\vphantom{\frac{\partial }{\partial x} F\left(x,y\right) \frac{\partial }{\partial y} F\left(x,y\right)}}\right.\kern-\nulldelimiterspace} \frac{\partial }{\partial y} F\left(x,y\right)} \right)dx
\]
Таким образом, тип стационарной точки определяется знаком выражения 
\[
\begin{array}{l} {A=\frac{\partial ^{2} }{\partial x^{2} } L\left(x_{k} ,y_{k} ,\lambda _{k} \right)+2\frac{\partial ^{2} }{\partial x\partial y} L\left(x_{k} ,y_{k} ,\lambda _{k} \right)\left({\frac{\partial }{\partial x} F\left(x_{k} ,y_{k} \right) \mathord{\left/{\vphantom{\frac{\partial }{\partial x} F\left(x_{k} ,y_{k} \right) \frac{\partial }{\partial y} F\left(x_{k} ,y_{k} \right)}}\right.\kern-\nulldelimiterspace} \frac{\partial }{\partial y} F\left(x_{k} ,y_{k} \right)} \right)+} \\ {+\frac{\partial ^{2} }{\partial y^{2} } L\left(x,y,\lambda \right)\left({\frac{\partial }{\partial x} F\left(x_{k} ,y_{k} \right) \mathord{\left/{\vphantom{\frac{\partial }{\partial x} F\left(x_{k} ,y_{k} \right) \frac{\partial }{\partial y} F\left(x_{k} ,y_{k} \right)}}\right.\kern-\nulldelimiterspace} \frac{\partial }{\partial y} F\left(x_{k} ,y_{k} \right)} \right)^{2} } \end{array}
\]

В случае $a>0,b<0$ или $a\in \left(-\frac{1}{4} ,0\right),b>0$ $A=2\lambda (\frac{1}{4} +a)$. Тогда $\left(x_{1} ,y_{1} \right)$ является точкой минимума, а $\left(x_{2} ,y_{2} \right)$ является точкой максимума.

В случае $a<-\frac{1}{4} ,b<0$ $A=2\lambda (\frac{1}{4} +a)$. Тогда $\left(x_{1} ,y_{1} \right)$ является точкой максимума, а $\left(x_{2} ,y_{2} \right)$ является точкой минимума.

В случае $a\in \left(-\frac{1}{4} ,0\right),b<0$ или $a<-16,b>0$ выражение под корнем оказывается отрицательным и стационарных точек нет.

С.2) Если $a=-\frac{1}{4} $, то $\lambda =0$ и в силу линейности функции $Q(x,y)$ стационарных точек нет. 



Таким образом:

\begin{enumerate}
\item  Единственной стационарной точки не существует.

\item  Две стационарные точки существуют в случае:


$a>0,b<0$ или $a\in \left(-16,0\right),b>0$. Тогда $\left(x_{1} ,y_{1} \right)$ является точкой минимума, а $\left(x_{2} ,y_{2} \right)$ является точкой максимума.

$a<-16,b<0$. Тогда $\left(x_{1} ,y_{1} \right)$ является точкой максимума, а $\left(x_{2} ,y_{2} \right)$ является точкой минимума.

\item Во всех остальных случаях стационарные точки отсутствуют.
\end{enumerate}



\end{solution}

\question В лотерее каждый десятый билет выигрывает, причём цена билета равна десяти рублям, а выигрыш составляет семьдесят рублей. Билетов очень-очень много, поэтому выигрыши по ним можно считать независимыми.
\begin{parts}
\part[5] Каковы математическое ожидание и дисперсия выигрыша при покупке восьми билетов? Имеется в~виду выигрыш с учётом затрат на~приобретение, так что он может быть отрицательным.

\begin{solution}
Имеем дело с восемью испытаниями по схеме Бернулли с вероятностью успеха $0.1$.  Пусть $X$ --- число успехов (выигрышей). Тогда $\E(X)=8\cdot 0.1=0.8$, $\Var(X)=8\cdot 0.1\cdot 0.9=0.72$. Сумма выигрыша с учётом стоимости билетов $Y=70X-80$ (здесь 80 --- стоимость восьми билетов). Получаем ответ:
\[
\E(Y)=70\E(X)-80=70\cdot 0.8-80=-24,
\]
\[
\Var(Y)=70^2 \Var(X)=3528.
\]

Разбалловка: 5 баллов за пункт (а), по одному баллу за $\E(X)$, $\E(Y)$, связь $Y$ и $X$, $\Var(X)$, $\Var(Y)$.
\end{solution}

\part[5]
Некто покупает лотерейные билеты до~третьего выигрыша. С~какой вероятностью ему придётся купить ровно двенадцать билетов?

\begin{solution}
На этот раз число испытаний не ограничено. Чтобы двенадцатый билет был третьим выигрышным, он должен сам быть выигрышным (вероятность этого - $0.1$), а одиннадцать предыдущих должны содержать ровно два выигрыша, вероятность чего равна $C_11^2\cdot 0.1^2\cdot 0.9^9$. Искомая вероятность: $C_11^2\cdot 0.1^2\cdot 0.9^9\cdot 0.1\approx 0.02$.

Разбалловка: 5 баллов за пункт (б), из них два за формулу Бернулли.
\end{solution}
\end{parts}

\newpage
\question Случайные величины $X_i$ независимы, а их распределение известно с~точностью до~параметра $p$:

\begin{center}
\begin{tabular}{lccc} \toprule
Значения & -3 & 0 & 1 \\ 
Вероятности & $0.1$ & $0.9-p$ & $p$ \\ \bottomrule
\end{tabular}
\end{center}

\begin{parts}
\part[5]
Пусть $p=0.3$. С какой вероятностью среднее в выборке $X_1, \ldots, X_{480}$ превысит значение $0.05$?
\begin{solution}
Найдём математическое ожидание и дисперсию $X_i$:
\[
\E(X_i)=-3\cdot 0.1+1\cdot p=p-0.3
\]
\[
\E(X_i^2 )=(-3)^2\cdot 0.1+1^2\cdot p=0.9+p,
\]
\[
\Var(X_i)=0.9+p-(p-0.3)^2=0.81+1.6p-p^2
\]
При $p=0.3$ получаем $\E(X_i )=0$, $\Var(X_i )=1.2$.
Объём выборки велик, так что выборочное среднее будет иметь приблизительно нормальное распределение: $\bar{X}\sim \cN(0,1.2/480=0.0025)$. Рассчитываем нужную нам вероятность, нормировав выборочное среднее:
\[
\P(\bar{X}>0.05)=\P\left(\bar{X} /\sqrt{0.0025}>0.05/\sqrt{0.0025} \right)=\P(\bar{X} /0.05>1)=0.159.
\]
Разбалловка: 5 баллов за пункт (а), по одному баллу за математическое ожидание, дисперсию, применение теоремы о распределении выборочного среднего (центральной предельной теоремы), нормирование, нахождение вероятности по таблицам;
\end{solution}

\part[5]
Докажите состоятельность оценки $\hat{p}=0.3+\frac{1}{n} \sum_{i=1}^n X_i$  для параметра $p$, где $n$ --- объём выборки.

\begin{solution}
Для доказательства состоятельности оценки достаточно показать, что она несмещённая, а её дисперсия стремится к нулю. Проверяем несмещённость:
\[
\E(\hat{p})=0.3+\E(\bar{X})=0.3+(p-0.3)=p.
\]
Ищем дисперсию: 
\[
\Var(\hat{p})=\Var(\bar{X})=(0.81+1.6p-p^2)/n.
\]
Ясно, что $\lim_{n\to\infty} \Var(\hat{p})=0$, так что оценка состоятельная.

Можно и не искать дисперсию $X_i$. Достаточно знать, что она конечна, а это следует из того, что множество значений $X_i$ конечно.


Разбалловка: 5 баллов за пункт (б), из них два – за достаточное условие состоятельности.
\end{solution}

\end{parts}



\question
Исследователь решил выяснить, есть ли связь между  гендерной принадлежностью и доходами индивида. В его распоряжении есть данные о заработных платах (переменная $wage$ --- средняя почасовая заработная плата в долларах), опыте (переменная $exper$ --- годы опыта) и поле  (дамми-переменная $gender$ принимает значение 1 для женщин). По 300-м наблюдениям он оценил следующее уравнение регрессии (предпосылки классической линейной регрессионной модели выполнены):
\[
\ln(wage_i) = \alpha + \beta_1 exper_i + \beta_2 exper_i^2 + \beta_3 gender_i + \varepsilon_i
\]

Результаты оценки уравнения представлены в таблице:
\begin{center}
\begin{tabular}{ccc} \toprule
Переменная & Коэффициент & Стандартная ошибка\\ \midrule
$exper$ & 0.0400 & 0.0134 \\
$exper^2$ & -0.0008 & 0.0004 \\
$gender$ & -0.0534 & 0.0847 \\
константа & -0.4860 & 0.2136 \\ \bottomrule
\end{tabular}
\end{center}


\begin{parts}
%\part[1]
%Почему исследователь решил включить в модель две переменных для наличия детей? 
%\begin{solution}
%Исследователь предполагает, что наличие детей разного возраста может оказывать разное по силе и направлению влияние на заработную плату женщины.
%\end{solution}

\part[1]
Выпишите оценённое уравнение регрессии.
\begin{solution}
\[
\widehat{\ln(wage_i)} = -0.4860 + 0.0400exper_i -0.0008 exper_i^2 - 0.0534 gender_i  
\]
\end{solution}

%\part[2]
%Как будут интерпретироваться величина коэффициент $\beta_3$ при переменной $educ$? 
%\begin{solution}
%При росте образования на 1 год заработная плата растёт примерно на $10.8$\%.
%\end{solution}

\part[6]
На уровне значимости 5\%-ов проверьте гипотезу о значимости связи гендерной принадлежности  и заработной платы против альтернативной об отсутствии связи. Выпишите нулевую и альтернативную гипотезы, укажите используемые формулы, рассчитайте необходимую статистику, укажите точный и асимптотический вид её распределения и сделайте вывод на её основе.

\begin{solution}
$H_0$: $\beta_3=0$, $H_a$: $\beta_3\neq 0$. $Z_{obs}=\frac{\hbeta_3}{se(\hbeta_3)}=-0.0534/ 0.0847\approx -0.6305$, $Z_{cr}\approx 1.95$. Расчетное значение тестовой статистики по модулю меньше критического, что не дает оснований отвергнуть нулевую гипотезу о незначимости коэффициента при переменной $kid6$. На уровне значимости 5\%-ов нет оснований утверждать, что существует связь между образованием и заработной платой. Точное распределение статистики --- $t_{296}$, асимптотическое --- $N(0;1)$.
\end{solution}


%\part[3]
%Рассчитайте  90\%-ый доверительный интервал для разницы математических ожиданий логарифмов заработной платы для женщины без детей и женщины с тремя детьми в возрасте 7, 10 и 13 лет при прочих равных условиях.
%\part
%Найдите точечную оценку для разницы в заработной плате женщины с опытом работы 5 лет, проучившейся 15 лет, с двумя детьми в возрасте 3 и 10 лет и для женщины с опытом работы 5 лет, проучившейся 15 лет, без детей.

%\begin{solution}
%Разница математических ожиданий логарифмов заработной платы для женщины без детей и женщины с детьми в возрасте 7, 10 и 13 лет при прочих равных условиях будет определяться значением коэффициента при переменной $kid18$.  Для расчета доверительного интервала для указанного коэффициента используется следующая формула: $[\hbeta_5 - Z_{cr} se(\hbeta_5); \hbeta_5 + Z_{cr} se(\hbeta_5)]$. 

%В данном случае $Z_{cr}=1.65$. Тогда нижняя граница интервала $-0.0125-1.65\cdot 0.0268=-0.0567$, верхняя граница $=-0.0125+1.65\cdot 0.0268=0.0317$.
%\end{solution}

\part[3]
Перечислите модельные предпосылки, которые были использованы при решении задачи

\begin{solution}
Детерминистическая версия:
\begin{enumerate}
\item Линейность зависимости $y$ от объясняющих переменных. 
\[
\ln(wage_i) = \alpha + \beta_1 exper_i + \beta_2 exper_i^2 + \beta_3 gender_i + \varepsilon_i
\]
\item Нет линейной зависимости между регрессорами. Матрица $X$ имеет полный ранг.
\item Нет систематической ошибки, $\E(\varepsilon_i)=0$
\item Гомоскедастичность $\Var(\varepsilon_i)=\sigma^2$
\item Некоррелированность ошибок $\Cov(\varepsilon_i,\varepsilon_j)=0$
\item Нормальность ошибок, $\varepsilon_i \sim N(0;\sigma^2)$
\end{enumerate}

Стохастическая версия:

\begin{enumerate}
\item Линейность зависимости $y$ от объясняющих переменных. 
\[
\ln(wage_i) = \alpha + \beta_1 exper_i + \beta_2 exper_i^2 + \beta_3 gender_i + \varepsilon_i
\]
\item С вероятностью один нет линейной зависимости между регрессорами. Матрица $X$ имеет полный ранг с вероятностью один.
\item Эндогенность, $\E(\varepsilon_i | X)=0$
\item Условная гомоскадастичность $\Var(\varepsilon_i |X)=\sigma^2$
\item Условная некоррелированность ошибок $\Cov(\varepsilon_i,\varepsilon_j |X)=0$
\item Нормальность ошибок, $\varepsilon_i \sim N(0;\sigma^2)$
\end{enumerate}

\end{solution}

\end{parts}

\end{questions}


\begin{figure}[b]
\caption{Таблица значений функции распределения для стандартной нормальной величины.}
  \begin{minipage}[b]{0.35\linewidth}
    \centering
    \begin{tikzpicture}
% define normal distribution function 'normaltwo'
    \def\normaltwo{\x,{4*1/exp(((\x-3)^2)/2)}}
 
% input y parameter
    \def\y{4.4}
 
% this line calculates f(y)
    \def\fy{4*1/exp(((\y-3)^2)/2)}
 
% Shade orange area underneath curve.
    \fill [fill=gray!30] (2.6,0) -- plot[domain=0:4.4] (\normaltwo) -- ({\y},0) -- cycle;
 
% Draw and label normal distribution function
    \draw[domain=0:6] plot (\normaltwo) node[right] {};
 
% Add dashed line dropping down from normal.
    \draw[dashed] ({\y},{\fy}) -- ({\y},0) node[below] {$x$};
 
% Optional: Add axis labels
%    \draw (-.2,2.5) node[left] {$f_Y(u)$};
    \draw (3,2) node[below] {$F(x)$};
 
% Optional: Add axes
    \draw[->] (0,0) -- (6.2,0) node[right] {};
%    \draw[->] (0,0) -- (0,5) node[above] {};
 
\end{tikzpicture}
%    \rule{6cm}{6cm} %to simulate an actual figure
\par\vspace{0pt}
  \end{minipage}%
  \begin{minipage}[b]{0.60\linewidth}
    \centering
\begin{tabular}{rr|rr|rr|rr}
  \hline
$x$ & $F(x)$ & $x$ & $F(x)$ & $x$ & $F(x)$ & $x$ & $F(x)$ \\ 
  \hline
0.050 & 0.520 & 0.750 & 0.773 & 1.450 & 0.926 & 2.150 & 0.984 \\ 
  0.100 & 0.540 & 0.800 & 0.788 & 1.500 & 0.933 & 2.200 & 0.986 \\ 
  0.150 & 0.560 & 0.850 & 0.802 & 1.550 & 0.939 & 2.250 & 0.988 \\ 
  0.200 & 0.579 & 0.900 & 0.816 & 1.600 & 0.945 & 2.300 & 0.989 \\ 
  0.250 & 0.599 & 0.950 & 0.829 & 1.650 & 0.951 & 2.350 & 0.991 \\ 
  0.300 & 0.618 & 1.000 & 0.841 & 1.700 & 0.955 & 2.400 & 0.992 \\ 
  0.350 & 0.637 & 1.050 & 0.853 & 1.750 & 0.960 & 2.450 & 0.993 \\ 
  0.400 & 0.655 & 1.100 & 0.864 & 1.800 & 0.964 & 2.500 & 0.994 \\ 
  0.450 & 0.674 & 1.150 & 0.875 & 1.850 & 0.968 & 2.550 & 0.995 \\ 
  0.500 & 0.691 & 1.200 & 0.885 & 1.900 & 0.971 & 2.600 & 0.995 \\ 
  0.550 & 0.709 & 1.250 & 0.894 & 1.950 & 0.974 & 2.650 & 0.996 \\ 
  0.600 & 0.726 & 1.300 & 0.903 & 2.000 & 0.977 & 2.700 & 0.997 \\ 
  0.650 & 0.742 & 1.350 & 0.911 & 2.050 & 0.980 & 2.750 & 0.997 \\ 
  0.700 & 0.758 & 1.400 & 0.919 & 2.100 & 0.982 & 2.800 & 0.997 \\ 
   \hline
\end{tabular}
\par\vspace{0pt}
\end{minipage}
\label{fig:test}
\end{figure}

\begin{flushright}
Удачи!
\end{flushright}

\end{document}