\documentclass[addpoints]{exam} % добавить или удалить answers в скобках, чтобы показать ответы
% \documentclass[addpoints, answers]{exam} % добавить или удалить answers в скобках, чтобы показать ответы


\usepackage[utf8]{inputenc}
\usepackage[russian]{babel}
%\usepackage[OT1]{fontenc}
\usepackage{booktabs}
\usepackage{amsmath}
\usepackage{tikz}
\usepackage{amsfonts}
\usepackage{amssymb}
\usepackage[left=2cm,right=2cm,top=2cm,bottom=2cm]{geometry}
\DeclareMathOperator{\E}{\mathbb{E}}
\DeclareMathOperator{\Var}{\mathbb{V}\mathrm{ar}}
\DeclareMathOperator{\Cov}{\mathbb{C}\mathrm{ov}}
\let\P\relax
\DeclareMathOperator{\P}{\mathbb{P}}
\newcommand{\cN}{\mathcal{N}}
\newcommand{\hbeta}{\hat{\beta}}

\usepackage{floatrow}
%\newfloatcommand{capbtabbox}{table}[][\FBwidth]

\begin{document}

\pagestyle{headandfoot}
\runningheadrule
\firstpageheader{НИУ-ВШЭ}{Высшая математика, стр. \thepage\ из \numpages}{17 июля 2015}
\firstpageheadrule
\runningheader{НИУ-ВШЭ}{Высшая математика, стр. \thepage\ из \numpages}{17 июля 2015}
\firstpagefooter{}{}{}
\runningfooter{}{}{}
\runningfootrule




\hqword{Задача}
\hpgword{Страница}
\hpword{Максимум}
\hsword{Баллы}
\htword{Итого}
\pointname{\%}
%\renewcommand{\solutiontitle}{\noindent\textbf{Решение:}\par\noindent}
\renewcommand{\solutiontitle}{}

%Таблица с результатами заполняется проверяющим работу. Пожалуйста, не делайте в ней пометок.

%\begin{center}
%  \gradetable[h][questions]
%\end{center}

\begin{center}
\textbf{Вариант А}
\end{center}

\begin{questions}

\question Функция $f$ задана формулой 
\[
f(x)=\begin{cases}
\frac{x}{1+\exp(1/x)}, \; \text{ если } x\neq 0 \\
0, \; \text{ если } x = 0
\end{cases}
\]
\begin{parts}
\part[8]
Найдите правую, $f'_+(0)$, и левую, $f'_-(0)$, производные функции $f$ в точке $x=0$
\begin{solution}
Правая:
\[
\lim_{x\to 0^+}\frac{f(x)-f(0)}{x}=\lim_{x\to 0^+}\frac{1}{1+\exp(1/x)}=0
\]
Левая:
\[
\lim_{x\to 0^-}\frac{f(x)-f(0)}{x}=\lim_{x\to 0^-}\frac{1}{1+\exp(1/x)}=1
\]

\end{solution}
\part[2]
Существует ли производная функции $f$ в точке $x=0$?
\begin{solution}
Левая и правая производные не равны, следовательно, производной в точке $x=0$ не существует.
\end{solution}

\end{parts}

\question[10] Вычислите интеграл
\[
\int  \ln (x + \sqrt{1+x^2}) \, dx
\]

\begin{solution}
Используем интегрирование по частям:
\[
\int \ln (x + \sqrt{1+x^2}) \, dx = x \ln (x + \sqrt{1+x^2}) - \int x \frac{1+\frac{x}{\sqrt{1+x^2}}}{x + \sqrt{1+x^2}} \, dx 
\]
Займёмся вторым интегралом:
\[
\int x \frac{1+\frac{x}{\sqrt{1+x^2}}}{x + \sqrt{1+x^2}} \, dx =\int x \frac{x+\sqrt{1+x^2}}{(x + \sqrt{1+x^2})\sqrt{1+x^2}} \, dx =
\frac{1}{2}\int  \frac{d(1+x^2)}{\sqrt{1+x^2}} = \sqrt{1+x^2} + C
\]
Итого, ответ:
\[
x \ln (x + \sqrt{1+x^2}) - \sqrt{1+x^2} + C
\]

\end{solution}

\question Рассмотрим систему уравнений $Ax=b$, где 

\[
x=\begin{pmatrix}
x_1 \\
x_2 \\
x_3 
\end{pmatrix}, \; 
A=\begin{pmatrix}
1 & -2 & \alpha \\
2 & 1 &  -1\\
-1 & 7 & -1  
\end{pmatrix}, \;
b=\begin{pmatrix}
1 \\
\beta \\
2
\end{pmatrix}.
\]

\begin{parts}
\part[6]
Найдите ранг и определитель матрицы $A$  как функции от параметра $\alpha$
\begin{solution}
Находим определитель (например, разложением по столбцу содержащему $\alpha$):
\[
\det A=15\alpha
\]
Определитель обращается в ноль только при $\alpha = 0$. Отсюда делаем вывод и про ранг матрицы.
При $\alpha = 0$ ранг матрицы равен двум, а при $\alpha \neq 0$ он равен трём.
\end{solution}
\part[4]
Определите количество решений системы в зависимости от значений параметров $\alpha$ и $\beta$
\begin{solution}
Если $\alpha \neq 0$, то решение системы единственно вне зависимости от $\beta$. При $\alpha=0$ строки матрицы $A$ линейно зависимы. Матрица $A$ необратима и система либо не имеет решений, либо имеет бесконечное множество решений. Выясним зависимость между строками матрицы $A$, выразим третью строку матрицы через первые две:
\[
(-1;7;-1)=y_1(1;-2;0)+y_2(2;1;-1)
\]
Решая эту систему находим, что $y_1=-3$, а $y_2=1$. Если это же соотношение выполняется для столбца $b$, то система имеет бесконечное количество решений, иначе --- ни одного.
\[
2=-3\cdot 1 + 1\cdot \beta
\]

Следовательно, при $\alpha=0$ и $\beta=5$ система имеет бесконечное количество решений, а при $\alpha=0$ и $\beta\neq 5$ --- ни одного.
\end{solution}
%\part 
%Какие значения принимает ранг матрицы $C$, состоящей из матрицы $A$ и приписанного справа вектора $b$, $C=[A \; b]$, в зависимости от параметров $\alpha$ и $\beta$?

\end{parts}


\question Вектор-строка $a$ состоит из последовательных чисел от 1 до 4, $a=(1,2,3,4)$. Матрица $A$ задана соотношением $A=a^Ta$. 

\begin{parts}
\part[5]
Найдите собственные числа матрицы $A$
\begin{solution}
Ранг матрицы равен количеству ненулевых собственных чисел. Ранг произведения матриц не превосходит ранга сомножителей, поэтому ранг матрицы $A$ равен одному. Матрица $A$ ненулевая, поэтому у неё три нулевых собственных числа и одно ненулевое. 

Заметим, что $A\cdot a^T=(a^Ta)a^T=a^T(aa^T)=a^T\cdot (1+4+9+16)=30a^T$. Следовательно, четвёртое собственное число --- 30.
\end{solution}

\part[3]
Для максимального собственного числа укажите хотя бы один собственный вектор
\begin{solution}
Попутно в прошлом пункте мы нашли, что у числа $30$ есть собственный вектор $a^T=(1,2,3,4)^T$.
\end{solution}
\part[2]
Является ли матрица $A$ положительно определённой? Положительно полуопределённой?
\begin{solution}
У матрицы $A$ нулевые и положительные собственные числа. Она является положительно полуопределённой, а положительно определённой не является.
\end{solution}
\end{parts}




\question Задано дифференциальное уравнение 
\[
x\frac{dy}{dx}-xy-e^x=0
\]
\begin{parts}
\part[8]
Решите дифференциальное уравнение
\begin{solution}

\[
y'=y+e^x/x
\]
Уравнение линейное. Решаем методом вариации постоянной. Однородное уравнение имеет вид   
\[
y'=y
\] 
 
Его общее решение имеет вид (экспонента подходит, а общее решение однородного уравнения пропорционально любому частному)
\[
y(x)=Ce^x
\]
поэтому в исходном уравнении делаем замену $y(x)=z(x)e^x$:
\[
z' e^x+ze^x=ze^x+e^x/x 
\]

\[
z'=1/x
\]

Интегрируя получаем
\[
z(x)=\ln |x| +C
\] 
 
Вoзвращая подстановку, получаем решение
\[
y(x)=(\ln |x| +C)e^x
\]

\end{solution}


\part[2]
Дайте схематический рисунок интегральных кривых
\begin{solution}
\begin{tikzpicture}
  \draw[->] (-3,0) -- (1.5,0) node[right] {$x$};
  \draw[->] (0,-3) -- (0,3) node[above] {$y$};
  \draw[domain=-3:-0.02,smooth,variable=\x,blue] plot ({\x},{ln(abs(\x))+1)*exp(\x)});
  \draw[domain=-3:-0.05,smooth,variable=\x,red] plot ({\x},{ln(abs(\x))+2)*exp(\x)});
  
  \draw[domain=0.02:1,smooth,variable=\x,blue] plot ({\x},{ln(abs(\x))+1)*exp(\x)});
  \draw[domain=0.02:0.6,smooth,variable=\x,red] plot ({\x},{ln(abs(\x))+2)*exp(\x)});
\end{tikzpicture}
\end{solution}
\end{parts}

\question[10] 
Исследуйте на экстремумы функцию $F(x,y)= x^3+8y^3-12xy+23$

\begin{solution}
Найдем точки, подозрительные на экстремум, решая следующую систему уравнений:
\[
\begin{cases}
\frac{\partial F}{\partial x}=3x^2-12y=0 \\
\frac{\partial F}{\partial y}=24y^2-12x=0 \\
\end{cases}
\]
Получим точки:  $(0;0)$, $(2;1)$.
Далее необходимо проверить выполнение условий второго порядка. Для этого найдем матрицу вторых производных исследуемой функции:
\[
\begin{pmatrix}
6x & -12 \\
-12 & 48y 
\end{pmatrix}.
\]



Проверим знакоопределенность этой матрицы в каждой из найденных подозрительных точек.


Для точки  $(0;0)$ имеем следующую матрицу:  
$\begin{pmatrix}
0 & -12 \\
-12 & 0
\end{pmatrix}$. 
Находим угловые миноры, $\Delta_1=0$, $\Delta_2=0-12^2<0$. Следовательно, точка $(0;0)$  не является точкой экстремума.


Для точки $(2;1)$  имеем следующую матрицу:
$\begin{pmatrix}
12 & -12 \\
-12 & 48
\end{pmatrix}.$  
Находим угловые миноры, $\Delta_1=12>0$, $\Delta_2=12\cdot 48-12^2>0$.  Следовательно, точка $(2;1)$  является точкой минимума.

Значение функции в этой точке: $F(x,y)=15$.
\end{solution}



\question Рассмотрим функцию $Q(x,y)=4x-y$, аргументы которой удовлетворяют условию $b+ax^2+y^2=0$.

Найдите при каких значениях параметров $a$ и $b$  функция $Q(x,y)$:

\begin{parts}
\part[4]
будет иметь ровно одну условную стационарную точку, определите, является ли данная точка экстремумом;
\part[4]
будет иметь более одной условной стационарной точки, определите, являются ли данные точки экстремумами;
\part[2]
не будет иметь стационарных точек.  
\end{parts}

Указание. Для нахождения условных стационарных точек используйте метод множителей Лагранжа. Дополнительных исследований проводить не требуется. 

\begin{solution}
\begin{enumerate}
\item Если $a=0,b\le 0$, то ограничению удовлетворяют все значения аргумента x при $y=\sqrt{-b} $. В этом случае остается найти экстремум функции $4x-\sqrt{-b} $, которого, очевидно не существует. В данном случае стационарных точек нет. 

\item  Если $a\ge 0,b>0$, то ограничению не удовлетворяет ни одна точка. В данном случае стационарных точек нет.

\item  Остается рассмотреть два случая $a>0$, $b<0$ и $a<0$, $b$-любое. В этом случае используем метод множителей Лагранжа. Функция Лагранжа имеет вид $L\left(x,y,\lambda \right)=4x-y+\lambda \left(b+ax^{2} +y^{2} \right)$. Безусловные стационарные точки функции Лагранжа совпадают с условными стационарными точками в постановке задачи. Стационарная точка определяется из условий 
\end{enumerate}

\[
\left\{\begin{array}{c} {\frac{\partial }{\partial x} L\left(x,y,\lambda \right)=4+2\lambda ax=0} \\ {\frac{\partial }{\partial y} L\left(x,y,\lambda \right)=-1+2\lambda y=0} \\ {\frac{\partial }{\partial \lambda } L\left(x,y,\lambda \right)=b+ax^{2} +y^{2} =0} \end{array}\right. 
\]

Решение данной системы уравнений существует не всегда. 

С.1) При условии, что $a\ne -16$ решением являются точки

 $\left\{\begin{array}{c} {x_{k} =-\frac{2}{\lambda a} } \\ {y_{k} =\frac{1}{2\lambda } } \end{array}\right. ,\, \lambda =\pm \sqrt{-\frac{16+a}{4ab} } $. Далее решение при положительном значении $\lambda $ назовем $\left(x_{1} ,y_{1} \right)$, а при отрицательном $\lambda $ - $\left(x_{2} ,y_{2} \right)$. Определим их тип. Достаточным условием существования условного экстремума в условной стационарной точке является постоянство знака второго дифференциала функции Лагранжа при учете условия. Второй дифференциал имеет вид: $d^{2} L\left(x,y,\lambda \right)=\frac{\partial ^{2} }{\partial x^{2} } L\left(x,y,\lambda \right)dx^{2} +2\frac{\partial ^{2} }{\partial x\partial y} L\left(x,y,\lambda \right)dxdy+\frac{\partial ^{2} }{\partial y^{2} } L\left(x,y,\lambda \right)dy^{2} $ Из ограничения следует, что 
\[
dy=-\left({\frac{\partial }{\partial x} F\left(x,y\right) \mathord{\left/{\vphantom{\frac{\partial }{\partial x} F\left(x,y\right) \frac{\partial }{\partial y} F\left(x,y\right)}}\right.\kern-\nulldelimiterspace} \frac{\partial }{\partial y} F\left(x,y\right)} \right)dx
\]
Таким образом, тип стационарной точки определяется знаком выражения 
\[
\begin{array}{l} {A=\frac{\partial ^{2} }{\partial x^{2} } L\left(x_{k} ,y_{k} ,\lambda _{k} \right)+2\frac{\partial ^{2} }{\partial x\partial y} L\left(x_{k} ,y_{k} ,\lambda _{k} \right)\left({\frac{\partial }{\partial x} F\left(x_{k} ,y_{k} \right) \mathord{\left/{\vphantom{\frac{\partial }{\partial x} F\left(x_{k} ,y_{k} \right) \frac{\partial }{\partial y} F\left(x_{k} ,y_{k} \right)}}\right.\kern-\nulldelimiterspace} \frac{\partial }{\partial y} F\left(x_{k} ,y_{k} \right)} \right)+} \\ {+\frac{\partial ^{2} }{\partial y^{2} } L\left(x,y,\lambda \right)\left({\frac{\partial }{\partial x} F\left(x_{k} ,y_{k} \right) \mathord{\left/{\vphantom{\frac{\partial }{\partial x} F\left(x_{k} ,y_{k} \right) \frac{\partial }{\partial y} F\left(x_{k} ,y_{k} \right)}}\right.\kern-\nulldelimiterspace} \frac{\partial }{\partial y} F\left(x_{k} ,y_{k} \right)} \right)^{2} } \end{array}.
\] 

В случае $a>0,b<0$ или $a\in \left(-16,0\right),b>0$ $A=2\lambda (16+a)$. Тогда $\left(x_{1} ,y_{1} \right)$ является точкой минимума, а $\left(x_{2} ,y_{2} \right)$ является точкой максимума.

В случае $a<-16,b<0$ $A=2\lambda (16+a)$. Тогда $\left(x_{1} ,y_{1} \right)$ является точкой максимума, а $\left(x_{2} ,y_{2} \right)$ является точкой минимума.

В случае $a\in \left(-16,0\right),b<0$ или $a<-16,b>0$ выражение под корнем оказывается отрицательным и стационарных точек нет.

С.2) Если $a=-16$, то $\lambda =0$ и в силу линейности функции $Q(x,y)$ стационарных точек нет. 



Таким образом:

\begin{enumerate}
\item  Единственной стационарной точки не существует.

\item  Две стационарные точки существуют в случае:


$a>0,b<0$ или $a\in \left(-16,0\right),b>0$. Тогда $\left(x_{1} ,y_{1} \right)$ является точкой минимума, а $\left(x_{2} ,y_{2} \right)$ является точкой максимума.

$a<-16,b<0$. Тогда $\left(x_{1} ,y_{1} \right)$ является точкой максимума, а $\left(x_{2} ,y_{2} \right)$ является точкой минимума.

\item Во всех остальных случаях стационарные точки отсутствуют.
\end{enumerate}
\end{solution}


\question  Меткий стрелок Василиса стреляет по мишени, каждый раз поражая цель с вероятностью $0.95$ независимо от других выстрелов. По условиям соревнования Василиса стреляет десять раз, причём за каждое попадание она получает три очка, а за промах теряет одно очко, так что может получить и отрицательный результат.
\begin{parts}
\part[5]
Каковы математическое ожидание и дисперсия числа полученных Василисой очков?
\begin{solution}
Имеем дело с десятью испытаниями по схеме Бернулли с вероятностью успеха $0.95$.  Пусть $X$ --- число успехов (попаданий). Тогда $\E(X)=10 \cdot 0.95=9.5$, $\Var(X)=10\cdot 0.95\cdot 0.05=0.475$. Число полученных Василисой очков $Y=3X-(10-X)=4X-10$. Отсюда ответ:
\[
\E(Y)=4\E(X)-10=4 \cdot 9.5-10=28,
\]
\[
\Var(Y)=16\Var(X)=7.6.
\]
\end{solution}

\part[5]
На следующих соревнованиях правила другие: Василиса будет стрелять до второго промаха. С~какой вероятностью она сделает ровно шесть выстрелов?

\begin{solution}
На этот раз число испытаний не ограничено. Чтобы в стрельбе до второго промаха было сделано шесть выстрелов, нужно в первых пяти попытках промахнуться ровно один раз (вероятность этого равна $C_5^1\cdot 0.95^4 \cdot 0.05$), а в шестом выстреле ещё раз промахнуться, что произойдёт с вероятностью $0.05$. Искомая вероятность: $C_5^1\cdot 0.95^4\cdot 0.05\cdot 0.05 \approx 0.01$.


Разбалловка: пять баллов за пункт (а), по одному баллу за $\E(X)$, $\E(Y)$, связь $Y$ и $X$, $\Var(X)$, $\Var(Y)$;
пять баллов за пункт (б), из них два - за формулу Бернулли.
\end{solution}


\end{parts}

\newpage
\question  Случайные величины $X_i$ независимы, а их распределение известно с точностью до параметра $p$:

\begin{center}
\begin{tabular}{lccc} \toprule
Значения & -1 & 0 & 2 \\ 
Вероятности & $p$ & $0.8-p$ & $0.2$ \\ \bottomrule
\end{tabular}
\end{center}

\begin{parts}
\part[5]
Пусть $p=0.4$. С какой вероятностью среднее в выборке $X_1, \ldots, X_{120}$ превысит значение $0.08$?

\begin{solution}
Найдём математическое ожидание и дисперсию $X_i$:
\[
\E(X_i)=-p+2\cdot 0.2=0.4-p
\]
\[
\E(X_i^2 )=(-1)^2 p+2^2\cdot 0.2=0.8+p,
\]
\[
\Var(X_i)=0.8+p-(0.4-p)^2=0.64+1.8p-p^2
\]
При $p=0.4$ получаем $\E(X_i )=0$, $\Var(X_i )=1.2$.
Объём выборки велик, так что выборочное среднее будет иметь приблизительно нормальное распределение: $\bar{X}\sim \cN(0,1.2/120=0.01)$. Рассчитываем нужную нам вероятность, нормировав выборочное среднее:
\[
\P(\bar{X}>0.08)=\P\left(\bar{X} /\sqrt{0.01}>0.08/\sqrt{0.01} \right)=\P(\bar{X} /0.1>0.8)=0.212.
\]
Разбалловка: 5 баллов за пункт (а), по одному баллу за математическое ожидание, дисперсию, применение теоремы о распределении выборочного среднего (центральной предельной теоремы), нормирование, нахождение вероятности по таблицам;
\end{solution}

\part[5]
Докажите состоятельность оценки $\hat{p}=0.4-\frac{1}{n} \sum_{i=1}^n X_i$  для параметра $p$, где $n$ --- объём выборки.

\begin{solution}
Для доказательства состоятельности оценки достаточно показать, что она несмещённая, а её дисперсия стремится к нулю. Проверяем несмещённость:
\[
\E(\hat{p})=0.4-\E(\bar{X})=0.4-(0.4-p)=p.
\]
Ищем дисперсию: 
\[
\Var(\hat{p})=\Var(\bar{X})=(0.64+1.8p-p^2)/n.
\]
Ясно, что $\lim_{n\to\infty} \Var(\hat{p})=0$, так что оценка состоятельная.

Можно и не искать дисперсию $X_i$. Достаточно знать, что она конечна, а это следует из того, что множество значений $X_i$ конечно.


Разбалловка: 5 баллов за пункт (б), из них два – за достаточное условие состоятельности.
\end{solution}
\end{parts}


\question
Выпускник бакалавриата, изучавший эконометрику, принимает решение о поступлении в магистратуру и хочет оценить возможную прибавку к заработной плате, которую обеспечит ему диплом магистра. В его распоряжении есть данные о заработных платах (переменная $wage$ --- средняя почасовая заработная плата в долларах), опыте (переменная $exper$ --- годы опыта) и  уровне образования (переменная $educ$ --- количество лет обучения).  По 200-м наблюдениям он оценил следующее уравнение регрессии (предпосылки классической линейной регрессионной модели  выполнены):
\[
\ln(wage_i) = \alpha + \beta_1 exper_i + \beta_2 exper_i^2 + \beta_3 educ_i   + \varepsilon_i
\]

Результаты оценки уравнения представлены в таблице:
\begin{center}
\begin{tabular}{ccc} \toprule
Переменная & Коэффициент & Стандартная ошибка\\ \midrule
$exper$ & 0.0389 & 0.0048 \\
$exper^2$ & -0.0007 & 0.0001 \\
$educ$ & 0.0841 & 0.0070 \\
константа & 0.3905 & 0.1022 \\ \bottomrule
\end{tabular}
\end{center}

\begin{parts}
%\part[1]
%Почему выпускник решил включить в модель квадрат опыта? 
%\begin{solution}
%Выпускник включил в модель квадрат опыта, поскольку предполагает нелинейную зависимость логарифма заработной платы от опыта. Он считает, что с ростом опыта заработная плата может изменяться немонотонно, например, при малых значениях опыта его увеличение будет приводить к росту заработной платы, а при больших значениях, наоборот, к её уменьшению.
%\end{solution}
\part[1]
Выпишите оценённое уравнение регрессии.
\begin{solution}
\[
\widehat{\ln(wage_i)} = 0.3905 + 0.0389 exper_i -0.0007 exper_i^2 + 0.0841 educ_i  
\]
\end{solution}

%\part[2]
%Как будут интерпретироваться величина коэффициента $\beta_3$ при переменной $educ$?
%\begin{solution}
%При росте образования на 1 год заработная плата растёт примерно на $8.41$\%.
%\end{solution}

\part[6]
На уровне значимости 5\%-ов проверьте гипотезу о связи образования и заработной платы против альтернативной гипотезы об отсутствии связи. Выпишите нулевую и альтернативную гипотезы, укажите используемые формулы, рассчитайте необходимую статистику, укажите точный и асимптотический вид её распределения и сделайте вывод на её основе.

\begin{solution}
$H_0$: $\beta_3=0$, $H_a$: $\beta_3\neq 0$. $Z_{obs}=\frac{\hbeta_3}{se(\hbeta_3)}=0.0841/0.0070\approx 12.01$, $Z_{cr}\approx 1.95$. Расчетное значение тестовой статистики по модулю больше критического, что дает основания отвергнуть нулевую гипотезу о незначимости коэффициента при переменной $educ$. На уровне значимости 5\%-ов есть основания утверждать, что существует связь между образованием и заработной платой. Точное распределение статистики --- $t_{196}$, асимптотическое --- $N(0;1)$.
\end{solution}

%\part[3]
%Рассчитайте 90\%-ый доверительный интервал для разницы математических ожиданий логарифмов заработной платы мужчин и женщин с одинаковым опытом и образованием.

%\begin{solution}
%Разница математических ожиданий логарифмов заработной платы для мужчин и женщин с одинаковыми характеристиками будет определяться значением коэффициента при переменной $gender$. Для расчета доверительного интервала для указанного коэффициента используется следующая формула: $[\hbeta_4 - Z_{cr} se(\hbeta_4); \hbeta_4 + Z_{cr} se(\hbeta_4)]$. 

%В данном случае $Z_{cr}=1.65$. Тогда нижняя граница интервала $-0.3372-1.65 \cdot 0.0363=-0.3970$, верхняя граница $-0.3372+1.65\cdot 0.0363=-0.2773$.
%\end{solution}

%\part
%Найдите точечную оценку для разницы в заработной плате выпускника бакалавриата (считайте, что срок обучения в школе составляет 11 лет) с опытом работы 2 года и выпускника магистратуры без опыта работы.

\part[3]
Перечислите модельные предпосылки, которые были использованы при решении задачи

\begin{solution}
Детерминистическая версия:
\begin{enumerate}
\item Линейность зависимости $y$ от объясняющих переменных. 
\[
\ln(wage_i) = \alpha + \beta_1 exper_i + \beta_2 exper_i^2 + \beta_3 educ_i  + \varepsilon_i
\]
\item Нет линейной зависимости между регрессорами. Матрица $X$ имеет полный ранг.
\item Нет систематической ошибки, $\E(\varepsilon_i)=0$
\item Гомоскедастичность $\Var(\varepsilon_i)=\sigma^2$
\item Некоррелированность ошибок $\Cov(\varepsilon_i,\varepsilon_j)=0$
\item Нормальность ошибок, $\varepsilon_i \sim N(0;\sigma^2)$
\end{enumerate}

Стохастическая версия:

\begin{enumerate}
\item Линейность зависимости $y$ от объясняющих переменных. 
\[
\ln(wage_i) = \alpha + \beta_1 exper_i + \beta_2 exper_i^2 + \beta_3 educ_i  + \varepsilon_i
\]
\item С вероятностью один нет линейной зависимости между регрессорами. Матрица $X$ имеет полный ранг с вероятностью один.
\item Эндогенность, $\E(\varepsilon_i | X)=0$
\item Условная гомоскадастичность $\Var(\varepsilon_i |X)=\sigma^2$
\item Условная некоррелированность ошибок $\Cov(\varepsilon_i,\varepsilon_j |X)=0$
\item Нормальность ошибок, $\varepsilon_i \sim N(0;\sigma^2)$
\end{enumerate}

\end{solution}

\end{parts}

\end{questions}



\begin{figure}[b]
\caption{Таблица значений функции распределения для стандартной нормальной величины}
  \begin{minipage}[b]{0.35\linewidth}
    \centering
    \begin{tikzpicture}
% define normal distribution function 'normaltwo'
    \def\normaltwo{\x,{4*1/exp(((\x-3)^2)/2)}}
 
% input y parameter
    \def\y{4.4}
 
% this line calculates f(y)
    \def\fy{4*1/exp(((\y-3)^2)/2)}
 
% Shade orange area underneath curve.
    \fill [fill=gray!30] (2.6,0) -- plot[domain=0:4.4] (\normaltwo) -- ({\y},0) -- cycle;
 
% Draw and label normal distribution function
    \draw[domain=0:6] plot (\normaltwo) node[right] {};
 
% Add dashed line dropping down from normal.
    \draw[dashed] ({\y},{\fy}) -- ({\y},0) node[below] {$x$};
 
% Optional: Add axis labels
%    \draw (-.2,2.5) node[left] {$f_Y(u)$};
    \draw (3,2) node[below] {$F(x)$};
 
% Optional: Add axes
    \draw[->] (0,0) -- (6.2,0) node[right] {};
%    \draw[->] (0,0) -- (0,5) node[above] {};
 
\end{tikzpicture}
%    \rule{6cm}{6cm} %to simulate an actual figure
\par\vspace{0pt}
  \end{minipage}%
  \begin{minipage}[b]{0.60\linewidth}
    \centering
\begin{tabular}{rr|rr|rr|rr}
  \hline
$x$ & $F(x)$ & $x$ & $F(x)$ & $x$ & $F(x)$ & $x$ & $F(x)$ \\ 
  \hline
0.050 & 0.520 & 0.750 & 0.773 & 1.450 & 0.926 & 2.150 & 0.984 \\ 
  0.100 & 0.540 & 0.800 & 0.788 & 1.500 & 0.933 & 2.200 & 0.986 \\ 
  0.150 & 0.560 & 0.850 & 0.802 & 1.550 & 0.939 & 2.250 & 0.988 \\ 
  0.200 & 0.579 & 0.900 & 0.816 & 1.600 & 0.945 & 2.300 & 0.989 \\ 
  0.250 & 0.599 & 0.950 & 0.829 & 1.650 & 0.951 & 2.350 & 0.991 \\ 
  0.300 & 0.618 & 1.000 & 0.841 & 1.700 & 0.955 & 2.400 & 0.992 \\ 
  0.350 & 0.637 & 1.050 & 0.853 & 1.750 & 0.960 & 2.450 & 0.993 \\ 
  0.400 & 0.655 & 1.100 & 0.864 & 1.800 & 0.964 & 2.500 & 0.994 \\ 
  0.450 & 0.674 & 1.150 & 0.875 & 1.850 & 0.968 & 2.550 & 0.995 \\ 
  0.500 & 0.691 & 1.200 & 0.885 & 1.900 & 0.971 & 2.600 & 0.995 \\ 
  0.550 & 0.709 & 1.250 & 0.894 & 1.950 & 0.974 & 2.650 & 0.996 \\ 
  0.600 & 0.726 & 1.300 & 0.903 & 2.000 & 0.977 & 2.700 & 0.997 \\ 
  0.650 & 0.742 & 1.350 & 0.911 & 2.050 & 0.980 & 2.750 & 0.997 \\ 
  0.700 & 0.758 & 1.400 & 0.919 & 2.100 & 0.982 & 2.800 & 0.997 \\ 
   \hline
\end{tabular}
\par\vspace{0pt}
\end{minipage}
\label{fig:test}
\end{figure}

\begin{flushright}
Удачи!
\end{flushright}

\end{document}