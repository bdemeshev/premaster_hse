%\documentclass[addpoints, answers]{exam} % добавить или удалить answers в скобках, чтобы показать ответы
\documentclass[addpoints]{exam} % добавить или удалить answers в скобках, чтобы показать ответы

\usepackage{etex}

\usepackage{verbatim}

\usepackage[utf8]{inputenc}
\usepackage[russian]{babel}
%\usepackage[OT1]{fontenc}
\usepackage{booktabs}
\usepackage{amsmath}
\usepackage{tikz}
\usepackage{amsfonts}
\usepackage{amssymb}
\usepackage[left=2cm, right=2cm, top=2cm, bottom=2cm]{geometry}
\DeclareMathOperator{\E}{\mathbb{E}}
\DeclareMathOperator{\Var}{\mathbb{V}\mathrm{ar}}
\DeclareMathOperator{\Cov}{\mathbb{C}\mathrm{ov}}
\DeclareMathOperator{\Corr}{\mathbb{C}\mathrm{orr}}
\DeclareMathOperator{\sgn}{sgn}
\DeclareMathOperator{\tr}{trace}
\DeclareMathOperator{\trace}{trace}
\DeclareMathOperator{\rank}{rank}

\let\P\relax
\DeclareMathOperator{\P}{\mathbb{P}}
\newcommand{\cN}{\mathcal{N}}
\newcommand{\RR}{\mathbb{R}}
\newcommand{\hbeta}{\hat{\beta}}
\newcommand{\hb}{\hat{\beta}}


\usepackage{floatrow}
%\newfloatcommand{capbtabbox}{table}[][\FBwidth]

\begin{document}

\pagestyle{headandfoot}
\runningheadrule
\firstpageheader{Higher School of Economics}{Further mathematics, p. \thepage\ of \numpages}{demo 2018}
\firstpageheadrule
\runningheader{Higher School of Economics}{Further mathematics, p. \thepage\ of \numpages}{demo 2018}
\firstpagefooter{}{}{}
\runningfooter{}{}{}
\runningfootrule




\hqword{Задача}
\hpgword{Страница}
\hpword{Максимум}
\hsword{Баллы}
\htword{Итого}
\pointname{\%}
%\renewcommand{\solutiontitle}{\noindent\textbf{Решение:}\par\noindent}
\renewcommand{\solutiontitle}{}

%Таблица с результатами заполняется проверяющим работу. Пожалуйста, не делайте в ней пометок.

%\begin{center}
%  \gradetable[h][questions]
%\end{center}

\begin{center}
\textbf{Exam demo-version} % Вариант А
\end{center}

\begin{questions}

\question[10] Evaluate the following limit:

\[
\lim_{x \to 0} \sqrt[x]{\cos \sqrt{x}}
\]

\begin{solution}

\[
\lim_{x \rightarrow 0} \sqrt[x]{\cos \sqrt{x}} = \lim_{x \rightarrow 0} \left({\cos \sqrt{x}}\right) ^{1/x} =  \lim_{x \rightarrow 0} \exp
\left( \frac{\ln (\cos \sqrt{x})}{x} \right) = \exp \left( \lim_{x \rightarrow 0}
\frac{\ln (\cos \sqrt{x})}{x} \right),
\]

In the last equality we interchanged limit and continuous function. Now we use Taylor's expansion:

\[
\frac{\ln (\cos \sqrt{x})}{x} = \frac{\ln (1-
\frac{x}{2}+\frac{x^2}{24}+o(x^2))}{x} = \frac{-\frac{x}{2}+o(x)}{x}
\]

\noindent It follows that

\[
\lim_{x \rightarrow 0} \frac{\ln (\cos \sqrt{x})}{x} = -\frac{1}{2},
\]

\noindent And finally

\[
\lim_{x \rightarrow 0} \sqrt[x]{\cos \sqrt{x}} = \exp \left( -\frac{1}{2}\right) = \frac{1}{\sqrt{e}}.
\]
\end{solution}

\question[10] Find and classify the discontinuity points of the following function:

\[
f(x) = {\sgn} \left(\sin \left( \frac{\pi}{x}\right)\right).
\]

\begin{solution}
The discontinuity points are: the point $x=0$, as denominator is zero, and the points $x=1/k, k \in \mathbb{Z}$, as $\sin \left( \frac{\pi}{x}\right)$ changes sign.


At the points $x=1/k, k \in \mathbb{Z}$ the function has first order discontinuities, as one-side limits exist but are not equal.

For example, let's consider $k=1$. In a small right neighbourhood of the point $x=1$, the function $\sin \left( \frac{\pi}{x} \right)$ is positive, as for $x \in (1,2)$ one has $\frac{\pi}{2} < \frac{\pi}{x} < \pi$. For all points of this neighbourhood one has $f(x) = 1$, hence, $\lim_{x \rightarrow 1+0} f(x) = 1$.

In a small right neighbourhood of the point $x=1$, the function $\sin \left( \frac{\pi}{x} \right)$ is negative as for $x \in (1/2,1)$ one has $\pi < \frac{\pi}{x} < 2 \pi$. For all points of this neighbourhood one has $f(x) = -1$, hence, $\lim_{x \rightarrow 1-0} f(x) = -1$. Similar neighbourhoods may be found for other values of $k$.



At the point $x=0$ the function has second order discontinuity, as one sided limits do not exist. Let's consider the sequences $a_n = \frac{2}{1+4 n}, n
\in \mathbb{N}$ and $b_n = \frac{2}{3+4 n}, n \in \mathbb{N}$, convergint to zero from the right side. Then
\[
f(a_n) = {\sgn} \left(\sin \left( \frac{\pi}{\frac{2}{1+4 n}}\right)\right) = {\sgn} \left( \sin \left( \frac{\pi}{2} + 2 \pi n \right)\right)=1
\]
And
\[
f(b_n) = {\sgn} \left(\sin \left( \frac{\pi}{\frac{2}{3+4 n}}\right)\right) = {\sgn} \left( \sin \left( \frac{3 \pi}{2} + 2 \pi n \right)\right)=-1.
\]

We have shown that the limit of $f(x)$ for $x$ converging to zero from the right does not exist. One may prove that the limit from the left does not exist by considering sequences $-a_n$ and $-b_n$.
\end{solution}





\question Matrix $A$ is given by
\[
A=\begin{pmatrix}
3 & 2 & 0 \\
1 & 4 & 0 \\
0 & 0 & 10 \\
\end{pmatrix}.
\]

\begin{parts}

\part[6] Find the eigenvalues and eigenvectors of $A$;
\begin{solution}

First, we solve the equation $\det(A- \lambda I) = 0$. The roots are $\lambda_1 = 2$, $\lambda_2 = 5$, $\lambda_3=10$.

The eigenvectors are
\[
v_1 = \begin{pmatrix}
2 \\
-1 \\
0 \\
\end{pmatrix}; \;
v_2 = \begin{pmatrix}
1 \\
1 \\
0 \\
\end{pmatrix}; \;
v_3 = \begin{pmatrix}
0 \\
0 \\
1 \\
\end{pmatrix};
\]

\end{solution}

\part[4] Find the eigenvalues of $4A^{-1} + 2I$, where $I$ is identity matrix.

\begin{solution}
Multiplication by 4 does not change eigenvalues. So we find eigenvalues of $A^{-1}+0.5I$. They are $a_1 = 0.5 + 0.5 =1$, $a_2 = 0.2 + 0.5 = 0.7$, $a_3=0.1 + 0.5 = 0.6$.
\end{solution}


\end{parts}



\question The characteristic polynomial of a matrix $B$ is given by $f(\lambda) = 6\lambda - 5\lambda^2 - \lambda^3$.


\begin{parts}

\part[6] Find dimensions of $B$, $\rank B$, $\det B$, sum of diagonal elements of $B$;


\begin{solution}
The highest power of $\lambda$ is 3, so the dimension of $B$ is $3\times 3$. We can solve for $\lambda$ and find $\lambda_1=0$, $\lambda_2=1$, $\lambda_3=-6$. So, $\det B =0$ and $\trace B =\sum b_{ii} = 0 + 1 -6 = -5$.
\end{solution}

\part[2] Suppose additionaly that $B$ is symmetric.
Can we find random variables with covariance matrix $B$?


\begin{solution}

The quadratic form $B$ is indefinite, so $B$ is not a valid covariance matrix.

\end{solution}


\part[2] Suppose additionaly that $B$ is symmetric.
How many solutions does equation $v^TBv=-2018$ has?


\begin{solution}

The quadratic form $B$ is indefinite, so the equation has infinitely many solutions.

\end{solution}

\end{parts}


\question[10] Solve the following differential equations
\begin{parts}
  \part[5] $y'' - 2y' - 8y = 0$,

\begin{solution}
  Let us solve the characteristic equation
  \[
  \lambda ^2 - 2\lambda  - 8 = 0 \text{}
  \]
  corresponding to the differential equation (a). It is easy to see that $\lambda_1 = 4$ and $\lambda_2 = -2$ are the solutions of this characteristic equation. Hence, the general solution of the differential equation (a) is
  \[
  y_a(x) = {C_1}{e^{4x}} + {C_2}{e^{ - 2x}} \text{, \quad where $C_1, \, C_2 \in \mathbb{R}$.}
  \]
\end{solution}

  \part[5] $y'' - 2y' - 8y = e^x - 8\cos 2x$.

\begin{solution}
  To solve the differential equation (b) we have to find the particular solutions of the following differential equations
  \begin{equation}\label{djsk73}
    y'' - 2y' - 8y = e^x \text{,}
  \end{equation}
  and
  \begin{equation}\label{ayu61h}
    y'' - 2y' - 8y = - 8\cos 2x \text{.}
  \end{equation}

  We seek the particular solution of the differential equation (\ref{djsk73}) in the form
  \begin{equation}\label{vah5gs}
    y(x) = A e^x \text{.}
  \end{equation}
  Substituting expression (\ref{vah5gs}) into equation (\ref{djsk73}), we obtain $A = - 1 / 9$.

  We seek the particular solution of the differential equation (\ref{ayu61h}) in the form
  \begin{equation}\label{h72hjs}
  y(x) = {B_1} \cos 2x + {B_2} \sin 2x \text{.}
  \end{equation}
  Substituting expression (\ref{h72hjs}) into equation (\ref{ayu61h}), we obtain $B_1 = 3 / 5$ and $B_2 = 1/5$.

  The particular solution of the differential equation (b) is a sum of particular solutions of equations (\ref{djsk73}) and (\ref{ayu61h}). Thus, the particular solution of the differential equation (b) is
  \[
      y(x) = -\frac{1}{9} e^x + \frac{3}{5} \cos 2x + \frac{1}{5} \sin 2x \text{.}
  \]
  Therefore, the general solution of equation (b) is
  \[
      y_b(x) = {C_1}{e^{4x}} + {C_2}{e^{ - 2x}} -\frac{1}{9} e^x + \frac{3}{5} \cos 2x + \frac{1}{5} \sin 2x \text{, \quad where $C_1, \, C_2 \in \mathbb{R}$.}
  \]
\end{solution}


\end{parts}





\question[10] Find the points of maximum of the function
\[
F\left(u,v\right)=\sqrt{u}\left(\sqrt{u}-2\right)-\sqrt{v}\left(\sqrt{v}-2\right),
\]
given that  $\sqrt{u}\le 2,\ \sqrt{v}\le 2$


\begin{solution}
  \begin{enumerate}
  \item  We use the change of variables $x=\sqrt{u},\ y=\sqrt{v}$. Using algebraic manipulations we transform $G$ into: $G\left(x,y\right)={\left(x-1\right)}^2-{\left(y-1\right)}^2+3$. Now constraints have the form $x\in \left[0,2\right],\ y\in \left[0,2\right]$

  \item  First we check for internal extrema:

  \[\frac{\partial G\left(x,y\right)}{\partial x}=2\left(x-1\right),\ \frac{\partial G\left(x,y\right)}{\partial y}=-2\left(y-1\right),\ \frac{{\partial }^2G\left(x,y\right)}{\partial x\partial y}=\left( \begin{array}{cc}
  2 & 0 \\
  0 & -2 \end{array}
  \right)\]


Using Sylvester's criterion we find that the point $(1,1)$ is not a maximum of $G(x,y)$.

\item  Now we check for corner solutions.

Line $\left\{x=0,\ y\in \left[0,2\right]\right\},\ G\left(0,y\right)=1-{\left(y-1\right)}^2$. At the point (0, 1) we have a \textbf{\underbar{maximum}} equal to 1, with values at the borders equal to $0$.

Line $\left\{y=0,\ x\in \left[0,2\right]\right\},\ G\left(x,0\right)={\left(x-1\right)}^2-1$. At the point (1, 0) we have a minimum equal to $(-1)$, with values at the borders equal to 0.

Line $\left\{x=2,\ y\in \left[0,2\right]\right\},\ G\left(0,y\right)=1-{\left(y-1\right)}^2$. At the point (2, 1) we have a \textbf{\underbar{maximum}} equal to 1, with values at the borders equal to $0$.

Line $\left\{y=2,\ x\in \left[0,2\right]\right\},\ G\left(x,2\right)={\left(x-1\right)}^2-1$. At the point (1, 2) we have a minimum equal to $(-1)$, with values at the borders equal to 0.

\item  Now we do inverse substitution.
\end{enumerate}

Answer: Maxumum points: (0, 1) и (4, 1)


\end{solution}

\newpage

\question
There are three coins in the bag. Two coins are unbiased, and for the third coin the probability of «head» is equal to $0.8$. James Bond chooses one coin at random from the bag and tosses it

\begin{parts}
\part[5] What is the probability that it will show «head»?
\begin{solution}
\[
\P(B) = \P(\text{head}) = \frac{2}{3} \cdot 0.5 + \frac{1}{3} \cdot 0.8 = \frac{18}{30} = \frac{3}{5} = 0.6
\]
\end{solution}

\part[5] What is the conditional probability that the coin is unbiased if it shows «head»?
\begin{solution}
\[
\P(A|B) = \frac{\P(A\cap B)}{\P(B)} = \frac{\frac{2}{3} \cdot 0.5}{0.6} = \frac{10}{18} = \frac{5}{9}
\]
\end{solution}

\end{parts}


\question
The pair of random variables $X$ and $Y$ with $\E(X)=0$ and $\E(Y)=1$ has the following covariance matrix


\[
\begin{pmatrix}
10 & -2 \\
-2 & 9 \\
\end{pmatrix}.
\]



\begin{parts}
\part[5] Find $\Var(X+Y)$, $\Corr(X, Y)$, $\Cov(X - 2Y +1 , 7 + X + Y)$
\begin{solution}
  \[
  \Var(X+Y) = 10 + 9 - 2 \cdot 2 = 15
  \]
  \[
  \Corr(X, Y) = \frac{-2}{\sqrt{10\cdot 9}}
  \]
  \[
  \Cov(X - 2Y +1 , 7 + X + Y) = \Var(X) - 2\Cov(Y, X) + \Cov(X, Y) - 2 \Var(Y) = 10 - (-2) - 18 = -6
  \]
\end{solution}

\part[5] Find the value of $a$ if it is known that $X$ is independent of $Y - aX$.
\begin{solution}
  For independent variables the covariance is equal to zero: $\Cov(X, Y - aX) = 0$. Hence, $a = \frac{\Cov(X, Y)}{\Var(X)} = -0.2$.
\end{solution}

\end{parts}



\question You have  height measurements of a random sample of 100 persons, $y_1$, \ldots, $y_{100}$. It is known that $\sum_{i=1}^{100} y_i = 15800$ and $\sum_{i=1}^{100} y_i^2 = 2530060$.

\begin{parts}
\part[3] Calculate unbiased estimate of population mean and population variance of the height

\begin{solution}
Sample mean: $\bar y = 15800/100=158$.

Unbiased estimate of the variance:
\[
\hat\sigma^2 = \frac{\sum (y_i-\bar y )^2}{n-1}=\frac{\sum y_i^2 - n \bar y^2}{n-1}=340
\]
\end{solution}
\part[3] At 4\% significance test the null-hypothesis that the population mean is equal to 155 cm, against two-sided alternative.

\begin{solution}
Observed value of $Z$-statistics
\[
Z_{obs}=\frac{158-155}{\sqrt{340}/\sqrt{100}}=1.63
\]

Critical value of $Z$-statistics $Z_{crit}=2.05$.

Conclusion: hypothesis $H_0$ is not rejected.
\end{solution}
\part[2] Find the p-value
\begin{solution}
Using tables we find that the area under the curve to the right of $1.63$ is approximately 5\%. Hense p-value is equal to $10\%$.
\end{solution}
\part[2] Find the 96\% confidence interval for the population mean
\begin{solution}
The confidence interval has the form
\[
[158 - 2.05 \cdot \sqrt{340/100} ; 158 + 2.05 \cdot \sqrt{340/100} ]
\]
Finally: $[154.2;161.8]$
\end{solution}
\end{parts}



\question Let $X = (X_1, \, \ldots, \, X_n)$ be a random sample from normal distribution with zero mean and unknown variance $\sigma^2 > 0$.

If $\xi \sim \cN(0, \, \sigma^2)$, then $\E[\xi^4] = 3\sigma^2$.

\begin{parts}
\part[2] Derive the log-likelihood function of a random sample $X$.
\begin{solution}
  The likelihood function of a random sample $X$ is
  \[
  L({x_1},\; \ldots ,\;{x_n};{\sigma ^2}) = {f_{{X_1},\; \ldots ,\;{X_n}}}({x_1},\; \ldots ,\;{x_n};{\sigma ^2}) =
  \]
  \[
  = \prod\nolimits_{i = 1}^n {{f_{{X_i}}}({x_i};{\sigma ^2})}  = \prod\nolimits_{i = 1}^n {\frac{1}{{\sqrt {2\pi {\sigma ^2}} }}{e^{ - \frac{{x_i^2}}{{2{\sigma ^2}}}}}}  = {(2\pi )^{ - n/2}}{({\sigma ^2})^{ - n/2}}{e^{ - \frac{{\sum\nolimits_{i = 1}^n {x_i^2} }}{{2{\sigma ^2}}}}} \text{.}
  \]
  Hence, the log-likelihood function of a random sample $X$ is
  \[
  l({x_1},\; \ldots ,\;{x_n};{\sigma ^2}): = \ln \mathcal{L}({x_1},\; \ldots ,\;{x_n};{\sigma ^2}) =  - \frac{n}{2}\ln (2\pi ) - \frac{n}{2}\ln ({\sigma ^2}) - \frac{{\sum\nolimits_{i = 1}^n {x_i^2} }}{{2{\sigma ^2}}} \text{.}
  \]
\end{solution}

\part[2] Find the estimator of the parameter $\sigma^2$ using maximum likelihood method.

\begin{solution}
  Let us write the likelihood equation:
  \[
  \frac{{\partial l({x_1},\; \ldots ,\;{x_n};{\sigma ^2})}}{{\partial ({\sigma ^2})}} =  - \frac{n}{{2{\sigma ^2}}} + \frac{{\sum\nolimits_{i = 1}^n {x_i^2} }}{{2{\sigma ^4}}} = 0 \text{.}
  \]
  Solving this equation with respect to $\sigma^2$, we find $\sigma^2 = \sum_{i=1}^{n}x_i^2$. Hence, the maximum likelihood  estimator of the parameter $\sigma^2$ is ${\widehat{{\sigma ^2}}_{ML}} = \tfrac{1}{n}\sum\nolimits_{i = 1}^n {X_i^2}$.
\end{solution}

\part[2] Using the realization of a random sample $x = (1, \, -2, \, 0, \, 1)$ find the maximum likelihood estimate of the parameter $\sigma^2$ derived in (b).

\begin{solution}
  The maximum likelihood estimate of the parameter $\sigma^2$ is
  \[
  {\widehat{{\sigma ^2}}_{ML}} = \tfrac{1}{4}\left( {{1^2} + {{( - 2)}^2} + {0^2} + {1^2}} \right) = \tfrac{3}{2} \text{.}
  \]
\end{solution}

\part[2] Find the Fisher information $I_n(\sigma^2)$ about the parameter $\sigma^2$ contained in $n$ observations of a random sample.

\begin{solution}
  Considering that $l({x_1};{\sigma ^2}) =  - \tfrac{1}{2}\ln (2\pi ) - \tfrac{1}{2}\ln ({\sigma ^2}) - \tfrac{{x_1^2}}{{2{\sigma ^2}}}$, we arrive at
  \[
  \frac{{\partial l({x_1};{\sigma ^2})}}{{\partial ({\sigma ^2})}} =  - \frac{1}{{2{\sigma ^2}}} + \frac{{x_1^2}}{{2{\sigma ^4}}} = \frac{{x_1^2 - {\sigma ^2}}}{{2{\sigma ^4}}} \text{.}
  \]
  Therefore, the Fisher information $I_1(\sigma^2)$ about the parameter $\sigma^2$ contained in a single observation of a random sample is
  \[
  {I_1}({\sigma ^2}) = \E\left[ {{{\left( {\frac{{\partial l({X_1};{\sigma ^2})}}{{\partial ({\sigma ^2})}}} \right)}^2}} \right] = \E\left[ {\frac{{{{\left( {X_1^2 - {\sigma ^2}} \right)}^2}}}{{4{\sigma ^8}}}} \right] = \frac{{\E\left[ {{{\left( {X_1^2 - {\sigma ^2}} \right)}^2}} \right]}}{{4{\sigma ^8}}}\mathop  = \limits^{\E[X_1^2] = {\sigma ^2}}
  \]
  \[
  = \frac{{\operatorname{D} (X_1^2)}}{{4{\sigma ^8}}} = \frac{{\E[X_1^4] - {{\left( {\E[X_1^2]} \right)}^2}}}{{4{\sigma ^8}}} = \frac{{3{\sigma ^4} - {{\left( {{\sigma ^2}} \right)}^2}}}{{4{\sigma ^8}}} = \frac{{2{\sigma ^4}}}{{4{\sigma ^8}}} = \frac{1}{{2{\sigma ^4}}} \text{.}
  \]
  Thus, ${I_n}({\sigma ^2}) = n \cdot {I_1}({\sigma ^2}) = \frac{n}{{2{\sigma ^4}}}$.

\end{solution}


\part[2] Is the estimator $\widehat{\sigma^2} = \tfrac{1}{n}\sum_{i=1}^{n}X_i^2$ an unbiased estimator of the parameter $\sigma^2$?

\begin{solution}
  The estimator $\widehat{\sigma^2} = \tfrac{1}{n}\sum_{i=1}^{n}X_i^2$ is unbiased, since
  \[
  \E[\widehat{{\sigma ^2}}] = \E\left[ {\tfrac{1}{n}\sum\nolimits_{i = 1}^n {X_i^2} } \right] = \tfrac{1}{n}\sum\nolimits_{i = 1}^n {\E[X_i^2]}  = \tfrac{1}{n}\sum\nolimits_{i = 1}^n {{\sigma ^2}}  = \tfrac{1}{n}n{\sigma ^2} = {\sigma ^2} \text{.}
  \]
\end{solution}


\part[2] Is the estimator $\widehat{\sigma^2} = \tfrac{1}{n}\sum_{i=1}^{n}X_i^2$ an efficient estimator of the parameter $\sigma^2$?

\begin{solution}
  The estimator $\widehat{\sigma^2} = \tfrac{1}{n}\sum_{i=1}^{n}X_i^2$ is efficient, as
  \[
  \operatorname{D} (\widehat{{\sigma ^2}}) = \operatorname{D} \left( {\tfrac{1}{n}\sum\nolimits_{i = 1}^n {X_i^2} } \right) = \tfrac{1}{{{n^2}}}\sum\nolimits_{i = 1}^n {\operatorname{D} (X_i^2)}  = \tfrac{1}{{{n^2}}}\sum\nolimits_{i = 1}^n {2{\sigma ^4}}  = \tfrac{1}{{{n^2}}}n2{\sigma ^4} = \frac{{2{\sigma ^4}}}{n} \text{,}
  \]
  and
  \[
  I_n^{ - 1}({\sigma ^2}) = {\left( {\frac{n}{{2{\sigma ^2}}}} \right)^{ - 1}} = \frac{{2{\sigma ^2}}}{n} = \operatorname{D} (\widehat{{\sigma ^2}}) \text{.}
  \]

\end{solution}



\part[2] Is the estimator $\widehat{\sigma^2} = \tfrac{1}{n}\sum_{i=1}^{n}X_i^2$ a consistent estimator of the parameter $\sigma^2$?

\begin{solution}
  As random variables $X_1^2, \ldots, X_n^2, \ldots$ are independent, have the same distribution, and  have finite means, then the we can apply the law of large numbers, according to which we have
  \[
  \tfrac{1}{n}\sum\nolimits_{i = 1}^n {X_i^2} \, \mathop  \to \limits^\mathbb{P} \,  \E[X_i^2] = {\sigma ^2} \text{\quad \; as $n \rightarrow \infty$.}
  \]
\end{solution}



\part[2] Find the following probability limit $\mathrm{plim}_{n \rightarrow \infty}\sqrt{\tfrac{1}{n}\sum_{i=1}^{n}X_i^2}$.

\begin{solution}
  As $\tfrac{1}{n}\sum\nolimits_{i = 1}^n {X_i^2} \mathop  \to \limits^\mathbb{P} \E[X_i^2] = {\sigma ^2}$, and the function $g(x) = \sqrt{x}$ is continuous, by Slutsky's theorem we derive
  \[
  \sqrt {\tfrac{1}{n}\sum\nolimits_{i = 1}^n {X_i^2} }  = g\Bigl(\tfrac{1}{n}\sum\nolimits_{i = 1}^n {X_i^2} \Bigr) \, \mathop  \to \limits^\mathbb{P} \,  g({\sigma ^2}) = \sqrt {{\sigma ^2}}  = \sigma \text{\quad \; as $n \rightarrow \infty$.}
  \]
\end{solution}



\end{parts}



\end{questions}



\begin{figure}[b]
\caption{Distribution function of a standard normal random variable}
  \begin{minipage}[b]{0.35\linewidth}
    \centering
    \begin{tikzpicture}
% define normal distribution function 'normaltwo'
    \def\normaltwo{\x,{4*1/exp(((\x-3)^2)/2)}}

% input y parameter
    \def\y{4.4}

% this line calculates f(y)
    \def\fy{4*1/exp(((\y-3)^2)/2)}

% Shade orange area underneath curve.
    \fill [fill=gray!30] (2.6,0) -- plot[domain=0:4.4] (\normaltwo) -- ({\y},0) -- cycle;

% Draw and label normal distribution function
    \draw[domain=0:6] plot (\normaltwo) node[right] {};

% Add dashed line dropping down from normal.
    \draw[dashed] ({\y},{\fy}) -- ({\y},0) node[below] {$x$};

% Optional: Add axis labels
%    \draw (-.2,2.5) node[left] {$f_Y(u)$};
    \draw (3,2) node[below] {$F(x)$};

% Optional: Add axes
    \draw[->] (0,0) -- (6.2,0) node[right] {};
%    \draw[->] (0,0) -- (0,5) node[above] {};

\end{tikzpicture}
%    \rule{6cm}{6cm} %to simulate an actual figure
\par\vspace{0pt}
  \end{minipage}%
  \begin{minipage}[b]{0.60\linewidth}
    \centering
\begin{tabular}{rr|rr|rr|rr}
  \hline
$x$ & $F(x)$ & $x$ & $F(x)$ & $x$ & $F(x)$ & $x$ & $F(x)$ \\
  \hline
0.050 & 0.520 & 0.750 & 0.773 & 1.450 & 0.926 & 2.150 & 0.984 \\
  0.100 & 0.540 & 0.800 & 0.788 & 1.500 & 0.933 & 2.200 & 0.986 \\
  0.150 & 0.560 & 0.850 & 0.802 & 1.550 & 0.939 & 2.250 & 0.988 \\
  0.200 & 0.579 & 0.900 & 0.816 & 1.600 & 0.945 & 2.300 & 0.989 \\
  0.250 & 0.599 & 0.950 & 0.829 & 1.650 & 0.951 & 2.350 & 0.991 \\
  0.300 & 0.618 & 1.000 & 0.841 & 1.700 & 0.955 & 2.400 & 0.992 \\
  0.350 & 0.637 & 1.050 & 0.853 & 1.750 & 0.960 & 2.450 & 0.993 \\
  0.400 & 0.655 & 1.100 & 0.864 & 1.800 & 0.964 & 2.500 & 0.994 \\
  0.450 & 0.674 & 1.150 & 0.875 & 1.850 & 0.968 & 2.550 & 0.995 \\
  0.500 & 0.691 & 1.200 & 0.885 & 1.900 & 0.971 & 2.600 & 0.995 \\
  0.550 & 0.709 & 1.250 & 0.894 & 1.950 & 0.974 & 2.650 & 0.996 \\
  0.600 & 0.726 & 1.300 & 0.903 & 2.000 & 0.977 & 2.700 & 0.997 \\
  0.650 & 0.742 & 1.350 & 0.911 & 2.050 & 0.980 & 2.750 & 0.997 \\
  0.700 & 0.758 & 1.400 & 0.919 & 2.100 & 0.982 & 2.800 & 0.997 \\
   \hline
\end{tabular}
\par\vspace{0pt}
\end{minipage}
\label{fig:test}
\end{figure}

\begin{flushright}
Good luck!
\end{flushright}

\end{document}
